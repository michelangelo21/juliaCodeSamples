\documentclass[11pt]{article}
\usepackage[T1]{fontenc}
\usepackage{fontspec}
\usepackage{polyglossia}
\usepackage{amsmath}
%\usepackage{amssymb}
%\topmargin -2cm
%\textheight 23cm
%\oddsidemargin -1cm
%\evensidemargin -1cm
%\textwidth 17.5cm
\usepackage[lmargin=1cm,rmargin=1.5cm,tmargin=1cm,bmargin=2cm]{geometry}
\usepackage{natbib}
\usepackage{graphicx}
\usepackage{tikz}
\usepackage{pgfplots}
% \usepgfplotslibrary{external} 
% \tikzexternalize[prefix=figures/]
\usetikzlibrary{intersections}
\usetikzlibrary{backgrounds}
\usepackage{minted}
\usepackage{listings}

\usepackage[mathletters]{ucs}
\usepackage[utf8x]{inputenc}

\usepackage{unicode-math}


\usepackage{siunitx}

\setmonofont{DejaVu Sans Mono}[Scale=MatchLowercase]


\pgfplotsset{compat=newest}

\title{Próg Fredericksza w ciekłym krysztale nematycznym}
\author{Michał Łukomski}
\date{Styczeń 2021}

\begin{document}
    \maketitle

    Symulacje zostały przeprowadzone na sieci o rozmiarze $L^2$, gdzie $L=20$. 
    $\xi^*$ oraz $E^*$ zostały wyrażone w jednostkach zredukowanych $\xi^* = \xi / (k_B T)$, $E^* = E / (k_B T)$.
    
    Symulacje startowano z konfiguracją uporządkowaną - wszystkie kąty zostały ustawione na $\SI{0}{\degree}$.

    Sieci zostały poddane ewolucji przez $30 000 MCS$, ($1MCS$ = iteracyjne przejście po wszystkich węzłach układu) a następnie przez kolejne $200 000 MCS$ podczas których $n_{eff}$ (a w jednym przejściu także $<\cos^2(\phi)>(z)$) było próbkowane co $100 MCS$. Otrzymano więc próbki o rozmiarach 2000, z których obliczono średnie.
    Wyniki przedstawiono na wykresach. 

    Kąty mogły przyjmować całkowite wartości z przedziału $[\SI{-90}{\degree} : \SI{90}{\degree}]$.

    Zakres natężenia zewnętrznego (pionowego) pola użyty w symulacjach to
    \begin{itemize}
        \item $[0 : 3.5]$ ze skokiem $\Delta E^* = 0.001$.
    \end{itemize}

    Inne parametry wykorzystane we wszystkich symulacjach to
    \begin{itemize}
        \item $\xi^* = 20$
        \item $n_0 = 1.5$
        \item $n_e = 1.7$
        \item $\Delta \phi = \SI{10}{\degree}$
    \end{itemize}

    Próg Fredericksza odczytano z wykresu jako $E_F^* = 0.73$

\begin{figure}
    \centering
    \begin{tikzpicture}[/tikz/background rectangle/.style={fill={rgb,1:red,1.0;green,1.0;blue,1.0}, draw opacity={1.0}}, show background rectangle]
\begin{axis}[point meta max={nan}, point meta min={nan}, legend cell align={left}, title={Effective refracting index as a function of reduced external electric field - ordered initial conditions}, title style={at={{(0.5,1)}}, anchor={south}, font={{\fontsize{8 pt}{10.4 pt}\selectfont}}, color={rgb,1:red,0.0;green,0.0;blue,0.0}, draw opacity={1.0}, rotate={0.0}}, legend style={color={rgb,1:red,0.0;green,0.0;blue,0.0}, draw opacity={1.0}, line width={1}, solid, fill={rgb,1:red,1.0;green,1.0;blue,1.0}, fill opacity={1.0}, text opacity={1.0}, font={{\fontsize{8 pt}{10.4 pt}\selectfont}}, text={rgb,1:red,0.0;green,0.0;blue,0.0}, at={(1.02, 1)}, anchor={north west}}, axis background/.style={fill={rgb,1:red,1.0;green,1.0;blue,1.0}, opacity={1.0}}, anchor={north west}, xshift={1.0mm}, yshift={-1.0mm}, width={145.4mm}, height={99.6mm}, scaled x ticks={false}, xlabel={$E^*$}, x tick style={color={rgb,1:red,0.0;green,0.0;blue,0.0}, opacity={1.0}}, x tick label style={color={rgb,1:red,0.0;green,0.0;blue,0.0}, opacity={1.0}, rotate={0}}, xlabel style={at={(ticklabel cs:0.5)}, anchor=near ticklabel, font={{\fontsize{11 pt}{14.3 pt}\selectfont}}, color={rgb,1:red,0.0;green,0.0;blue,0.0}, draw opacity={1.0}, rotate={0.0}}, xmajorgrids={true}, xmin={-0.105}, xmax={3.605}, xtick={{0.0,1.0,2.0,3.0}}, xticklabels={{$0$,$1$,$2$,$3$}}, xtick align={inside}, xticklabel style={font={{\fontsize{8 pt}{10.4 pt}\selectfont}}, color={rgb,1:red,0.0;green,0.0;blue,0.0}, draw opacity={1.0}, rotate={0.0}}, x grid style={color={rgb,1:red,0.0;green,0.0;blue,0.0}, draw opacity={0.1}, line width={0.5}, solid}, extra x ticks={{0.2,0.4,0.6,0.8,1.2,1.4,1.6,1.8,2.2,2.4,2.6,2.8,3.2,3.4,3.6}}, extra x tick labels={}, extra x tick style={grid={major}, x grid style={color={rgb,1:red,0.0;green,0.0;blue,0.0}, draw opacity={0.05}, line width={0.5}, solid}, major tick length={0.1cm}}, axis x line*={left}, x axis line style={color={rgb,1:red,0.0;green,0.0;blue,0.0}, draw opacity={1.0}, line width={1}, solid}, scaled y ticks={false}, ylabel={$<n_{eff}>$}, y tick style={color={rgb,1:red,0.0;green,0.0;blue,0.0}, opacity={1.0}}, y tick label style={color={rgb,1:red,0.0;green,0.0;blue,0.0}, opacity={1.0}, rotate={0}}, ylabel style={at={(ticklabel cs:0.5)}, anchor=near ticklabel, font={{\fontsize{11 pt}{14.3 pt}\selectfont}}, color={rgb,1:red,0.0;green,0.0;blue,0.0}, draw opacity={1.0}, rotate={0.0}}, ymajorgrids={true}, ymin={1.515671212842574}, ymax={1.7053687996259443}, ytick={{1.55,1.6,1.6500000000000001,1.7000000000000002}}, yticklabels={{$1.55$,$1.60$,$1.65$,$1.70$}}, ytick align={inside}, yticklabel style={font={{\fontsize{8 pt}{10.4 pt}\selectfont}}, color={rgb,1:red,0.0;green,0.0;blue,0.0}, draw opacity={1.0}, rotate={0.0}}, y grid style={color={rgb,1:red,0.0;green,0.0;blue,0.0}, draw opacity={0.1}, line width={0.5}, solid}, extra y ticks={{1.52,1.53,1.54,1.56,1.57,1.58,1.59,1.61,1.62,1.6300000000000001,1.6400000000000001,1.6600000000000001,1.6700000000000002,1.6800000000000002,1.6900000000000002}}, extra y tick labels={}, extra y tick style={grid={major}, y grid style={color={rgb,1:red,0.0;green,0.0;blue,0.0}, draw opacity={0.05}, line width={0.5}, solid}, major tick length={0.1cm}}, axis y line*={left}, y axis line style={color={rgb,1:red,0.0;green,0.0;blue,0.0}, draw opacity={1.0}, line width={1}, solid}]
    \addplot[color={rgb,1:red,0.0;green,0.6056;blue,0.9787}, name path={22b6c70f-2634-4310-9d2d-48be4be1b7b3}, only marks, draw opacity={1.0}, line width={0}, solid, mark={*}, mark size={1.5 pt}, mark repeat={1}, mark options={color={rgb,1:red,0.0;green,0.0;blue,0.0}, draw opacity={0.0}, fill={rgb,1:red,0.0;green,0.6056;blue,0.9787}, fill opacity={1.0}, line width={0.75}, rotate={0}, solid}]
        table[row sep={\\}]
        {
            \\
            0.259  1.6978202464730645  \\
            2.234  1.5432443898802009  \\
            2.462  1.5383769825963358  \\
            2.166  1.5447022862925264  \\
            2.851  1.5295383620852943  \\
            2.154  1.5450267722888336  \\
            0.785  1.6814900109469648  \\
            0.756  1.69100710568753  \\
            3.387  1.5211316828061647  \\
            2.509  1.5373422409031319  \\
            0.418  1.6977476048779925  \\
            2.378  1.5401454657206994  \\
            1.719  1.5570325572202053  \\
            3.463  1.521067374724601  \\
            3.169  1.5217185735994163  \\
            3.109  1.522275000484881  \\
            2.388  1.5399601418100552  \\
            2.993  1.5254522795662575  \\
            0.761  1.6853694719845065  \\
            0.721  1.6928651220999635  \\
            0.171  1.6978408641974516  \\
            2.816  1.5304141459190366  \\
            3.225  1.5214681840023925  \\
            1.672  1.558638072916439  \\
            0.58  1.697431841287041  \\
            0.437  1.697535089291652  \\
            2.819  1.5303336230704467  \\
            3.299  1.5212633768442638  \\
            1.249  1.5816449394200904  \\
            2.767  1.5315778755662572  \\
            0.127  1.6979118250127356  \\
            0.626  1.6968046337366396  \\
            1.088  1.598142538001969  \\
            0.99  1.6134737485222188  \\
            0.935  1.6248852311262605  \\
            2.116  1.545879132199836  \\
            1.512  1.5652591684268056  \\
            3.069  1.5231191334393779  \\
            3.063  1.5231141618417647  \\
            0.55  1.6973954829333109  \\
            2.191  1.5442333208245147  \\
            0.892  1.634193949293939  \\
            0.889  1.636000468811691  \\
            1.255  1.580769010072519  \\
            2.963  1.5264259190499856  \\
            0.197  1.6977796215486143  \\
            2.276  1.5423473669583827  \\
            1.276  1.579520609290797  \\
            2.511  1.5373575699834567  \\
            1.322  1.5759139778997642  \\
            1.001  1.6110549513241263  \\
            3.263  1.5213535979373254  \\
            2.565  1.5361680391623342  \\
            3.034  1.5241884693467849  \\
            3.044  1.52385672976882  \\
            3.466  1.5210608334942346  \\
            2.255  1.5428047402834506  \\
            0.918  1.6279753377920279  \\
            0.172  1.6978673159881068  \\
            0.762  1.684697154321532  \\
            2.315  1.5414592073289883  \\
            2.513  1.5372940551765564  \\
            1.288  1.5782783726490517  \\
            3.254  1.5213599885982192  \\
            2.891  1.528513556995719  \\
            1.423  1.5696567333092122  \\
            0.147  1.6978859459601532  \\
            0.499  1.697537490174692  \\
            2.26  1.5426637879145515  \\
            2.246  1.542885604225657  \\
            0.215  1.697801163218106  \\
            0.834  1.655478565721448  \\
            1.805  1.5541287194525162  \\
            3.313  1.5212223120388002  \\
            2.147  1.5451238081287368  \\
            2.463  1.5382808556610081  \\
            2.667  1.5339889651620753  \\
            1.678  1.5583940496264161  \\
            3.321  1.521220924186742  \\
            0.181  1.697853030639084  \\
            1.574  1.562399301306102  \\
            3.195  1.5215280988725124  \\
            3.174  1.5217008279718611  \\
            2.677  1.5337353974407006  \\
            0.125  1.6979083301156115  \\
            0.462  1.6976805017782677  \\
            2.989  1.5255283538495183  \\
            3.053  1.5236112345670785  \\
            0.982  1.6146456965708114  \\
            0.007  1.6978737761283795  \\
            3.454  1.5210711732302469  \\
            2.444  1.5387046447831252  \\
            1.289  1.5783650956410091  \\
            3.036  1.5239982994114518  \\
            1.369  1.5728868776152258  \\
            1.584  1.5619857797409313  \\
            3.345  1.5211978254217957  \\
            2.982  1.5257827572689309  \\
            2.286  1.542059403602439  \\
            1.999  1.548820417156781  \\
            0.348  1.6977777549454496  \\
            0.038  1.6978827453289689  \\
            2.139  1.545441939350156  \\
            0.743  1.6922672233888825  \\
            1.617  1.5607135841816013  \\
            3.206  1.52152865383097  \\
            0.227  1.6979265743871688  \\
            2.57  1.536002708907576  \\
            1.534  1.5643890335512758  \\
            3.428  1.521096100594318  \\
            2.108  1.5461608223095098  \\
            1.899  1.5514864263666022  \\
            3.315  1.5212453844800267  \\
            0.902  1.631956319749883  \\
            1.471  1.5672146911790807  \\
            2.066  1.5470703816426543  \\
            1.417  1.5700918206727044  \\
            1.142  1.5914990561348072  \\
            3.478  1.5210541752928644  \\
            1.722  1.5568537344260558  \\
            1.4  1.5710392807783309  \\
            0.856  1.6465024890870328  \\
            2.952  1.5266053395740853  \\
            2.568  1.5361267994667624  \\
            0.089  1.6978798673726792  \\
            1.274  1.579643945659493  \\
            1.982  1.5491818419464911  \\
            2.942  1.5269710024704994  \\
            0.886  1.637090311338126  \\
            0.091  1.6979179742397437  \\
            1.478  1.5668995336109859  \\
            1.773  1.555186247986811  \\
            2.173  1.544654603773181  \\
            2.383  1.5399974797171392  \\
            0.062  1.6979081585528928  \\
            3.248  1.521384676603107  \\
            1.518  1.5649518390662651  \\
            2.591  1.5355414693524456  \\
            2.134  1.5454682103428536  \\
            0.27  1.697814324229027  \\
            1.191  1.586350115647175  \\
            2.863  1.5291246598925194  \\
            1.521  1.5646615307181795  \\
            3.353  1.5211779905256044  \\
            3.196  1.5215571546953433  \\
            2.544  1.5366407062764398  \\
            3.395  1.521129760688739  \\
            1.166  1.5891396721892832  \\
            0.266  1.6978426653642884  \\
            1.753  1.555825284784931  \\
            0.955  1.6201632981648695  \\
            0.473  1.6975405210828378  \\
            0.106  1.6978478075774113  \\
            0.907  1.6314816159972396  \\
            0.775  1.6789902298096773  \\
            2.236  1.543211885691929  \\
            0.087  1.6978798507807669  \\
            1.18  1.5877143205940318  \\
            0.689  1.695784839436051  \\
            3.337  1.5212113124029774  \\
            0.917  1.6281671720545803  \\
            0.949  1.6213732723210854  \\
            2.7  1.5332120570650452  \\
            1.825  1.553627088820441  \\
            1.394  1.5713440948208686  \\
            1.317  1.5761028984544398  \\
            3.298  1.5212593483307542  \\
            1.062  1.600990981816864  \\
            2.856  1.529437693744415  \\
            3.402  1.5211249674332745  \\
            2.909  1.5278672754313056  \\
            2.075  1.5469120385521777  \\
            1.657  1.5592803678961993  \\
            3.25  1.5213490549468591  \\
            3.42  1.5211114181744967  \\
            2.759  1.5318286933759828  \\
            2.252  1.5428036596962236  \\
            1.932  1.550538977685741  \\
            2.351  1.5407131539410153  \\
            2.398  1.539752430165966  \\
            2.739  1.5323175370910187  \\
            2.248  1.542891617207868  \\
            1.367  1.5731653331421278  \\
            1.252  1.5812564728719893  \\
            2.115  1.5458632164204407  \\
            3.495  1.5210454976115355  \\
            2.441  1.5387407711330545  \\
            0.504  1.6975198011359027  \\
            3.398  1.521126111967818  \\
            2.146  1.5452161069199746  \\
            3.47  1.5210702907140883  \\
            0.164  1.6978510342695674  \\
            2.983  1.5259109401127031  \\
            2.152  1.545108219526764  \\
            0.043  1.6979051806883285  \\
            0.603  1.6973788570272927  \\
            2.142  1.5452629951347414  \\
            0.508  1.6974484697415366  \\
            0.965  1.6183932725585433  \\
            1.404  1.5707813364430732  \\
            2.308  1.5415858764764012  \\
            2.782  1.5312696585230081  \\
            2.847  1.5296332568258146  \\
            1.871  1.552206866578577  \\
            1.54  1.5639857039136003  \\
            0.522  1.6975474014409344  \\
            1.981  1.549227125459964  \\
            1.227  1.5831615269959602  \\
            1.099  1.5967153449480673  \\
            3.014  1.5246127510127105  \\
            2.708  1.5329312427918396  \\
            0.139  1.6978644032673689  \\
            0.62  1.6970688677304255  \\
            0.43  1.6976095863569158  \\
            1.023  1.6077584019428015  \\
            1.868  1.5523046637808067  \\
            2.556  1.5363410416296448  \\
            0.503  1.6976091677384229  \\
            2.436  1.538923484044123  \\
            0.755  1.6857695847585963  \\
            1.181  1.5874681234152634  \\
            2.772  1.5315067076813487  \\
            0.357  1.6976737647030495  \\
            3.039  1.5238771428725875  \\
            0.921  1.6272078818430133  \\
            2.294  1.5419397720186865  \\
            2.228  1.5433403362453673  \\
            2.161  1.5448540656225267  \\
            0.673  1.695828374806338  \\
            3.238  1.521410349293256  \\
            0.857  1.6459834766292745  \\
            1.612  1.5608377631513404  \\
            1.968  1.5496071216150566  \\
            1.529  1.5645892023943286  \\
            1.974  1.5493827525694486  \\
            1.754  1.5558529704716346  \\
            2.164  1.5447875970935039  \\
            2.218  1.5435618246803415  \\
            0.506  1.6975180122462972  \\
            0.4  1.697683661078072  \\
            2.742  1.5322019407602474  \\
            2.313  1.5414777841550407  \\
            2.216  1.5436264747576196  \\
            0.277  1.697822665171039  \\
            1.73  1.5566088574033023  \\
            2.384  1.5399745466054577  \\
            0.044  1.6978761778002753  \\
            1.21  1.5846972664051613  \\
            1.95  1.5500294556876473  \\
            3.118  1.522178571861527  \\
            3.061  1.523223719814995  \\
            0.861  1.644931520041318  \\
            2.89  1.5284517105215483  \\
            1.903  1.55134341468166  \\
            3.171  1.5217019622620855  \\
            2.461  1.5384066326183992  \\
            0.869  1.6422739357688325  \\
            0.577  1.6972876047966319  \\
            0.684  1.6956164784555474  \\
            0.646  1.696574256647858  \\
            3.314  1.521238838752401  \\
            1.217  1.583766478517585  \\
            2.215  1.5436743441300862  \\
            1.713  1.557147075604868  \\
            1.401  1.5711116190440821  \\
            1.531  1.5643139236455437  \\
            1.607  1.5611243767431715  \\
            2.97  1.5262490685901775  \\
            0.406  1.6976946783369558  \\
            0.572  1.6972050206699403  \\
            0.116  1.697836654153997  \\
            0.149  1.6978939918924298  \\
            0.435  1.6977211639379408  \\
            1.325  1.5756295195685135  \\
            1.694  1.5578231749435238  \\
            0.734  1.6936014178892767  \\
            0.431  1.6975570723545763  \\
            1.853  1.5527393215367833  \\
            2.88  1.5287113128267693  \\
            0.777  1.6821762551064312  \\
            1.887  1.5516824574149386  \\
            0.687  1.6957516613291497  \\
            2.487  1.537751477319869  \\
            0.195  1.6979161892520256  \\
            0.242  1.6978427776952982  \\
            0.897  1.632937454926271  \\
            2.907  1.5279928010063384  \\
            2.74  1.5323029460686646  \\
            3.252  1.5213730764474733  \\
            1.008  1.6096269734850763  \\
            0.354  1.6977249030704056  \\
            0.848  1.6501970551145038  \\
            3.062  1.5233013176065595  \\
            0.081  1.6978542255299598  \\
            2.014  1.5484304274284775  \\
            2.016  1.5483363540137285  \\
            2.102  1.546254791494004  \\
            0.852  1.6464574823725338  \\
            0.638  1.6967537887521227  \\
            2.633  1.5346634780772692  \\
            1.476  1.5669823363163962  \\
            0.512  1.6975417988845576  \\
            3.041  1.523912278489622  \\
            3.392  1.521138366808603  \\
            2.721  1.5326853937544664  \\
            1.379  1.5722830631447917  \\
            1.842  1.5530679439643404  \\
            2.478  1.5379624618694394  \\
            0.552  1.6973211404985136  \\
            2.13  1.5455658889930692  \\
            0.876  1.6392236213544786  \\
            0.158  1.6978568803890663  \\
            2.414  1.5393436999929413  \\
            3.048  1.5235731248963054  \\
            2.527  1.5369792185576914  \\
            3.308  1.5212352345962639  \\
            0.229  1.6978938206151697  \\
            0.898  1.6346603727696054  \\
            3.017  1.524594847642942  \\
            0.165  1.6978272521701276  \\
            1.554  1.563297226601684  \\
            0.44  1.6976717500545375  \\
            2.917  1.5277219099407748  \\
            2.632  1.5346937367749516  \\
            1.874  1.5521402646571962  \\
            2.671  1.5337758042258296  \\
            0.252  1.6978582180997044  \\
            1.291  1.5782792768762004  \\
            1.774  1.555127388048749  \\
            2.231  1.5432724682053198  \\
            1.514  1.5650566950467668  \\
            0.836  1.6540821733812208  \\
            1.625  1.5604261431051047  \\
            0.278  1.6978195617622327  \\
            0.244  1.6978741706330227  \\
            2.508  1.537406609106021  \\
            1.262  1.5803083711118728  \\
            2.732  1.5324502560168283  \\
            1.221  1.5839378653292886  \\
            3.27  1.521310766080244  \\
            1.419  1.5699967056959094  \\
            0.628  1.69724566567861  \\
            1.066  1.6005731009642408  \\
            2.961  1.5264036206319078  \\
            3.176  1.521712275539421  \\
            1.618  1.5607507202168158  \\
            3.227  1.5214347595537314  \\
            1.63  1.5601584738007532  \\
            1.847  1.5529993931330486  \\
            1.541  1.5639333860390485  \\
            3.057  1.523502126339382  \\
            2.377  1.5402180911172336  \\
            2.755  1.5318684305137442  \\
            1.159  1.590053717815565  \\
            0.896  1.635273646839519  \\
            2.41  1.5394911210745377  \\
            0.76  1.6856255343549622  \\
            3.32  1.5212449657520848  \\
            2.346  1.540822006221142  \\
            3.447  1.5210835675651562  \\
            3.494  1.5210502806926058  \\
            1.642  1.5597404616114596  \\
            0.63  1.697051657285522  \\
            1.756  1.5558156827662737  \\
            1.035  1.6061858308689931  \\
            2.165  1.5447974862188367  \\
            2.227  1.5433994415529886  \\
            0.066  1.6979104893247752  \\
            3.431  1.5211063943522847  \\
            2.704  1.5330806146228424  \\
            2.389  1.5398121746171511  \\
            3.301  1.5212687767659914  \\
            0.112  1.6978817871426184  \\
            2.762  1.5316801889872895  \\
            1.988  1.5490559400154624  \\
            2.881  1.5287398611458105  \\
            0.521  1.697425809074808  \\
            3.296  1.5212708359726526  \\
            3.23  1.5214287297234887  \\
            2.156  1.5450452768131815  \\
            0.764  1.687101068489801  \\
            2.493  1.537657196801368  \\
            0.04  1.6979082402678813  \\
            2.629  1.534740323084156  \\
            2.882  1.5286876736976653  \\
            1.646  1.5596919205579367  \\
            0.191  1.6978598525255815  \\
            0.363  1.6977871294795492  \\
            2.055  1.5473734220541144  \\
            2.007  1.5485174042400174  \\
            0.487  1.6976203502589187  \\
            1.589  1.5619241124589363  \\
            2.609  1.5352292306135014  \\
            1.56  1.5631363386711865  \\
            0.264  1.6978654498614534  \\
            3.115  1.5222008547779242  \\
            0.983  1.6152397017305762  \\
            2.492  1.5377031337199167  \\
            1.829  1.5534752130723062  \\
            2.189  1.544175557530864  \\
            0.095  1.6978979630565727  \\
            1.553  1.5633736925729829  \\
            0.198  1.697892215274497  \\
            2.36  1.5404202308458623  \\
            1.695  1.5577748547203134  \\
            1.24  1.5820653223812402  \\
            3.085  1.522802116547321  \\
            3.5  1.5210464775688437  \\
            0.297  1.6978263368217013  \\
            0.282  1.6977424620025223  \\
            0.289  1.6977964550584914  \\
            2.96  1.5264361124290318  \\
            0.784  1.67709259517469  \\
            2.893  1.5285187110969947  \\
            0.295  1.6977884004581518  \\
            1.875  1.5521298424769756  \\
            2.724  1.5326384134067452  \\
            1.195  1.5864670794682703  \\
            1.551  1.563393236355276  \\
            3.122  1.522185381210071  \\
            0.446  1.6976909929567134  \\
            2.237  1.5431370408898795  \\
            2.418  1.5392689169227394  \\
            1.846  1.5528714889214308  \\
            3.409  1.5211196523102792  \\
            0.937  1.6231928215930784  \\
            1.304  1.57740675995255  \\
            1.302  1.5775907355238639  \\
            3.161  1.5217985919723578  \\
            2.079  1.546824432764586  \\
            1.019  1.6080581834286247  \\
            1.545  1.5639093877386212  \\
            2.217  1.5436032442100487  \\
            1.15  1.5906542976527769  \\
            0.815  1.6614594451321312  \\
            2.681  1.5335967644853423  \\
            0.891  1.6363262178144755  \\
            1.09  1.5981948535623163  \\
            1.073  1.6000762280503278  \\
            0.632  1.6967635518779496  \\
            3.303  1.5212599988332451  \\
            3.168  1.5217313685869496  \\
            1.109  1.5949880065651294  \\
            2.495  1.537623069689047  \\
            3.212  1.521494433699487  \\
            3.246  1.5213959879631591  \\
            2.545  1.536619003613027  \\
            2.723  1.532647074327653  \\
            2.05  1.5475244885900536  \\
            3.266  1.521331981013179  \\
            2.078  1.5468118131687545  \\
            3.133  1.5220620286819113  \\
            3.142  1.5220019211843598  \\
            0.41  1.6976003891694245  \\
            0.558  1.6973546706718605  \\
            2.75  1.5320272311066128  \\
            1.421  1.56968146835493  \\
            3.242  1.521402812801543  \\
            0.719  1.693083619452179  \\
            3.211  1.5215028367741643  \\
            1.328  1.5756204006211516  \\
            2.403  1.5395797123260666  \\
            0.24  1.6978562257348995  \\
            1.087  1.5984283908336376  \\
            1.599  1.5614325042342156  \\
            2.611  1.535171441329123  \\
            2.551  1.5364463485382716  \\
            1.949  1.550059844544465  \\
            3.405  1.5211344942124954  \\
            1.156  1.5902605236489213  \\
            1.061  1.6017048996811094  \\
            2.19  1.54418135441632  \\
            0.039  1.6978477725200483  \\
            0.199  1.6978729339067018  \\
            3.415  1.5211064342491019  \\
            2.22  1.5435581640932463  \\
            0.611  1.6968456586519993  \\
            2.502  1.5374943348822063  \\
            1.311  1.5767300482606437  \\
            1.764  1.55542787458787  \\
            3.317  1.521224231753924  \\
            2.09  1.546482827859478  \\
            0.597  1.697093050748906  \\
            0.478  1.6976517053228624  \\
            1.114  1.5946932868808903  \\
            1.072  1.6002608839167605  \\
            0.985  1.6142746401102412  \\
            1.331  1.5754287633373045  \\
            3.421  1.521118097794386  \\
            1.327  1.575717749034109  \\
            2.605  1.5352380920536242  \\
            3.027  1.5242527324372477  \\
            3.449  1.5210814608065  \\
            0.709  1.6950568136879487  \\
            1.315  1.5764493206748575  \\
            2.833  1.5300056754450913  \\
            2.507  1.5374227661371245  \\
            2.464  1.5383140564921125  \\
            0.752  1.6899218885412852  \\
            1.102  1.5963770010699148  \\
            0.459  1.697623929518313  \\
            1.108  1.5955422592905488  \\
            1.734  1.5564081185519099  \\
            2.765  1.5316517676618888  \\
            1.393  1.5713183790757002  \\
            1.635  1.5600289551105957  \\
            0.703  1.6954710884833626  \\
            1.458  1.5678137361992204  \\
            3.28  1.521301917945163  \\
            1.106  1.5959802086906325  \\
            1.41  1.5702408638645209  \\
            2.394  1.5397756097614095  \\
            2.555  1.536355130633593  \\
            2.473  1.5380973293266726  \\
            0.371  1.697785478346114  \\
            3.173  1.5216967519715743  \\
            2.424  1.5391157656341208  \\
            3.338  1.521192327321352  \\
            0.283  1.6977916523461543  \\
            1.015  1.6092718327365934  \\
            2.225  1.5434146468514651  \\
            0.141  1.6978181821323322  \\
            0.458  1.6976330790456948  \\
            1.945  1.550183251312689  \\
            1.51  1.5652147962267025  \\
            0.031  1.6978778616608592  \\
            1.877  1.5521037035135914  \\
            0.344  1.697884517569016  \\
            1.136  1.5923164902849545  \\
            3.175  1.5217019327201782  \\
            2.638  1.5345960837203392  \\
            3.461  1.5210821701555493  \\
            0.906  1.6319295963860418  \\
            0.153  1.6979133125898027  \\
            3.157  1.5217846335075587  \\
            1.231  1.5828132228544107  \\
            1.586  1.5620178818149648  \\
            2.032  1.547923020142095  \\
            0.884  1.6372435907317688  \\
            2.987  1.525367735702009  \\
            1.782  1.554848527336825  \\
            0.754  1.6897902127900888  \\
            3.455  1.5210787576940934  \\
            1.452  1.5684026397551065  \\
            2.557  1.536311637306789  \\
            0.59  1.6972475314506212  \\
            1.785  1.5548406527346734  \\
            1.893  1.551455408328703  \\
            3.103  1.5224303608535046  \\
            3.281  1.5212986052591972  \\
            2.584  1.5357406197330667  \\
            3.391  1.5211383051069032  \\
            3.348  1.5211859409379278  \\
            2.157  1.5449862940700456  \\
            1.862  1.5525105232694765  \\
            0.6  1.696930976630043  \\
            0.822  1.6589703668460427  \\
            0.654  1.6964733915471781  \\
            1.532  1.564261477635435  \\
            2.204  1.5439151952324817  \\
            1.42  1.569970150883445  \\
            2.295  1.54187027143998  \\
            2.528  1.5369968209780378  \\
            1.947  1.5501509216789338  \\
            1.801  1.554261262231112  \\
            1.245  1.5818124641203717  \\
            1.345  1.5745184223219788  \\
            0.722  1.6943215538837775  \\
            2.776  1.5313560981256338  \\
            2.669  1.5338577538442142  \\
            3.407  1.5211131030151364  \\
            0.607  1.6968581004605312  \\
            0.238  1.6978602755367402  \\
            1.39  1.5716238464700896  \\
            1.526  1.5644562709084582  \\
            0.279  1.6977877070049556  \\
            1.69  1.5580115407561659  \\
            0.008  1.697936281167871  \\
            2.573  1.535925020112858  \\
            2.03  1.5479936365144562  \\
            0.944  1.622164303545178  \\
            0.318  1.6978496755825292  \\
            0.736  1.6935780621414116  \\
            3.223  1.5214884602827454  \\
            0.374  1.697797765415448  \\
            1.1  1.5964746655910844  \\
            2.912  1.5279650877980466  \\
            0.524  1.6975129973806342  \\
            1.219  1.5840696292840595  \\
            2.697  1.5331880747225013  \\
            0.0  1.6978789832609218  \\
            3.178  1.5216503652236075  \\
            0.479  1.6975260147042222  \\
            0.167  1.6978660391190403  \\
            2.841  1.5297798179198048  \\
            1.352  1.5741124583624797  \\
            0.476  1.6976009343521388  \\
            0.69  1.6952542262937886  \\
            1.267  1.579947130720147  \\
            1.356  1.5735359730777823  \\
            1.198  1.585657153342031  \\
            1.205  1.5851600229677119  \\
            0.096  1.6978825676945317  \\
            0.226  1.6978072159276412  \\
            2.716  1.532806380321249  \\
            2.832  1.5300580083292088  \\
            1.958  1.5498192393380459  \\
            1.403  1.5709071033346944  \\
            2.879  1.5287529531153328  \\
            1.391  1.5715320427627486  \\
            2.674  1.5337113373379476  \\
            1.196  1.5862109037849503  \\
            0.65  1.6965614235280744  \\
            2.773  1.5314817904026068  \\
            2.305  1.5416634163595286  \\
            1.525  1.5644887697749792  \\
            0.975  1.616551606317045  \\
            0.686  1.6964500615802824  \\
            0.133  1.697927813945437  \\
            2.813  1.5304571972874705  \\
            0.54  1.697377962700065  \\
            0.563  1.6973617609730827  \\
            0.045  1.6978970049970634  \\
            1.82  1.5537390051764173  \\
            2.072  1.5469836029455155  \\
            3.007  1.524879611631271  \\
            1.203  1.5853238890793744  \\
            1.475  1.567067340281924  \\
            1.422  1.5699403951059254  \\
            1.14  1.5917473976197283  \\
            1.158  1.589639784226421  \\
            0.912  1.629859172079278  \\
            1.207  1.5849959903616802  \\
            1.371  1.5728447691734748  \\
            0.963  1.6175297206202925  \\
            0.987  1.613558201209168  \\
            1.572  1.5625504132263073  \\
            3.099  1.52248899321909  \\
            0.511  1.6974417367603405  \\
            0.541  1.697331274219359  \\
            1.739  1.5562774364346155  \\
            3.129  1.522051062856777  \\
            0.623  1.6968823037570018  \\
            0.018  1.6978627055446867  \\
            0.627  1.6969395749533867  \\
            3.004  1.5249824764922204  \\
            2.73  1.5324680222990439  \\
            2.902  1.528197815270825  \\
            1.851  1.5527914002633991  \\
            3.29  1.521287452738366  \\
            1.718  1.5569985687935104  \\
            1.578  1.562378854499872  \\
            2.653  1.534251691916977  \\
            0.647  1.6964037061562653  \\
            1.046  1.603597455854875  \\
            0.835  1.6547614456132718  \\
            0.234  1.697844021657933  \\
            1.652  1.559313479517644  \\
            0.578  1.6972767415314225  \\
            1.692  1.5579249212721258  \\
            2.505  1.5374243043868592  \\
            2.83  1.5300106339768114  \\
            2.901  1.528114016518059  \\
            0.789  1.6740018578482705  \\
            2.077  1.5468132103056247  \\
            2.136  1.5454981253908668  \\
            1.178  1.587838571116594  \\
            0.455  1.697661761581423  \\
            1.908  1.5510784915692395  \\
            1.637  1.5600160021287377  \\
            3.358  1.52118026061506  \\
            2.898  1.5282730679530538  \\
            1.933  1.55048094933619  \\
            3.016  1.5246350872285832  \\
            3.416  1.5211149459378168  \\
            2.065  1.5471477505496467  \\
            0.962  1.6186313922457614  \\
            3.019  1.5245323934115098  \\
            0.192  1.69786628711697  \\
            0.951  1.6210476709370483  \\
            2.624  1.5348844798862826  \\
            1.743  1.55607036833555  \\
            0.083  1.6978736531557153  \\
            2.232  1.5432409999496084  \\
            0.323  1.6978392671824796  \\
            0.093  1.6978828711468161  \\
            1.281  1.5787396338721487  \\
            1.852  1.5527488734768706  \\
            3.342  1.521190173213279  \\
            1.577  1.5623713328459348  \\
            2.874  1.528964909256404  \\
            1.226  1.583349733385273  \\
            3.373  1.5211537140850098  \\
            2.4  1.539672187902601  \\
            1.293  1.5780043243558113  \\
            1.712  1.5573287731889678  \\
            3.377  1.5211521967789559  \\
            2.202  1.543881368402351  \\
            2.39  1.5398935080014022  \\
            1.951  1.549992493345083  \\
            2.239  1.5430945278111197  \\
            1.845  1.552985851522121  \\
            2.178  1.544539095617074  \\
            0.072  1.6979290224155879  \\
            2.012  1.5484589161699682  \\
            1.709  1.5572837969173356  \\
            0.027  1.6979194109294424  \\
            1.437  1.5688444411170266  \\
            2.758  1.5317837617033527  \\
            1.495  1.565856439655985  \\
            0.51  1.6975891437749817  \\
            0.253  1.697879164498016  \\
            2.481  1.5378669601401194  \\
            1.357  1.5737057638350869  \\
            2.855  1.5293974713023828  \\
            3.473  1.5210640292587143  \\
            2.676  1.533699150249686  \\
            1.038  1.6050453576070627  \\
            1.704  1.557468689976773  \\
            0.086  1.6978860679483616  \\
            1.59  1.5618376584006533  \\
            2.273  1.5423761937734533  \\
            2.805  1.5306760235725319  \\
            0.026  1.6978918931271945  \\
            2.058  1.547256621672187  \\
            0.245  1.6978313018169038  \\
            1.351  1.5739377430578623  \\
            0.797  1.6731235057693803  \\
            1.265  1.5803644160810946  \\
            0.333  1.6977899217985235  \\
            3.125  1.5221486197418665  \\
            2.52  1.5371138428001152  \\
            0.879  1.6393156967721647  \\
            0.132  1.697900197355562  \\
            1.517  1.5649501242113115  \\
            2.793  1.5309149742031019  \\
            2.637  1.5345603796649427  \\
            3.124  1.5220569290791008  \\
            2.092  1.5465446315178721  \\
            2.824  1.5301666042265676  \\
            2.526  1.5370145105400144  \\
            2.673  1.5337569535383884  \\
            1.063  1.6015275974224723  \\
            1.31  1.5768300150370085  \\
            2.973  1.526039768250738  \\
            3.117  1.52213225713206  \\
            0.491  1.6976016460861647  \\
            0.9  1.6336101034735457  \\
            1.034  1.6066138399698429  \\
            2.307  1.5416086375907956  \\
            0.913  1.6312346528827122  \\
            1.608  1.5610197188932067  \\
            1.606  1.5611061424505015  \\
            0.016  1.6978421604705463  \\
            1.23  1.582868460114648  \\
            0.527  1.6974584991970807  \\
            0.832  1.6554948858354959  \\
            1.835  1.553200508149677  \\
            0.609  1.6971231780060085  \\
            3.399  1.5211248586156336  \\
            1.163  1.5892868801648856  \\
            1.264  1.5803301345379661  \\
            1.92  1.5508363861332188  \\
            1.542  1.5638887239041666  \\
            1.84  1.553187198112529  \\
            2.648  1.5343401268598849  \\
            1.611  1.5609136953410299  \\
            0.557  1.697354795301739  \\
            0.587  1.6972479631679656  \\
            3.024  1.5244289873971548  \\
            2.391  1.5398635851415183  \\
            0.113  1.697850692427567  \\
            0.18  1.6979295552798643  \\
            2.073  1.546870641041576  \\
            1.914  1.5509742729500327  \\
            1.432  1.5691499042272317  \\
            3.352  1.5211839560743796  \\
            1.601  1.561295919774175  \\
            1.04  1.6049720034669441  \\
            1.068  1.6006872162842074  \\
            2.364  1.540421536062788  \\
            3.382  1.5211555591771653  \\
            1.431  1.5693404198799346  \\
            1.767  1.5553735093603822  \\
            2.271  1.5424191148317907  \\
            0.135  1.697892856088248  \\
            2.245  1.5429805477587608  \\
            0.442  1.697678946997701  \\
            2.76  1.5317657147912984  \\
            0.957  1.619478430775275  \\
            2.517  1.5371443626157548  \\
            2.66  1.5340483077039344  \\
            1.134  1.5924179499380011  \\
            2.617  1.5350145031942408  \\
            2.521  1.53713881088605  \\
            1.638  1.5598816754070006  \\
            2.049  1.547495945331567  \\
            3.424  1.521089540353245  \\
            1.808  1.5541667735217366  \\
            0.586  1.6970709719102188  \\
            0.629  1.6970330143243302  \\
            0.209  1.6978493367345038  \\
            2.113  1.5459934866514682  \\
            0.235  1.6978776656540011  \\
            1.57  1.5627128132633819  \\
            2.38  1.5401145025411598  \\
            3.262  1.521335980002039  \\
            2.781  1.531237703882935  \\
            1.924  1.5507843167516506  \\
            0.137  1.6978919267778498  \\
            1.055  1.6026877984148082  \\
            2.866  1.5291527059042802  \\
            2.028  1.547955594233606  \\
            2.091  1.546515871489164  \\
            2.844  1.5296781319878092  \\
            0.382  1.6977349280879526  \\
            0.544  1.6973163258144617  \\
            2.336  1.5410025943755508  \\
            0.961  1.6189301103519498  \\
            2.265  1.5425326279486318  \\
            3.467  1.5210731486529316  \\
            1.96  1.5497837306945488  \\
            1.636  1.5600290255332347  \\
            1.726  1.5567003493778113  \\
            3.468  1.5210690614984683  \\
            0.223  1.6977787252092071  \\
            1.788  1.554644651173939  \\
            2.311  1.5415186020984128  \\
            1.103  1.595848336683  \\
            1.407  1.5704952894015474  \\
            1.079  1.5994175748885822  \\
            0.68  1.6959808893829036  \\
            2.542  1.5366245452046499  \\
            0.471  1.6975515169511022  \\
            3.013  1.5245628423909978  \\
            0.115  1.6978850241886603  \\
            2.292  1.5419972995568547  \\
            1.745  1.556083477799065  \\
            2.598  1.5354173078929285  \\
            2.203  1.5439508894046494  \\
            1.896  1.551555166971646  \\
            0.996  1.6120171454209873  \\
            2.795  1.5309164540232276  \\
            0.356  1.6977306744518312  \\
            2.754  1.5319111262951033  \\
            3.222  1.5214614753963684  \\
            3.438  1.5210958666485657  \\
            1.307  1.5770976860060282  \\
            1.251  1.5811700050352764  \\
            3.077  1.5228174823363871  \\
            1.443  1.568756698885767  \\
            0.06  1.6978563625483274  \\
            2.92  1.527608559696886  \\
            0.528  1.6974913397133284  \\
            3.055  1.5234448833349523  \\
            0.336  1.6977863137066067  \\
            0.774  1.684486099210111  \\
            0.538  1.6974904856027453  \\
            3.437  1.5210956880895181  \\
            1.018  1.6086303531957757  \\
            2.254  1.5427207311774636  \\
            0.033  1.6978914773233824  \\
            1.522  1.5647760781396116  \\
            2.345  1.5408492076262694  \\
            2.301  1.5418213334745006  \\
            1.263  1.5801277474430813  \\
            3.219  1.5214993717513206  \\
            2.588  1.5356579874535567  \\
            0.766  1.685270121975039  \\
            0.343  1.6976998186576944  \\
            0.42  1.6977316779842184  \\
            1.301  1.5772524683849793  \\
            2.699  1.5331563155560382  \\
            2.726  1.5325699430852624  \\
            2.77  1.5315440073132067  \\
            0.32  1.697788009341635  \\
            1.644  1.5596394948254921  \\
            3.498  1.5210475155598973  \\
            2.442  1.5387235998989295  \\
            1.883  1.5518774164531997  \\
            1.056  1.6022208214183367  \\
            0.243  1.6978445386945429  \\
            3.156  1.521820096723904  \\
            2.488  1.5377936396949972  \\
            0.655  1.6966436279070796  \\
            3.154  1.5218379221371476  \\
            0.262  1.6977773137856573  \\
            1.16  1.5894744188636953  \\
            3.146  1.5218958904724689  \\
            1.021  1.6081081953065424  \\
            1.096  1.5971834624005836  \\
            1.38  1.5723012150062254  \\
            2.84  1.529749728222015  \\
            1.296  1.5779992961566172  \\
            0.303  1.697770503150785  \\
            2.32  1.541392170129121  \\
            0.795  1.6713859325218194  \\
            0.316  1.6977996103115727  \\
            1.287  1.5786653474256314  \\
            2.583  1.5357832374620808  \\
            1.752  1.5558069259139868  \\
            3.332  1.521208203303054  \\
            1.268  1.5800330191500527  \\
            1.059  1.6019404592610769  \\
            0.849  1.6484169664546071  \\
            1.19  1.586810975232027  \\
            0.875  1.6404876573193086  \\
            3.277  1.5213000517224131  \\
            3.093  1.5226085175852473  \\
            2.867  1.5290374534598985  \\
            1.35  1.5740955043031377  \\
            3.087  1.522733541265843  \\
            0.658  1.6965124651416736  \\
            0.449  1.69765723077344  \\
            3.209  1.5215478910805527  \\
            2.445  1.5385333248109876  \\
            2.448  1.5386377821913126  \\
            2.654  1.5342177573001174  \\
            3.253  1.5213907265395086  \\
            1.425  1.5694039642605844  \\
            1.581  1.5620086457574833  \\
            1.208  1.5850212464558993  \\
            3.408  1.5211172935461181  \\
            0.758  1.6898062658604565  \\
            2.923  1.5275429990108584  \\
            1.436  1.56905730492338  \\
            2.713  1.5328708064900272  \\
            0.545  1.6974862892554194  \\
            3.376  1.521147880712946  \\
            0.5  1.697529925771125  \\
            0.704  1.6948417694662823  \\
            0.741  1.6936624693070668  \\
            1.346  1.5743486158549462  \\
            1.621  1.5604994560444985  \\
            0.355  1.6977399965983622  \\
            3.033  1.5240985543582242  \\
            0.929  1.6258844494066558  \\
            1.863  1.552439271272626  \\
            1.66  1.559025756464378  \\
            3.24  1.5214237512752131  \\
            2.224  1.5434780308376896  \\
            2.038  1.5478170769029476  \\
            1.341  1.5745767742735999  \\
            0.494  1.6976301417554909  \\
            0.426  1.6977152903542148  \\
            1.964  1.5496690813347795  \\
            3.354  1.5211721581191613  \\
            1.681  1.5582585285128046  \\
            0.328  1.6977838537431593  \\
            0.707  1.6954460711363948  \\
            0.306  1.697766740734485  \\
            2.554  1.5363833322558693  \\
            1.305  1.5769624760189902  \\
            0.993  1.6127768207894893  \\
            2.285  1.5421466455174941  \\
            0.273  1.6977844893722795  \\
            1.885  1.5518834371064085  \\
            2.404  1.5395848711780524  \\
            0.49  1.6975905762575898  \\
            1.498  1.5658882394111235  \\
            2.951  1.5267968460438148  \\
            2.929  1.527405789379028  \\
            1.148  1.5907727174793576  \\
            0.302  1.697802400196907  \\
            1.098  1.596645697285541  \\
            0.999  1.6117782560460916  \\
            0.821  1.6585863634839744  \\
            1.439  1.5688513191570246  \\
            3.152  1.5218777700577912  \\
            2.2  1.5439936164516033  \\
            3.089  1.522578407891061  \\
            1.209  1.5848460510809628  \\
            0.573  1.6972329233702923  \\
            0.042  1.6978960016281304  \\
            0.294  1.6977857509561547  \\
            1.049  1.603608171737133  \\
            2.627  1.5347691906718879  \\
            2.211  1.5437249712923966  \\
            2.083  1.5466245078571843  \\
            0.895  1.6348657448706683  \\
            2.693  1.5333698414170953  \\
            1.575  1.5624505232452692  \\
            1.836  1.5532869901474913  \\
            3.224  1.5214750587934105  \\
            1.218  1.5842638415284986  \\
            1.048  1.6034064157081858  \\
            1.747  1.5561418579256086  \\
            0.928  1.625756098597262  \\
            0.021  1.6979034310152599  \\
            2.655  1.5342086320826223  \\
            2.71  1.5329617229917372  \\
            1.277  1.5793740589983059  \\
            0.665  1.6962229842922114  \\
            1.992  1.5488919740564877  \\
            1.259  1.58075820461502  \\
            2.812  1.5305153788373134  \\
            3.433  1.5210922389931338  \\
            3.145  1.5218874153769029  \\
            2.934  1.5271832966922374  \\
            3.202  1.5215292113997736  \\
            0.619  1.6972778244150175  \\
            1.118  1.5943761051256524  \\
            1.765  1.5555254059354249  \\
            0.633  1.6969641812012477  \\
            3.499  1.5210418318609933  \\
            3.111  1.5222287424226275  \\
            0.361  1.6976559966155693  \\
            0.663  1.6964493046481304  \\
            2.954  1.526659813012985  \\
            2.538  1.5367292983204828  \\
            0.38  1.6977067966437591  \\
            2.666  1.533997964007591  \\
            0.011  1.69787352512384  \\
            3.475  1.521060846191934  \\
            2.082  1.5466831929231657  \\
            1.528  1.5645127879225986  \\
            3.234  1.5214315325268428  \\
            2.82  1.5302297074074533  \\
            3.394  1.5211399850082985  \\
            1.564  1.5628719518395753  \\
            2.33  1.541119947685273  \\
            0.783  1.6757293271525027  \\
            1.493  1.5660018160042828  \\
            1.641  1.559813603197768  \\
            3.183  1.5216289122066615  \\
            0.338  1.6977836415384373  \\
            2.558  1.53627650132194  \\
            0.425  1.6975695317730286  \\
            1.496  1.5659578231071085  \\
            3.481  1.5210558786366837  \\
            0.325  1.6977653351860953  \\
            3.383  1.521159306581177  \\
            2.447  1.5386720130894296  \\
            0.474  1.6975735722062053  \\
            0.105  1.697895942877897  \\
            3.197  1.5215593144484947  \\
            1.451  1.5683352738552343  \\
            0.383  1.6977523038411044  \\
            1.735  1.5564296056887243  \\
            0.668  1.6959783551403131  \\
            1.372  1.5728292369926145  \\
            3.191  1.5215984100011173  \\
            0.475  1.6976090412374911  \\
            1.154  1.5904061835854153  \\
            2.338  1.5409513251952653  \\
            2.585  1.5357436959929738  \\
            1.784  1.5549066749049063  \\
            2.026  1.5480933955289877  \\
            1.76  1.5555386135613634  \\
            2.817  1.5303708792136674  \\
            1.781  1.554953464860642  \\
            2.995  1.5253869819973476  \\
            2.413  1.5394176855768575  \\
            2.003  1.5486618657312081  \\
            2.771  1.5314663033441145  \\
            1.892  1.5516002819140937  \\
            1.007  1.6099561085146055  \\
            1.453  1.5680774423169301  \\
            0.186  1.6978143179899206  \\
            0.866  1.6435394225718216  \\
            3.259  1.5213687028231577  \\
            1.79  1.5547282315755835  \\
            3.364  1.5211640621158584  \\
            0.525  1.6975144231456325  \\
            2.361  1.5405319568859952  \\
            1.913  1.5509932637561397  \\
            1.448  1.5684237504004386  \\
            2.182  1.5444723149415784  \\
            2.843  1.5296159495208348  \\
            2.01  1.548540397147247  \\
            0.631  1.6970034513582457  \\
            3.302  1.5212607265786642  \\
            0.915  1.6306270539696248  \\
            1.338  1.574910473283095  \\
            1.172  1.5883121147474675  \\
            0.13  1.6978723239321667  \\
            1.811  1.5539507475443715  \\
            2.29  1.5419786593833424  \\
            2.319  1.5413879025376935  \\
            2.63  1.534721313492538  \\
            2.019  1.5483147521507326  \\
            1.261  1.580395706456224  \\
            2.451  1.538577797024507  \\
            2.661  1.5340941708365372  \\
            0.456  1.6976413926554836  \\
            2.44  1.5388135852919638  \\
            2.288  1.5420325985940573  \\
            1.37  1.5728837880931155  \\
            2.23  1.5433473327449099  \\
            2.587  1.5356342485155903  \\
            1.182  1.5871088283259613  \\
            3.497  1.521042449285344  \\
            2.278  1.5422744546354719  \\
            3.322  1.5212194512811934  \\
            1.354  1.573866188688955  \\
            1.472  1.567049348802036  \\
            2.373  1.5402438674601713  \\
            3.485  1.5210560520391712  \\
            0.793  1.673390744941141  \\
            0.324  1.697767787159968  \\
            3.127  1.5220374334090165  \\
            2.679  1.5336881895480088  \\
            3.278  1.521322929642833  \\
            1.749  1.5559574760653676  \\
            2.706  1.5330877479453184  \\
            1.143  1.5914370124707666  \\
            2.406  1.5395384992842795  \\
            2.504  1.5374608713898006  \\
            1.225  1.5835708064667704  \\
            1.489  1.5663259295272438  \\
            0.989  1.6134208237685488  \\
            2.097  1.5463619336779246  \\
            2.219  1.5435590870293385  \\
            1.47  1.5671724599009462  \\
            1.9  1.551440271878563  \\
            1.516  1.5650509061339584  \\
            1.101  1.5965103697105358  \\
            2.539  1.5366620292193256  \\
            2.964  1.5262684691074035  \\
            2.636  1.5345840218248314  \\
            1.849  1.5528630639705083  \\
            1.691  1.5580375080174254  \\
            2.399  1.5396649921824455  \\
            2.452  1.5385519649694446  \\
            1.027  1.6061322890126022  \\
            0.447  1.697633558340877  \\
            3.483  1.521042881918627  \\
            0.841  1.6516103320361346  \\
            0.549  1.6970479051461775  \\
            2.949  1.5267380399421404  \\
            2.641  1.5345271119664972  \\
            0.463  1.6975871456869618  \\
            1.696  1.557697084099489  \\
            2.537  1.5367536960212063  \\
            2.185  1.544335762461624  \\
            0.87  1.642555794933655  \\
            2.746  1.5320966736043629  \\
            1.812  1.553981025277162  \\
            0.529  1.697462559277871  \\
            2.048  1.5475477350513105  \\
            2.61  1.5351730183759527  \\
            0.1  1.6978495500279458  \\
            1.942  1.5502585919179053  \\
            1.519  1.5647714814781817  \\
            3.194  1.5215726784193324  \\
            1.484  1.566600457508265  \\
            3.106  1.5223909610630242  \\
            1.355  1.5738062690657335  \\
            3.446  1.5210812908248292  \\
            3.346  1.521202173066327  \\
            0.022  1.6979182444240917  \\
            2.496  1.5375925380610214  \\
            3.264  1.5213327480144772  \\
            0.537  1.6975039665547642  \\
            2.347  1.540727924363488  \\
            1.539  1.5639912333363668  \\
            1.55  1.5634422452903503  \\
            1.816  1.5537979042749535  \\
            0.591  1.6972048383440896  \\
            2.547  1.5365755930550404  \\
            2.698  1.5332504609421787  \\
            3.379  1.5211576628372712  \\
            1.479  1.566820601223006  \\
            0.804  1.6687222302957845  \\
            0.37  1.6977947234291115  \\
            1.333  1.5750992263849304  \\
            1.215  1.5842085298689492  \\
            0.403  1.6976994308846023  \\
            2.614  1.5350822157072082  \\
            0.071  1.6979180890184467  \\
            2.401  1.5396323590299879  \\
            0.346  1.6977832547341598  \\
            0.378  1.697661196774582  \\
            2.352  1.5406726452347546  \\
            0.939  1.6239904801974585  \\
            0.251  1.6978146855318728  \\
            0.981  1.6149386890779411  \\
            3.021  1.524430764999341  \\
            0.109  1.6978805402966533  \\
            3.482  1.5210527545171  \\
            1.503  1.5657268991329487  \\
            0.213  1.6977857552982494  \\
            0.882  1.6383202779781438  \\
            1.533  1.564285672854251  \\
            1.353  1.5740355686742529  \\
            2.8  1.5308454611696052  \\
            2.141  1.545294224443301  \\
            3.217  1.5214933382737548  \\
            2.986  1.5255552832543184  \\
            0.281  1.6978372929082726  \\
            2.981  1.5257341522979655  \\
            2.877  1.5288603915361016  \\
            0.208  1.6978555321411994  \\
            1.819  1.5537231206392588  \\
            0.387  1.6977082707874172  \\
            0.218  1.69786177606034  \\
            1.569  1.5627528690033035  \\
            3.306  1.521271998084008  \\
            0.669  1.6966019142353819  \\
            1.45  1.5683304615182172  \\
            1.909  1.5510337169440618  \\
            2.013  1.5483914494717492  \\
            2.121  1.5458245535712964  \\
            3.329  1.5211997167870512  \\
            0.864  1.6440167117696534  \\
            0.531  1.6975706114783262  \\
            2.85  1.5295504637346071  \\
            0.551  1.6973527443001297  \\
            2.393  1.5398354700879722  \\
            0.317  1.6977322949181866  \\
            1.03  1.605772174211554  \\
            0.808  1.6647210534382373  \\
            0.984  1.6143668855739415  \\
            3.228  1.5214644794775434  \\
            1.337  1.5749581394386394  \\
            0.497  1.6975452272112166  \\
            2.341  1.5408792933937099  \\
            0.964  1.6181052540944767  \\
            1.742  1.5562517350089908  \\
            3.445  1.521087228544152  \\
            0.054  1.6978902572830432  \\
            3.355  1.5211745977909323  \\
            2.65  1.5342861981546094  \\
            1.125  1.5936226970526362  \\
            3.006  1.5250983602362453  \\
            0.051  1.6978177472428249  \\
            2.477  1.538043306857836  \\
            3.357  1.5211710978109965  \\
            2.144  1.5452770897589654  \\
            0.909  1.6313673574603866  \\
            2.107  1.5460986666833978  \\
            1.615  1.5608536759097302  \\
            2.895  1.528329011682251  \\
            2.665  1.5339120271055937  \\
            0.187  1.6978696983826598  \\
            0.375  1.6976897357869671  \\
            3.051  1.5235169203852845  \\
            0.339  1.697786785166124  \\
            3.037  1.524027904850169  \\
            2.169  1.5447075394350738  \\
            2.652  1.5342663269128738  \\
            1.313  1.576595702291328  \\
            1.374  1.5725564294361662  \\
            3.324  1.5212313659322516  \\
            0.601  1.697228320522093  \\
            0.29  1.697805385253557  \\
            2.17  1.5446635475640726  \\
            0.122  1.6979118759336869  \\
            0.919  1.6274840983796681  \\
            1.32  1.5759722699902603  \\
            0.322  1.6977437825387096  \\
            0.724  1.6927971272317048  \\
            1.408  1.570594351883597  \\
            1.508  1.5653621696919666  \\
            0.966  1.6179071054151184  \\
            2.988  1.5256728593180224  \\
            2.453  1.53851042927639  \\
            3.028  1.5242873986801624  \\
            1.731  1.5566769361028292  \\
            1.779  1.555016576361713  \\
            2.647  1.5343349021100314  \\
            1.963  1.5496268897631964  \\
            2.223  1.543583394355464  \\
            2.822  1.5302799723614087  \\
            2.971  1.5260693812921164  \\
            2.106  1.5461370700940202  \\
            0.842  1.6512485478051602  \\
            0.34  1.6977954985804007  \\
            3.072  1.5230303969428234  \\
            2.181  1.5444118601975294  \\
            1.796  1.5544103971446277  \\
            2.656  1.5341433939702278  \\
            0.366  1.6977488010578656  \\
            1.721  1.5569008958272157  \\
            1.3  1.5773617763335979  \\
            2.534  1.5368633844957174  \\
            0.428  1.6977784121897739  \\
            0.711  1.6951431023370238  \\
            1.624  1.5603653083600564  \\
            3.071  1.5229710776482093  \\
            2.888  1.5286164813550267  \\
            0.429  1.6977907464646793  \\
            1.878  1.5520521627399493  \\
            3.457  1.5210714987558605  \\
            0.812  1.6624284135940388  \\
            0.138  1.6978074831051768  \\
            1.112  1.5948293037884813  \\
            0.423  1.6977126971023686  \\
            2.768  1.531566082305687  \\
            1.627  1.560374788295739  \\
            2.597  1.5354745427604113  \\
            1.024  1.6075514065793415  \\
            1.086  1.598665621329614  \\
            0.28  1.6978826732941743  \\
            0.992  1.612821749406453  \\
            2.314  1.5414965994431422  \\
            1.499  1.565853042744105  \\
            0.829  1.6549827355706337  \\
            2.596  1.5354893374124763  \\
            1.08  1.5995342398979826  \\
            3.268  1.5213449873429041  \\
            0.843  1.647886769569797  \\
            0.771  1.6833243561510063  \\
            0.922  1.62794668207702  \\
            3.086  1.5226864540438327  \\
            0.35  1.697775340803512  \\
            2.465  1.5382417301527287  \\
            2.096  1.5464146178284326  \\
            2.519  1.537198630400186  \\
            0.827  1.6568683595906213  \\
            1.729  1.5565947277592618  \\
            2.577  1.5358871399779719  \\
            0.292  1.6978232378929161  \\
            1.153  1.5904064990263664  \\
            3.404  1.521129416302084  \\
            3.187  1.5215762536208688  \\
            0.509  1.6975004003090115  \\
            0.723  1.6945049051323433  \\
            2.536  1.5367437224497307  \\
            0.023  1.697910506287102  \\
            0.735  1.6935315658068526  \\
            1.961  1.5497346549686029  \\
            1.675  1.5585931626698182  \\
            2.501  1.5375385831090533  \\
            1.693  1.557860530426081  \\
            2.177  1.544487324226997  \\
            3.172  1.5217050332448656  \\
            0.439  1.6977548872019839  \\
            1.405  1.5705828992988342  \\
            0.643  1.6966187934923496  \\
            1.659  1.5592174086309851  \\
            1.962  1.5497335738778601  \\
            2.14  1.5454394797994895  \\
            0.624  1.6969886505125413  \\
            3.385  1.5211391891436978  \\
            3.001  1.5251178591069934  \\
            2.686  1.533509265631204  \\
            0.498  1.6975500895849696  \\
            1.912  1.5511507559587172  \\
            0.77  1.6846229026633688  \\
            1.93  1.5506039029166792  \\
            0.569  1.697332495722341  \\
            1.309  1.5766071329774045  \\
            0.284  1.6978202625353904  \\
            1.169  1.5890178179636076  \\
            0.874  1.6414825840888267  \\
            2.927  1.5275106291949228  \\
            1.347  1.5742956382988074  \\
            1.854  1.552674082306386  \\
            3.493  1.5210502613226222  \\
            0.585  1.6972906324060877  \\
            3.108  1.522310836467922  \\
            2.264  1.5426412738516198  \\
            3.1  1.5224114734742802  \\
            1.763  1.555520675768397  \\
            2.046  1.5475849088518108  \\
            0.819  1.659468389756212  \\
            2.975  1.526009320540069  \\
            2.827  1.5301373390164899  \\
            1.902  1.5513366545005656  \\
            0.484  1.697573961652655  \\
            0.977  1.6167144757616418  \\
            2.321  1.5413798394401415  \\
            1.768  1.55526355288844  \\
            0.853  1.6492077655493127  \\
            0.483  1.6975542794656462  \\
            2.044  1.547553772858525  \\
            3.435  1.521089259609616  \\
            1.084  1.5987336247431991  \\
            0.969  1.6173483107223516  \\
            2.95  1.5267022420030327  \\
            0.275  1.6978252444267457  \\
            1.473  1.5672091762217422  \\
            3.272  1.521334562172102  \\
            0.788  1.6787725648803604  \\
            2.535  1.5367942230528084  \\
            2.687  1.5334780239282315  \\
            0.009  1.6978942990873007  \\
            0.677  1.6964112313723796  \\
            1.121  1.5939177645587634  \\
            2.299  1.541833354164141  \\
            2.962  1.5264648873648432  \\
            0.358  1.6977596362010945  \\
            0.839  1.6529567191169843  \\
            2.431  1.5389822312473136  \\
            2.969  1.5262176659229032  \\
            1.596  1.5614163961650844  \\
            2.945  1.5268972159415595  \\
            2.179  1.5444904041283787  \\
            1.411  1.570535990969999  \\
            2.02  1.548255499968489  \\
            3.073  1.5230637884984708  \\
            2.163  1.5448299436716957  \\
            3.135  1.5219724679993323  \\
            1.117  1.5944724194948305  \\
            3.378  1.521144286353083  \\
            2.162  1.5448572619549907  \\
            0.166  1.6978528459680446  \\
            3.422  1.5211124593709155  \\
            1.535  1.5642733657429124  \\
            0.261  1.6978087288574846  \\
            3.113  1.5222833421031419  \\
            0.953  1.6205554961478423  \\
            0.329  1.6977700859735045  \\
            3.184  1.521633510588401  \\
            3.448  1.521084098137599  \\
            2.823  1.5302761999062553  \\
            2.284  1.5421610730238764  \\
            3.361  1.5211745929115588  \\
            3.374  1.5211549676637717  \\
            2.848  1.5295944934594161  \\
            0.923  1.6280869025807692  \\
            2.586  1.5357358465072606  \\
            0.102  1.6979111053182394  \\
            0.334  1.6977341262298107  \\
            0.301  1.6978363390025206  \\
            0.337  1.6977516133837047  \\
            0.894  1.6354547781108315  \\
            3.375  1.5211580621517324  \\
            0.468  1.6976943261882704  \\
            3.043  1.523796351845156  \\
            1.167  1.5887652908975558  \\
            2.599  1.535431593566581  \\
            0.236  1.6978531891055102  \\
            2.607  1.5352074383687773  \\
            3.049  1.5236061888593282  \\
            0.579  1.6971422629599306  \\
            0.131  1.6978834795170745  \\
            3.204  1.521562849299331  \\
            2.814  1.5304746869855785  \\
            0.645  1.6970257268986508  \\
            0.188  1.6978891983623585  \\
            2.876  1.5288438366955737  \\
            2.845  1.5296274410562234  \\
            0.502  1.6974578464617271  \\
            0.397  1.6976939652381189  \\
            1.483  1.5666523135114008  \\
            1.336  1.5750176554354085  \\
            3.177  1.5216707273377705  \\
            2.272  1.5424093303732613  \\
            3.153  1.521826936664633  \\
            1.673  1.5587640230268691  \\
            0.119  1.697899882517636  \\
            2.475  1.537992109585684  \\
            2.335  1.541016850190762  \\
            2.618  1.5350418425168468  \\
            0.184  1.6978001576128108  \\
            3.271  1.5213064770929159  \\
            2.476  1.5380613098188225  \\
            2.047  1.5475848477139695  \\
            2.701  1.5332098317198635  \\
            0.408  1.6976927983897385  \\
            0.79  1.6766542313618837  \\
            3.2  1.5215423384411815  \\
            0.143  1.697892478224206  \\
            1.834  1.5533217933013344  \\
            1.833  1.5533220122061435  \\
            0.593  1.6973390733523037  \\
            0.01  1.697856709372078  \\
            2.98  1.5257937049846413  \\
            0.535  1.6974427135466597  \\
            3.257  1.5213542042305965  \\
            1.197  1.5861774866613905  \\
            3.464  1.5210755129654066  \\
            2.052  1.5474726180008402  \\
            1.626  1.5602620038169253  \\
            0.398  1.697608301704476  \\
            2.59  1.5355937469333174  \\
            0.908  1.6314463406912396  \\
            0.351  1.6977722849134698  \\
            3.46  1.521076444215459  \\
            0.972  1.6162752679871355  \\
            0.237  1.6978887851509026  \\
            1.891  1.5516660535181344  \\
            3.384  1.5211445704591646  \\
            0.452  1.69767856817184  \\
            2.685  1.5334944648265618  \\
            2.549  1.5365065048550286  \\
            1.684  1.5581252402653736  \\
            2.69  1.533379267804831  \\
            3.147  1.521873049294843  \\
            2.897  1.5283090918714695  \\
            2.425  1.539154141782081  \\
            1.49  1.566233604228744  \\
            1.563  1.5629799328494858  \\
            2.645  1.5344156174178065  \\
            0.547  1.6974084434936185  \\
            0.372  1.6977378988097684  \\
            0.725  1.6935918052521315  \\
            0.25  1.6978569959088279  \\
            1.814  1.5538886357100106  \\
            2.443  1.538806197378806  \\
            1.619  1.5605179496387132  \\
            1.632  1.56018690414597  \\
            3.148  1.5218876313683287  \\
            1.485  1.5665975408684516  \\
            1.717  1.557000675869102  \\
            0.389  1.6977717876361336  \\
            0.956  1.619469000947373  \\
            0.373  1.697785540490725  \\
            0.02  1.6979183891099943  \\
            1.697  1.5578547403105916  \\
            0.73  1.6936227615603667  \\
            0.017  1.6978725464373665  \\
            1.546  1.5637453660701264  \\
            1.057  1.60156881180402  \\
            1.643  1.5597202401908667  \\
            2.562  1.5361936653781854  \\
            0.002  1.697893215077544  \\
            0.15  1.6979094714005778  \\
            0.615  1.6970319844096335  \\
            1.312  1.5766863843234862  \\
            1.141  1.5916952627070653  \\
            2.634  1.5346847316693688  \\
            1.828  1.553416708843167  \\
            0.014  1.6979326717537988  \\
            2.635  1.53466613664179  \\
            3.309  1.5212446492738996  \\
            0.025  1.6978974068936659  \\
            0.162  1.6978637695910128  \\
            3.36  1.5211729645528749  \\
            3.412  1.5211132656937436  \\
            1.991  1.549008401317309  \\
            0.534  1.6973267703566277  \\
            0.708  1.6955831782349766  \\
            3.047  1.5236730345782057  \\
            2.735  1.5324050539631113  \\
            2.375  1.5402818837144117  \\
            1.939  1.5504017980063927  \\
            1.507  1.5654210748461685  \\
            0.312  1.6978197057719753  \\
            2.034  1.5479301290542555  \\
            3.269  1.521326634580007  \\
            1.573  1.5625347279792379  \\
            0.925  1.6272933962888643  \\
            1.275  1.5794428902639706  \\
            2.606  1.535232534770973  \\
            2.649  1.5343120903410175  \\
            0.144  1.6978405222249737  \\
            2.349  1.5408131623456187  \\
            2.428  1.538997998112501  \\
            2.9  1.5282457642677927  \\
            3.025  1.5244249580114777  \\
            0.828  1.6543265863221124  \\
            0.032  1.6978838782718715  \\
            1.748  1.5559636188693684  \\
            1.266  1.5800911873742678  \\
            2.616  1.5350155548570095  \\
            0.995  1.6126380756304568  \\
            3.4  1.5211260310380468  \\
            0.39  1.6976848016493729  \\
            3.397  1.521138580383532  \\
            1.921  1.5507953286807898  \\
            1.092  1.5972695454736054  \\
            1.682  1.5581880488368618  \\
            2.212  1.5436794827995515  \\
            2.829  1.5300645309283347  \\
            1.344  1.5744025401143424  \\
            3.292  1.5212812964010727  \\
            0.694  1.6957830624188532  \\
            1.864  1.552491176943181  \\
            0.414  1.6976627557406236  \\
            2.808  1.5306362012588228  \\
            2.103  1.5462060919165266  \\
            1.597  1.561679087866205  \\
            0.015  1.697899870154417  \\
            1.139  1.5918323322041499  \\
            2.745  1.5321424296049657  \\
            2.062  1.5472229071042065  \\
            0.049  1.6978761108157052  \\
            0.954  1.6207852167514842  \\
            1.286  1.5783346799284725  \\
            3.243  1.5214210896455218  \\
            2.037  1.5478461400605572  \\
            3.096  1.5224225373129332  \\
            0.464  1.6975998801215324  \\
            0.976  1.6156420169732688  \\
            2.783  1.5312746414931406  \\
            0.614  1.6971993415330557  \\
            0.807  1.6671886104003653  \\
            0.701  1.6954266348339007  \\
            0.773  1.6827515731041247  \\
            2.696  1.533304102908444  \\
            3.158  1.5218083477445736  \\
            2.309  1.5415705843036105  \\
            2.769  1.5316160214099968  \\
            2.257  1.5427679657006934  \\
            2.524  1.5370536453536536  \\
            1.247  1.5817258879019758  \\
            0.189  1.6979311539353856  \\
            3.185  1.5216443027301327  \\
            0.036  1.697893089309006  \\
            3.359  1.5211707802802694  \\
            0.667  1.6960999324400043  \\
            0.077  1.697887259970264  \\
            1.888  1.5517709307126133  \\
            3.249  1.5213443884809574  \\
            2.512  1.5373414417082725  \\
            1.017  1.6085719766675113  \\
            2.791  1.5310427797445452  \\
            0.256  1.697882489346195  \\
            1.6  1.561274160614247  \\
            3.181  1.521625755946464  \\
            2.579  1.5358626546590801  \\
            0.461  1.6976322672710769  \\
            1.538  1.5641529319141148  \\
            3.283  1.52131992514631  \\
            2.564  1.5361985488877163  \\
            0.986  1.6134995809287278  \\
            2.326  1.5412459513014742  \\
            1.25  1.5815104215354872  \\
            3.474  1.5210647119727039  \\
            0.8  1.6706811100930983  \\
            3.241  1.5213841266043655  \\
            0.121  1.6978813449677157  \\
            0.441  1.697670070876691  \\
            1.841  1.553146953399894  \\
            1.11  1.5956037062618402  \\
            1.797  1.554518769406884  \\
            0.753  1.6869539352093819  \\
            2.031  1.547949540046932  \\
            1.395  1.5713791767296117  \\
            2.407  1.5394789955914325  \\
            2.372  1.5403208696819208  \\
            1.133  1.592939774086857  \\
            2.023  1.5482033401929298  \\
            0.672  1.696211837063222  \\
            3.203  1.52154277956309  \\
            2.499  1.5375306910903046  \\
            2.379  1.540078288115783  \\
            2.438  1.5388000667317268  \\
            3.056  1.5234596185514386  \\
            2.053  1.5474794921540864  \\
            2.857  1.5294373453041041  \\
            0.833  1.6528184489457287  \\
            0.86  1.6444070693125588  \\
            2.621  1.5349263369902697  \\
            0.705  1.695556425123695  \\
            2.595  1.5355041734735377  \\
            2.342  1.5409164787947642  \\
            2.039  1.5477720232609973  \\
            1.107  1.5955861207582152  \\
            0.649  1.6969570737305202  \\
            2.741  1.532203776701438  \\
            0.699  1.6957300080417264  \\
            2.333  1.5411484939883162  \\
            0.782  1.6746065743466882  \\
            1.83  1.553451136912131  \\
            3.237  1.5214106601231534  \\
            0.142  1.69786481581876  \\
            0.727  1.6945425976438369  \\
            2.801  1.5307786353501027  \\
            2.275  1.5423547497769243  \\
            1.969  1.5495846909614077  \\
            2.131  1.5455594221282576  \\
            1.28  1.5790295986863043  \\
            0.814  1.6647539478796052  \\
            0.556  1.6973897839342411  \\
            1.282  1.5787370235806415  \\
            0.845  1.6497559568882896  \\
            0.868  1.6433283528202782  \\
            2.868  1.5290313719995763  \\
            2.998  1.5251342670268466  \\
            1.592  1.561835435714543  \\
            0.085  1.6978924077743547  \\
            1.622  1.5607003171899143  \\
            2.022  1.5481865190832913  \\
            1.973  1.5493804008267757  \\
            2.905  1.528024668042821  \\
            1.699  1.5576156369540937  \\
            2.175  1.544495929207293  \\
            1.925  1.5507199168007932  \\
            0.012  1.6978878133652164  \\
            2.233  1.543296126109512  \\
            1.005  1.6107434036545023  \\
            1.685  1.558173553879521  \\
            0.728  1.6938194468736392  \\
            0.605  1.697155893048824  \\
            3.192  1.5215476684340512  \\
            0.196  1.6978685933723032  \\
            1.501  1.5657077076088082  \\
            3.282  1.521307645262806  \\
            1.233  1.582788615127619  \\
            1.016  1.6081131134612487  \\
            0.349  1.6977641005310253  \\
            2.263  1.5426391593914153  \\
            1.07  1.6005222914896151  \\
            0.765  1.6827603500561374  \\
            0.305  1.6978043253751376  \\
            1.48  1.5667550165897173  \\
            1.505  1.5654067499282227  \\
            1.186  1.587222094243998  \\
            0.111  1.6978760885110666  \\
            1.155  1.5901917054794295  \\
            3.045  1.5237340662915737  \\
            3.128  1.5220712779322674  \\
            3.336  1.5212168395935814  \\
            2.915  1.5277229424840915  \\
            0.013  1.6979066126936653  \\
            3.251  1.5213574979595894  \\
            2.992  1.5254341203393247  \\
            1.926  1.5507193758186864  \\
            3.054  1.5234834215796562  \\
            1.104  1.595866556066935  \\
            2.376  1.5401430658503674  \\
            1.983  1.549195882822349  \\
            2.158  1.54492280303264  \\
            2.467  1.5382283989008052  \\
            1.179  1.5873543834439727  \\
            1.701  1.5576981179747518  \\
            3.436  1.5210943155038694  \\
            0.794  1.6719338657002412  \\
            0.311  1.6978070134049665  \\
            0.98  1.615271553676925  \\
            1.382  1.5721289153654991  \\
            1.316  1.5761518111223116  \\
            2.753  1.5319529166263495  \\
            2.612  1.5351337216325542  \\
            1.003  1.6113632232791313  \\
            1.396  1.5715549401894506  \\
            0.635  1.6972799847839783  \\
            0.967  1.617775189293828  \\
            0.514  1.6975668899671637  \\
            0.846  1.6515976976340776  \\
            3.247  1.5213875769006884  \\
            2.151  1.545137270985294  \\
            1.822  1.5536348811240956  \\
            1.906  1.5512124191622052  \\
            3.14  1.5218918282965204  \\
            0.718  1.695147118199997  \\
            1.258  1.5807365328846763  \\
            1.243  1.581995728432155  \\
            2.93  1.5273199320995452  \\
            2.744  1.5321459371448563  \\
            1.81  1.554033383357984  \\
            0.239  1.6977835415160616  \\
            2.913  1.527848632714423  \\
            1.454  1.5680042047336553  \\
            2.873  1.5289036765294732  \\
            2.889  1.5285365673086417  \\
            0.675  1.696111570532077  \\
            2.015  1.5483281643940423  \\
            2.489  1.537767582481718  \\
            3.18  1.5216552125293523  \\
            0.169  1.697918058060443  \\
            3.215  1.5214941774782214  \\
            1.511  1.565351418229851  \\
            3.06  1.5233679802846647  \\
            2.807  1.5305393475763853  \\
            0.402  1.6976928017546558  \\
            2.684  1.5335221514630908  \\
            1.882  1.551868022452349  \\
            2.571  1.536001822044812  \\
            2.956  1.526552690194477  \\
            1.242  1.5819131256779264  \\
            0.838  1.6521420407798038  \\
            2.48  1.5379182661718842  \\
            0.046  1.697880577246145  \\
            0.564  1.6972221510125476  \\
            2.291  1.5419932108354268  \\
            3.403  1.5211246008154236  \\
            0.12  1.6978462876467304  \\
            0.959  1.6188699540277358  \\
            2.421  1.539203770768049  \\
            0.706  1.6946771426456217  \\
            0.159  1.6978903430036916  \\
            0.561  1.697339491674524  \\
            0.697  1.6960653649649349  \\
            0.037  1.6978920677770466  \\
            1.728  1.5566623538539925  \\
            0.432  1.6976476319153575  \\
            2.382  1.5400521436672008  \\
            3.179  1.5216589421055409  \\
            2.662  1.534046996617984  \\
            3.0  1.52518107270609  \\
            0.858  1.6460830109861477  \\
            0.606  1.6972605987837157  \\
            0.47  1.6975789090535776  \\
            2.682  1.5335663292183914  \\
            3.34  1.521186184640675  \\
            1.934  1.5505085780917038  \\
            2.348  1.54070743142826  \\
            2.615  1.5350733365134797  \\
            0.916  1.6293029435280484  \\
            0.173  1.6978046643908253  \\
            1.097  1.5967138348160657  \\
            0.21  1.697886874245256  \\
            0.247  1.6978766730782093  \\
            2.287  1.542064559440657  \\
            2.186  1.5442827353821151  \\
            1.174  1.5883292122128752  \\
            2.498  1.537523119370846  \\
            1.593  1.5616066656085064  \\
            0.89  1.6369411087419152  \\
            3.311  1.5212421166075576  \\
            2.628  1.5347779186071888  \\
            1.568  1.5627623569417193  \\
            2.559  1.536297761083515  \\
            0.489  1.697575562165038  \\
            2.928  1.5274627117453594  \\
            3.453  1.5210771399422034  \\
            0.061  1.6978662222905507  \\
            2.017  1.5482752024649622  \\
            1.34  1.5746209005723095  \\
            2.959  1.5264016036144081  \\
            2.25  1.5429266004319302  \\
            1.273  1.5797716745137098  \\
            2.356  1.540569523264159  \\
            3.258  1.5213364085377417  \\
            3.232  1.5214141669538894  \\
            0.776  1.6828624703287227  \\
            0.714  1.6943245021893936  \\
            0.599  1.6971891819019478  \\
            0.006  1.6979032567157293  \\
            1.412  1.5701162300812332  \\
            3.201  1.521560946067759  \\
            2.561  1.5362219965237922  \\
            3.406  1.5211116996883447  \\
            2.129  1.5456323179864901  \\
            2.061  1.5471892594698218  \\
            0.174  1.6978222463208024  \\
            2.479  1.5379937453490282  \\
            1.455  1.5680480050240815  \\
            0.844  1.6513399000312103  \\
            0.648  1.6968968663043398  \\
            1.257  1.5807317636777791  \\
            3.11  1.5222557672762187  \\
            1.922  1.550789766037227  \\
            3.144  1.5218739627505806  \\
            1.414  1.5702494938955445  \\
            3.213  1.5215008453713197  \\
            1.656  1.559136887233568  \\
            0.424  1.6975894635671687  \\
            1.633  1.5601023898336341  \\
            0.625  1.6969474920824055  \\
            0.31  1.6978332664925497  \\
            3.058  1.5234356045104303  \\
            1.292  1.5780247900637716  \\
            0.681  1.6961924966136828  \\
            0.717  1.6941073985206825  \\
            1.895  1.5516255387841746  \\
            3.032  1.5240135398142975  \\
            3.419  1.5211040157618771  \\
            0.063  1.697866770497563  \\
            1.62  1.5606985045540802  \\
            2.722  1.5326537646344376  \\
            0.83  1.6549330034020049  \\
            3.45  1.5210832969321162  \\
            2.27  1.5424649891961366  \\
            2.35  1.540705884509813  \\
            1.859  1.5525463798591932  \\
            3.492  1.5210479494552076  \\
            2.262  1.5426199410688646  \\
            3.134  1.5220891214866095  \\
            1.067  1.6006851489330247  \\
            1.054  1.6027600104673208  \\
            0.816  1.6599195920362286  \\
            1.872  1.5522268659550293  \\
            2.695  1.5333085702175324  \\
            3.279  1.521296477368367  \\
            0.88  1.6384829264425806  \\
            2.281  1.542251460907497  \\
            1.676  1.5584492487842565  \\
            1.844  1.5529372174982707  \\
            2.567  1.5360965449205797  \\
            2.626  1.5347705501286197  \\
            1.031  1.605698431383278  \\
            3.31  1.5212476819654417  \\
            2.729  1.5325218902088544  \\
            2.796  1.5308836393883134  \\
            1.185  1.586874469233012  \\
            0.202  1.697862453884715  \\
            3.472  1.521061821421626  \\
            1.009  1.6101534183222463  \\
            1.239  1.5823151647939597  \\
            2.16  1.5449281901750818  \\
            0.103  1.69787714098969  \\
            2.997  1.525235492084044  \\
            2.613  1.5350867567517326  \\
            3.137  1.5218762159344592  \\
            0.602  1.6970536038855555  \\
            0.641  1.696784046595391  \\
            2.188  1.544294303084153  \\
            1.8  1.554348527701818  \\
            1.397  1.571111646005574  \\
            0.205  1.697925346951852  \\
            2.374  1.5402012412921078  \\
            2.516  1.5372380967943384  \\
            1.13  1.5929646092053182  \\
            1.004  1.6108145363174653  \\
            3.163  1.521759395909697  \\
            2.42  1.5392783053387582  \\
            2.894  1.5283620617595515  \\
            2.041  1.5476730208962497  \\
            1.559  1.5631731209104487  \\
            0.178  1.6978617083151595  \\
            1.67  1.5586825989837694  \\
            1.723  1.5568148228926666  \\
            3.486  1.5210526857703102  \\
            0.817  1.660765198273016  \\
            1.948  1.5501132305962884  \\
            1.211  1.5845247963964013  \\
            1.152  1.5904094138024212  \\
            2.122  1.5457812204226016  \\
            0.526  1.6974393802761711  \\
            3.363  1.5211725255195345  \\
            3.131  1.5220240454463954  \\
            0.742  1.6924541273110312  \\
            3.012  1.5247745163685975  \\
            3.064  1.5233001189283997  \\
            1.416  1.5701752385688885  \\
            0.664  1.696411559337864  \\
            1.566  1.5627435665266656  \\
            2.747  1.5321032680735012  \\
            1.026  1.607341013362858  \\
            2.602  1.535353909599316  \\
            3.289  1.5212810519536357  \\
            3.17  1.5217322065428627  \\
            1.011  1.6098897741011113  \\
            2.734  1.5323392255634363  \\
            1.051  1.6029916711072414  \\
            1.332  1.5750444418725507  \\
            2.133  1.5455028673163356  \\
            2.397  1.5396891893810642  \\
            2.837  1.5298492115863382  \\
            2.702  1.5330566858010821  \\
            2.458  1.538458064544022  \\
            0.75  1.689281801741531  \\
            2.176  1.544550467720504  \\
            2.435  1.5389388732879026  \\
            0.16  1.6979404831265044  \\
            0.85  1.6497421405427999  \\
            2.798  1.5308253146883346  \\
            1.706  1.5575779436946031  \\
            3.489  1.5210596626799504  \\
            0.182  1.6979385005433127  \\
            2.497  1.5376242216453047  \\
            1.993  1.548993645267044  \\
            2.021  1.5481960753399548  \\
            1.164  1.589000761012062  \\
            1.72  1.557016356260466  \\
            2.362  1.5404895436932589  \\
            0.362  1.6978381785506504  \\
            2.087  1.5466372501340178  \\
            3.356  1.521176704709002  \\
            0.831  1.6559964796166984  \\
            2.859  1.5292662926206124  \\
            2.474  1.5380758204273137  \\
            1.339  1.5748424364317053  \\
            2.054  1.5473676809416121  \\
            1.703  1.5575356864631003  \\
            2.6  1.5354267057432749  \\
            2.792  1.530994922623355  \\
            2.668  1.533946665341297  \\
            0.622  1.6969577268592368  \\
            1.081  1.5988407255308297  \\
            2.076  1.5469005045724444  \\
            0.862  1.6450015007980483  \\
            0.893  1.6358449713871568  \\
            2.91  1.5278515384974451  \\
            3.12  1.522196776855649  \\
            2.035  1.5478552880184517  \\
            1.798  1.5543971854829721  \\
            0.211  1.6978559665451787  \\
            3.04  1.5238270933661018  \\
            2.94  1.5271471375058696  \\
            0.417  1.6977513995675926  \\
            1.44  1.568907951303848  \\
            2.846  1.5296473828962573  \\
            1.87  1.5523682104344876  \\
            3.417  1.5211165920073215  \\
            1.558  1.5630763408079076  \\
            0.661  1.6962870980629685  \\
            1.629  1.5601419531427196  \\
            0.936  1.6240249018861432  \\
            3.327  1.5212092495542788  \\
            2.001  1.5487967622378969  \\
            2.145  1.545250024628633  \\
            3.02  1.5244211898356061  \\
            1.802  1.5542160899193738  \\
            0.381  1.6977040392633773  \\
            1.867  1.5523684333493954  \\
            0.347  1.6977762087354342  \\
            2.018  1.548282193906807  \\
            2.809  1.530690621260157  \\
            1.555  1.5632085138350946  \\
            0.134  1.6979223518722288  \\
            0.436  1.697717857479236  \\
            0.434  1.6976870303742992  \\
            0.099  1.6978821604653747  \\
            2.111  1.5460312756820285  \\
            2.095  1.5464284703275206  \\
            1.759  1.5555839060222714  \\
            3.325  1.5212282849278396  \\
            2.884  1.528684999348747  \\
            0.779  1.6784773510857531  \\
            0.488  1.6974991198914389  \\
            1.216  1.5844834979028997  \\
            0.507  1.6975485433049018  \\
            1.094  1.5974601048276167  \\
            1.145  1.5912633676644  \\
            0.052  1.697907777914622  \\
            3.477  1.5210610602108485  \\
            0.66  1.6967834480947652  \\
            1.623  1.560583320922405  \\
            0.445  1.697654725134493  \\
            3.102  1.5223382119933706  \\
            2.088  1.5465945628184212  \\
            3.05  1.5236166461949763  \\
            0.533  1.6975122542373509  \\
            2.213  1.5436820261437665  \\
            0.241  1.6978462121347964  \\
            1.183  1.5873417554110434  \\
            1.943  1.5502344838914943  \\
            0.52  1.697371936777817  \\
            3.414  1.5211125078675307  \\
            2.331  1.5411998234563091  \\
            1.039  1.6048702450207564  \\
            2.269  1.542442126407098  \\
            0.22  1.6978223290910452  \\
            1.459  1.5677522678963567  \\
            2.933  1.5272830220063174  \\
            0.293  1.6977918570698538  \\
            0.272  1.6977965230919299  \\
            3.068  1.5229577418039812  \\
            1.111  1.595584113058627  \\
            1.679  1.5584001366076472  \\
            0.216  1.6978736616485441  \\
            0.682  1.6963528784506838  \\
            0.979  1.6154704826921025  \\
            1.714  1.5570655169357348  \\
            0.004  1.6978739628832527  \\
            2.006  1.5485667009187134  \\
            1.363  1.5730952140572951  \\
            1.076  1.5998412219109612  \\
            1.795  1.5544937412593505  \\
            0.183  1.6978402130741737  \\
            2.659  1.5340866715814414  \\
            1.177  1.587766449663036  \\
            0.826  1.6594915627249427  \\
            0.413  1.6977294115960193  \\
            0.639  1.697056672998973  \\
            0.411  1.6976751066115232  \\
            2.871  1.528904928783614  \\
            0.035  1.6979400137219016  \\
            1.376  1.5725006221253657  \\
            2.689  1.533431653324524  \\
            2.412  1.5394103899034592  \\
            0.291  1.6977557691358078  \\
            1.984  1.5492252043322945  \\
            2.303  1.5417652504641006  \\
            2.369  1.540279943104458  \\
            1.05  1.6032542954713604  \\
            3.116  1.522196836446013  \\
            0.128  1.6978468732210448  \\
            2.922  1.5276217151577427  \\
            3.389  1.5211334346002863  \\
            3.079  1.52286299414667  \\
            2.514  1.5372250437504071  \\
            2.118  1.5458845459720711  \\
            3.015  1.5246531213408498  \\
            3.35  1.5211723924792127  \\
            0.482  1.6975250741159358  \\
            2.518  1.53709788808553  \\
            3.003  1.525108950848487  \\
            0.352  1.6976713758031403  \\
            2.069  1.5469819799744182  \\
            2.651  1.5342017124973188  \\
            0.118  1.6978893361955623  \\
            3.236  1.521405442372112  \\
            0.495  1.697598841584836  \\
            2.757  1.5318519089148726  \\
            0.443  1.6976515266534191  \\
            2.117  1.5458833535880008  \\
            1.954  1.5499527633912462  \\
            0.399  1.6975836274493765  \\
            2.947  1.5268213784010254  \\
            0.621  1.6967654836182835  \\
            1.543  1.5638253195061915  \\
            2.872  1.5289514422898647  \\
            2.427  1.5390969664781022  \\
            1.122  1.5942469861792354  \\
            1.585  1.5619028552791876  \\
            1.898  1.5514579807488877  \\
            3.265  1.5213477473895143  \\
            1.457  1.567801414292085  \\
            1.89  1.5516856476313325  \\
            1.238  1.5823115678732154  \\
            2.594  1.535534474457193  \\
            1.873  1.5521281365209039  \\
            0.068  1.6978302958017535  \\
            2.64  1.534568102263971  \\
            2.282  1.542169089250131  \\
            1.61  1.5609890221389164  \\
            1.653  1.5592545052255689  \\
            2.293  1.54190608806531  \\
            2.337  1.5409956563909344  \\
            3.362  1.5211811236024617  \\
            1.127  1.593969588342378  \\
            2.194  1.5440996151761122  \\
            2.766  1.5315945342485038  \\
            2.957  1.5266072829752098  \\
            2.546  1.5365508886699724  \\
            2.45  1.5385755461814123  \\
            0.837  1.6534323540528213  \\
            2.125  1.5456568252277925  \\
            2.172  1.544527899756564  \\
            2.196  1.5441257159787893  \\
            0.757  1.686620328161402  \\
            1.074  1.6000907179026997  \\
            1.409  1.5704336489837643  \\
            2.426  1.5391322608691445  \\
            2.764  1.5316896847574746  \\
            2.788  1.5311040686041826  \\
            3.233  1.5214097659171844  \\
            2.243  1.5429968144135549  \\
            1.935  1.5504103742825823  \\
            1.907  1.551192681469902  \\
            2.68  1.5336133541528172  \\
            1.151  1.5904789171670033  \\
            0.041  1.6979257092060445  \\
            1.504  1.5655843293755602  \\
            3.441  1.521098995383268  \\
            0.67  1.6966397905797264  \\
            2.575  1.535903575423966  \\
            0.472  1.697571925561134  \\
            2.353  1.5406277574498706  \\
            3.381  1.5211429078835423  \\
            2.806  1.5306552894634635  \\
            1.976  1.5493530484344813  \\
            3.312  1.5212382880573978  \\
            0.974  1.616518749386603  \\
            0.058  1.6979150460933818  \\
            1.766  1.555413404103096  \\
            0.802  1.6663344022133058  \\
            1.647  1.5595629188084548  \\
            1.319  1.576128604672463  \\
            2.763  1.531724113458102  \\
            1.998  1.5487571962252606  \\
            1.664  1.558842781230397  \\
            0.871  1.6406173531389674  \\
            0.308  1.6977703216825635  \\
            2.192  1.5442138050323262  \\
            0.887  1.6383560389202505  \\
            2.446  1.53871631236322  \\
            0.107  1.6979288684428337  \\
            0.073  1.6979104851158653  \\
            3.442  1.5210893871886517  \\
            2.371  1.5402488935170677  \\
            2.799  1.5307846271861965  \\
            3.496  1.5210444748321008  \\
            2.354  1.5406129073940917  \\
            2.247  1.5430120761844963  \\
            0.562  1.6974958544435006  \\
            1.757  1.5556911776566256  \\
            0.998  1.6122968417581585  \\
            0.796  1.6726249078777249  \\
            1.173  1.5885948818895619  \\
            0.825  1.6565667773513828  \\
            1.044  1.6042769772267855  \\
            2.803  1.5307399387589444  \\
            2.718  1.5327413519464521  \\
            2.002  1.5486402618131037  \\
            0.401  1.6976027059510588  \\
            3.476  1.5210598236223511  \\
            1.427  1.5696305956171108  \\
            1.065  1.601581904281664  \\
            0.409  1.6976845284626978  \\
            0.057  1.6978707997223863  \\
            1.487  1.5662905226242227  \\
            0.855  1.646977464711163  \\
            2.738  1.532259176460562  \\
            2.818  1.5303561988878358  \\
            0.444  1.6977283383961135  \\
            3.052  1.5235072269037555  \\
            0.61  1.6969802710318427  \\
            1.631  1.5602574193963876  \\
            1.093  1.5974201984528138  \\
            1.7  1.5576779192907275  \\
            2.56  1.5362003280928782  \\
            2.574  1.5360102033634302  \\
            2.297  1.541871315613565  \\
            1.229  1.5833131247323593  \\
            1.433  1.5691840761931815  \\
            0.791  1.677498847797368  \\
            0.001  1.697891331594167  \\
            0.056  1.697874623361833  \\
            0.516  1.697620329545388  \\
            1.157  1.589868684735151  \\
            2.728  1.5325796344780587  \\
            1.429  1.5693400940522142  \\
            0.269  1.6978236411118  \\
            3.452  1.521068560706919  \\
            3.067  1.5231075514816104  \\
            2.51  1.5373050995114101  \\
            2.063  1.5471131515631145  \\
            2.036  1.5478507356940956  \\
            0.914  1.6301673407803599  \\
            2.15  1.5451820562355139  \\
            1.481  1.5666645535439685  \\
            2.328  1.5411756165362436  \\
            3.114  1.5222744627720253  \\
            1.486  1.5664200144528992  \\
            2.253  1.5428458797806859  \\
            2.861  1.5293107907497903  \\
            2.756  1.5318180819586078  \\
            1.237  1.5824639558683464  \\
            2.071  1.5469522745083382  \\
            2.968  1.5261750463113908  \\
            0.155  1.6978780303930336  \\
            1.837  1.5532356494857273  \\
            3.07  1.5230340672877707  \\
            1.928  1.5506990230853457  \\
            0.693  1.69610860246166  \\
            1.927  1.5506685728844765  \\
            0.433  1.6976332158255039  \\
            2.714  1.5328698257238484  \\
            0.818  1.6597453270480147  \\
            0.299  1.697873050946051  \\
            1.761  1.555710912492112  \\
            2.737  1.5323382033429518  \\
            1.271  1.5798092590568034  \\
            0.314  1.6977715686604973  \\
            1.707  1.557307773781564  \\
            2.417  1.5393390352335488  \\
            1.918  1.5508684563395858  \\
            2.168  1.5447129387965093  \\
            1.123  1.5936227897081938  \\
            2.267  1.5425148292604378  \\
            1.032  1.6058751876981159  \\
            0.772  1.6834543229483407  \\
            0.453  1.6976606204092841  \\
            2.785  1.5311716983902222  \\
            2.249  1.5428497174247815  \\
            1.916  1.550979320656214  \\
            0.2  1.6978558423762533  \\
            0.48  1.6976461756635728  \\
            1.776  1.5550739780703713  \\
            2.209  1.5438201325968093  \\
            0.927  1.6259124955961177  \\
            0.288  1.6977699579903012  \\
            2.093  1.5465370186648537  \\
            1.737  1.556446736674789  \\
            2.094  1.5464198889093823  \\
            2.074  1.5469089482699188  \\
            3.276  1.5213027141565758  \\
            0.11  1.6978610021137854  \\
            3.293  1.5212752277286128  \\
            1.957  1.5498615733964713  \\
            1.738  1.5563208378338904  \\
            2.256  1.5427009515019083  \\
            1.751  1.5559200922224456  \\
            3.123  1.5221085207499816  \\
            1.702  1.557548015813118  \\
            2.099  1.546298084790893  \\
            0.644  1.6970012592973747  \\
            1.424  1.5696378295967044  \\
            1.583  1.5620026062918444  \\
            0.477  1.6976507455882766  \\
            0.81  1.6633285288929347  \\
            3.15  1.5218442458286758  \\
            0.067  1.6978715327433394  \\
            3.235  1.5214419263423469  \\
            2.167  1.5447303131745853  \\
            2.34  1.5409505312907994  \\
            1.649  1.5594331473407212  \\
            1.042  1.6050788387081583  \\
            2.259  1.542740069808941  \\
            0.246  1.6978389260783802  \\
            0.881  1.6381099242751307  \\
            3.151  1.5218464881943494  \\
            0.726  1.6950450790711835  \\
            2.593  1.5355257350550813  \\
            1.683  1.5581705902242042  \\
            1.298  1.5777222402381825  \\
            1.911  1.551078496282584  \\
            1.959  1.5498218475504537  \\
            1.01  1.6097287717006366  \\
            1.783  1.5548784323972635  \\
            1.334  1.5753201353828883  \\
            2.678  1.5337027734284527  \\
            2.784  1.531139506254283  \\
            2.261  1.5426134660507937  \\
            3.418  1.5211208214494156  \\
            1.966  1.549613848092093  \\
            1.413  1.5704193890590472  \\
            1.813  1.5539580118632943  \\
            3.288  1.5212675413036423  \\
            2.619  1.535016181732108  \\
            1.492  1.5661773676704676  \\
            2.137  1.5454642958678664  \\
            0.93  1.6260915583365714  \\
            1.482  1.566553873750713  \\
            1.194  1.5862876375864214  \\
            1.989  1.5490293341334203  \\
            2.984  1.5256634228089245  \\
            1.297  1.5778594227522915  \\
            3.386  1.5211455611790805  \\
            0.733  1.6941085697889182  \\
            1.236  1.582667500410768  \\
            2.415  1.53931855641437  \\
            3.443  1.5210863990834094  \\
            2.067  1.547094831201005  \\
            1.46  1.5678326607604007  \\
            1.126  1.5936274145703035  \\
            1.603  1.5612602985786055  \\
            2.491  1.5377334928302722  \\
            2.58  1.5358843489418543  \\
            0.422  1.6977535075795094  \\
            1.362  1.573303584530234  \\
            1.651  1.5593225031931364  \\
            3.267  1.5213395308394981  \\
            1.364  1.5732248022710473  \\
            0.941  1.623599179170189  \\
            3.035  1.5240461738211555  \\
            0.029  1.697917456191878  \\
            3.432  1.5210990210081248  \\
            2.639  1.5344762667656426  \\
            2.875  1.528876957429368  \\
            2.114  1.5460153090822595  \\
            1.674  1.5585979036659352  \\
            0.738  1.694051214584007  \\
            2.343  1.5408660935927698  \\
            1.162  1.5894143698756853  \\
            1.326  1.5758102666313452  \\
            3.328  1.5212056359513106  \\
            2.543  1.536614450809864  \\
            0.598  1.6971997629918845  \\
            1.188  1.5871028140300907  \\
            3.164  1.5217764235101718  \\
            2.387  1.5399874794538055  \\
            1.556  1.5632847112261283  \\
            1.435  1.5692178188980443  \\
            1.58  1.5622394363687553  \\
            0.994  1.6132307587942636  \\
            1.53  1.5644028184393963  \\
            1.99  1.548981764978205  \\
            2.794  1.530955948618033  \\
            1.224  1.5831760031475022  \\
            1.561  1.5629325678981951  \\
            0.101  1.6978831341337048  \\
            2.482  1.5379250631065662  \\
            0.206  1.697842350916986  \\
            1.557  1.5632684864605921  \\
            2.195  1.544102842580023  \\
            2.123  1.5457228182800755  \\
            1.858  1.5524737919611749  \\
            1.94  1.550293815174399  \\
            2.999  1.5251265151073021  \\
            2.566  1.536127824182272  \\
            1.246  1.5815157867763558  \\
            3.323  1.5212305835710578  \\
            0.581  1.6971991587886288  \\
            3.273  1.5213117179150126  \\
            3.305  1.521250562222323  \\
            0.092  1.6978126018452009  \\
            0.575  1.697250002742412  \\
            1.228  1.5833981772235193  \\
            2.672  1.533758552189727  \\
            2.201  1.543903131605922  \\
            1.193  1.5861713323051325  \\
            2.994  1.5253350974916207  \\
            3.19  1.5215973328014456  \\
            1.904  1.5513022496339257  \\
            3.333  1.521218726981165  \\
            3.423  1.5211062671625348  \\
            1.658  1.5592057560277288  \\
            3.275  1.521296183876814  \\
            1.789  1.5546949181765581  \\
            2.996  1.525261573546489  \\
            1.387  1.5717589291816247  \\
            3.199  1.5215438913889148  \\
            1.308  1.576900719317626  \\
            2.184  1.544368807293784  \\
            1.513  1.56507711192729  \\
            2.183  1.544298144283454  \\
            1.299  1.577707864875253  \\
            3.38  1.5211476318089787  \\
            2.174  1.5446815709442723  \\
            0.748  1.6923374982994615  \\
            1.524  1.5646637793633238  \\
            1.613  1.5608385195606247  \\
            1.138  1.5922212108676321  \\
            3.371  1.5211607014461763  \\
            2.339  1.5410003802931052  \\
            0.379  1.6977562019667978  \\
            0.467  1.6976288515461826  \\
            2.105  1.546217623640057  \\
            0.249  1.697823586839758  \\
            1.29  1.578248657148865  \\
            1.083  1.5989826141561578  \\
            1.711  1.557272361435909  \\
            0.258  1.6978364419957819  \\
            3.207  1.5215213429447916  \\
            2.222  1.5434889174347672  \\
            2.283  1.5421426329314503  \\
            3.39  1.5211330319913265  \\
            2.258  1.5427345502973082  \\
            0.152  1.697927747123699  \\
            2.553  1.5364165505698395  \\
            3.3  1.5212524558883231  \\
            0.388  1.6977227161154191  \\
            1.733  1.556443070214264  \\
            3.304  1.5212644962176618  \\
            0.571  1.6973500035779878  \\
            2.208  1.5438617290813406  \\
            1.857  1.5527240284764294  \\
            0.231  1.6978515617880099  \\
            1.894  1.5516006386028136  \\
            0.296  1.697838122127586  \\
            1.222  1.5839148193861292  \\
            0.745  1.69117955090341  \\
            2.086  1.5466104426906997  \\
            1.22  1.5838734511065151  \\
            3.411  1.5211214136225106  \\
            1.77  1.5553303571101416  \\
            2.529  1.5368974627169925  \\
            0.248  1.697820553217221  \\
            3.059  1.5233370816483258  \\
            1.648  1.5595020300094833  \\
            2.887  1.5285712638288003  \\
            0.3  1.69782372081917  \\
            0.219  1.6978115176026798  \\
            3.229  1.521440568696349  \\
            2.392  1.5398133295217398  \\
            3.469  1.521072106793036  \\
            2.705  1.5331206403389543  \\
            1.71  1.5573049747814274  \\
            0.698  1.695358868183331  \\
            1.821  1.5536416342228503  \\
            0.536  1.6974953145076095  \\
            2.322  1.5413194120219185  \\
            1.002  1.611069282323471  \\
            3.335  1.521216356349921  \\
            0.926  1.6267549650318778  \\
            2.608  1.535221178967315  \\
            1.881  1.5518783261549784  \\
            0.268  1.6978293440615717  \\
            3.141  1.5219395853883535  \\
            3.319  1.5212379864950951  \\
            0.567  1.6971243222148962  \\
            1.671  1.5586311302349256  \\
            0.14  1.6979051785834847  \\
            0.616  1.6971465947213675  \\
            0.228  1.6978747173460909  \\
            0.257  1.6978392426920794  \\
            1.037  1.6049815049748126  \\
            2.862  1.529173352191327  \\
            2.132  1.545499676246625  \\
            1.028  1.6069948410405173  \\
            3.487  1.521051386102541  \\
            0.542  1.6974489765330214  \\
            2.644  1.5344426772438973  \\
            2.631  1.534693803614859  \\
            0.335  1.6978069106916784  \\
            3.491  1.5210475047556702  \\
            1.366  1.5728903625472765  \\
            1.929  1.5505591004371344  \\
            0.888  1.636498368164544  \\
            2.057  1.5472835702031644  \\
            2.625  1.5348706890592636  \\
            0.806  1.6651956225916995  \\
            2.777  1.5313776681838858  \\
            0.653  1.6963365162732051  \\
            1.232  1.5827008114730685  \\
            0.287  1.6978190870740009  \\
            0.671  1.6965791925488334  \\
            0.617  1.696846546327884  \\
            0.394  1.697696170200622  \\
            1.398  1.571108546330185  \\
            1.199  1.5859916810119792  \\
            0.274  1.6978773435669898  \\
            2.197  1.5439913746415748  \\
            1.368  1.5729660034142954  \\
            0.95  1.6212585477853523  \\
            3.316  1.5212396342671635  \\
            1.955  1.5499461171610596  \\
            3.285  1.5212936794472185  \\
            0.678  1.695961574968594  \\
            1.201  1.5854033630297732  \\
            0.958  1.6191131174374451  \\
            2.08  1.546788959739486  \\
            0.778  1.6817911067716216  \\
            2.135  1.5454585136034478  \\
            1.253  1.5812436438386483  \\
            1.786  1.5547335047446447  \\
            1.777  1.5550535483022656  \\
            2.622  1.534946112215923  \\
            0.082  1.6978897401162765  \\
            0.968  1.6174786300141895  \\
            3.132  1.521948172272626  \\
            1.931  1.550634785515441  \\
            0.108  1.6978821819158854  \\
            2.603  1.5352865384886383  \\
            2.748  1.5321018753197835  \\
            2.098  1.5463990772161071  \\
            0.971  1.6169226597179225  \\
            1.33  1.5753597344019026  \\
            0.683  1.6956964841561295  \\
            3.138  1.5219622871021552  \\
            0.059  1.6978795057050766  \\
            0.367  1.697790310691661  \\
            2.298  1.5418506866484722  \\
            0.332  1.6977650296586109  \\
            0.555  1.697399174425183  \\
            0.559  1.6974638783598488  \\
            0.486  1.69754076207448  \\
            1.464  1.5675912760475668  \\
            3.367  1.5211695462439343  \\
            0.365  1.6977695353039342  \\
            2.07  1.5470050064947953  \\
            2.736  1.5322875625164587  \\
            1.831  1.5534042573010733  \\
            0.513  1.6975032247486248  \\
            1.994  1.548794878705128  \\
            2.332  1.541073302354639  \\
            1.071  1.6007849833786327  \\
            1.843  1.5531427895200463  \\
            2.027  1.5480618095764083  \\
            3.458  1.5210716665097348  \\
            0.854  1.647030750740665  \\
            1.869  1.5522601672163163  \\
            1.335  1.5752365993259243  \\
            0.146  1.6978686323556367  \\
            1.688  1.5580423551042688  \\
            3.287  1.521277659553611  \\
            1.383  1.5721370580809633  \\
            0.519  1.6976139370528531  \\
            0.078  1.6978680504312038  \\
            2.104  1.5461654547419579  \\
            2.385  1.5399797597526304  \\
            2.439  1.5388822850990551  \\
            3.344  1.5212065834776536  \\
            0.873  1.6414689759204697  \\
            2.344  1.540883983335708  \\
            3.396  1.5211339115549238  \\
            0.583  1.6973580606260354  \\
            1.727  1.556704967261497  \\
            0.883  1.6383880779399027  \\
            0.313  1.697690526086371  \\
            1.582  1.56213939917684  \\
            0.532  1.6974539661484112  \\
            1.053  1.6031421204995044  \\
            2.787  1.5310890125535308  \\
            0.469  1.6976812639554766  \\
            2.009  1.548487838632375  \\
            2.523  1.5370678650424245  \\
            3.205  1.5215043016349967  \\
            0.938  1.6240710597763481  \\
            0.076  1.697860349268579  \\
            1.987  1.5490927114449011  \\
            2.455  1.5384824771019427  \\
            0.666  1.6963718531070515  \\
            0.84  1.6517666346993995  \\
            3.065  1.523076578491008  \\
            0.565  1.6974927600363405  \\
            2.852  1.5295114285799496  \\
            2.033  1.5479066854843768  \\
            0.637  1.6969719298902683  \\
            2.279  1.5422521735045072  \\
            2.731  1.5324488178988493  \\
            3.139  1.521985013965111  \\
            3.347  1.5212031810003057  \\
            0.592  1.6972349575636503  \\
            2.0  1.5487734493929968  \\
            1.975  1.5493797130180296  \\
            0.047  1.6979022979188585  \\
            3.351  1.5211807257329004  \\
            2.919  1.5276572316137484  \\
            2.357  1.5405784574965407  \\
            1.609  1.561006499971865  \\
            2.124  1.5456887935876815  \\
            3.331  1.5212034966496217  \\
            2.416  1.5393130671283792  \\
            1.956  1.5498551441588095  \\
            0.326  1.6977548769631539  \\
            2.11  1.5460978743078673  \\
            3.193  1.521580109399196  \\
            3.08  1.5227748950244857  \\
            3.343  1.5211986017869954  \\
            3.231  1.5214299882809432  \\
            2.576  1.5359330707089391  \\
            0.943  1.6223493722034492  \\
            0.715  1.6942772249180267  \\
            1.705  1.557541483345239  \\
            2.578  1.5358742321893402  \\
            2.193  1.5440884745183825  \\
            0.613  1.6969303068116568  \\
            2.484  1.537873646420799  \\
            1.426  1.5696581083622432  \\
            3.023  1.5244026267821003  \\
            1.384  1.5716496039648853  \\
            2.937  1.5271962982477643  \\
            3.091  1.5225687267460426  \\
            2.761  1.5317761915155435  \\
            2.238  1.5431285438096884  \\
            0.518  1.6974542262883454  \\
            2.1  1.5462967613354535  \\
            1.663  1.558867765194183  \\
            1.223  1.583911630426806  \\
            2.838  1.529860840155572  \\
            2.515  1.5371893916322008  \\
            0.64  1.69673892784243  \\
            0.341  1.6977438332607413  \\
            1.137  1.5921854049093305  \\
            0.92  1.6275413348792585  \\
            0.539  1.6974483483286658  \\
            1.886  1.5518240073949012  \\
            3.413  1.5211119800249666  \\
            0.746  1.6913168017131623  \\
            1.069  1.6008283422252587  \\
            0.94  1.6232305310956134  \\
            1.953  1.549972861464635  \\
            1.213  1.5844972528979975  \\
            1.146  1.5908205280117966  \\
            3.368  1.5211625806318056  \\
            0.204  1.6978610516750405  \\
            1.803  1.5541445052464582  \\
            2.089  1.5465235921145462  \\
            3.37  1.5211429256549878  \\
            0.747  1.6892192239618302  \\
            2.572  1.5359660831869413  \\
            0.588  1.6972251143063926  \\
            1.192  1.5865300063876797  \\
            1.359  1.5736245742340051  \\
            0.952  1.6204650806078589  \\
            0.813  1.663016474359299  \\
            3.112  1.5222177213609545  \\
            0.074  1.6979000743887056  \\
            2.366  1.540392648813105  \\
            2.229  1.5433574649915005  \\
            2.955  1.5266497138571253  \\
            0.767  1.68775607796192  \\
            0.911  1.6301766589777258  \\
            1.373  1.5727666649616832  \\
            3.444  1.5210808175355595  \\
            3.095  1.5225755387233089  \\
            2.367  1.5403726868673346  \\
            2.274  1.5423871263511983  \\
            0.96  1.618861624282605  \\
            1.202  1.5856994641610156  \\
            3.216  1.521495034392237  \\
            0.393  1.6977620340128583  \\
            0.395  1.6977004297993425  \\
            2.899  1.5281676012214058  \\
            2.199  1.5440448734053582  \\
            3.104  1.522310165795836  \\
            2.423  1.5392141846212712  \\
            0.407  1.6977319377316904  \\
            0.359  1.697765986246711  \\
            0.662  1.6968449637165228  \\
            2.601  1.5353735146099803  \\
            1.415  1.5701913041673954  \\
            0.321  1.6977189293269084  \\
            0.079  1.6978710891604987  \\
            2.831  1.5300314040817804  \\
            1.915  1.5510103355642575  \\
            1.269  1.5797708718570824  \\
            2.358  1.5405246023346026  \\
            1.889  1.551723785499116  \\
            1.171  1.5885673037366363  \\
            1.826  1.5534831586263242  \\
            0.168  1.6978998600044894  \\
            0.255  1.6977583288725138  \\
            2.712  1.5328845430125313  \\
            0.285  1.6977778776762364  \\
            1.861  1.5525467385208882  \\
            0.117  1.6978847918758768  \\
            3.092  1.522563774477938  \\
            1.113  1.5950312163203493  \\
            2.235  1.5432177341719617  \\
            1.865  1.552408042125552  \\
            0.221  1.6978899459191876  \\
            1.234  1.5828868145983959  \\
            1.977  1.5493115278336513  \\
            3.189  1.5215935423021671  \\
            2.62  1.5349709968585772  \\
            0.851  1.6481196638599818  \\
            3.318  1.5212299613078553  \\
            3.22  1.5214792093580305  \\
            1.444  1.5687116848307114  \\
            2.051  1.5474997340204666  \\
            1.746  1.5560519428307806  \\
            0.692  1.6955867606633415  \\
            1.204  1.5854259459487576  \\
            1.385  1.5720335968773593  \\
            1.85  1.5528477891559331  \\
            2.148  1.5451680181927494  \\
            2.789  1.5310678995689788  \\
            1.381  1.572152437233475  \\
            3.119  1.5221986330117454  \\
            0.594  1.6971755849879337  \\
            2.085  1.5466641323883978  \\
            2.749  1.532012548037849  \\
            0.457  1.6976393520439794  \\
            2.72  1.5327470868413058  \\
            0.787  1.6751173718596806  \\
            2.711  1.5329063057496988  \\
            2.449  1.5386212583871919  \\
            1.799  1.5543397348055865  \\
            0.674  1.6962552399433097  \\
            0.019  1.6978651500590252  \\
            0.176  1.6978621731253978  \\
            2.363  1.5404355882785392  \\
            0.003  1.697884020400505  \\
            0.656  1.6966040265641795  \\
            1.941  1.5502227655942142  \\
            0.114  1.697892619551964  \\
            1.26  1.5805441177863029  \\
            0.177  1.6978300845481722  \\
            1.294  1.5778393648624267  \\
            2.751  1.5319599470330396  \\
            0.608  1.6972375689760506  \\
            1.591  1.5617354519363176  \\
            2.525  1.5370523035058428  \\
            1.855  1.552660617635101  \\
            1.944  1.5502399074305713  \\
            0.786  1.6775243978614018  \\
            1.724  1.5567542285845308  \\
            0.055  1.69790590578696  \\
            2.853  1.5294989097913176  \\
            1.477  1.5670715054678455  \\
            1.06  1.6012999709751332  \\
            2.658  1.534069170930098  \\
            0.885  1.638291465238935  \\
            1.897  1.5514094900885296  \\
            2.904  1.5280620945471903  \\
            0.652  1.6966399415557711  \\
            2.459  1.538444558109611  \\
            3.366  1.5211627328669401  \\
            1.547  1.5636015473502982  \\
            0.713  1.694140413715397  \\
            3.018  1.5245139621192352  \\
            0.175  1.6978690286622624  \\
            0.78  1.6744047852243267  \\
            1.447  1.5685387575676175  \\
            0.222  1.697800108723385  \\
            2.277  1.5423076222581136  \\
            0.46  1.6976149682961594  \\
            1.12  1.593917822921383  \\
            2.642  1.534456413033204  \\
            3.008  1.5250175788964386  \\
            0.781  1.6775148158471658  \\
            2.733  1.5323626023461456  \\
            2.043  1.5477028797187138  \\
            1.105  1.595764303010743  \\
            2.468  1.5381626676066467  \\
            2.985  1.5257111511162342  \\
            2.883  1.528732036819422  \\
            2.936  1.5272466805259108  \\
            1.132  1.5925330367077966  \\
            2.903  1.5281368220370344  \\
            3.484  1.5210477354655285  \\
            1.095  1.5970362572033177  \\
            0.415  1.6976171322543174  \\
            2.18  1.5444417289499959  \\
            1.806  1.5541891929450344  \\
            0.566  1.6972931561743312  \\
            0.546  1.69740774720302  \\
            1.467  1.5672995865982586  \\
            2.965  1.5262793003256214  \\
            2.752  1.5319803992802417  \\
            1.667  1.55885388495227  \\
            1.466  1.5675287621867333  \\
            1.971  1.5495548867225057  \\
            1.465  1.5675432077746967  \\
            2.623  1.5348732468462822  \\
            0.676  1.696384708556581  \\
            0.568  1.6973898397925284  \\
            1.995  1.548902367319877  \\
            2.99  1.5255159108681924  \\
            1.321  1.575927325615594  \\
            0.412  1.6977808100152425  \\
            2.779  1.5312735228772232  \\
            0.612  1.6971179490290753  \\
            1.052  1.6032130636854287  \\
            1.972  1.5494430960383194  \\
            2.472  1.5381249451359058  \\
            0.496  1.697569024232359  \\
            1.17  1.5885916576648895  \\
            1.866  1.5523447045074166  \\
            1.469  1.5674467226112467  \\
            2.289  1.542086325918579  \\
            3.488  1.5210512641989655  \\
            0.739  1.693713062090221  \\
            2.692  1.5334105724809308  \\
            0.391  1.6977381353793903  \\
            1.279  1.5790333726314376  \\
            1.732  1.5565597860999518  \\
            3.439  1.5210964028770275  \\
            2.552  1.5363931930182018  \\
            0.123  1.697915814161927  \\
            0.19  1.6978289055605216  \\
            3.182  1.5216713381577227  \\
            2.024  1.5481090106456648  \\
            1.686  1.5581434398795657  \\
            3.479  1.5210518147710506  \\
            0.596  1.697209144394698  \\
            0.946  1.6217378152059887  \\
            1.985  1.5490275203194799  \\
            1.124  1.593428871594873  \\
            1.129  1.592970941560568  \\
            0.405  1.697717308600775  \\
            2.485  1.537833479931866  \\
            1.461  1.5675851679197434  \\
            2.396  1.5397919284901902  \\
            0.427  1.6976181363347522  \\
            2.743  1.5321780470766861  \\
            1.923  1.5507090012445195  \\
            3.465  1.5210707088094777  \\
            0.589  1.6971284018120978  \\
            2.532  1.5369005023898936  \\
            3.326  1.5212195836674691  \\
            3.307  1.5212583792938519  \\
            0.33  1.6977689876095432  \\
            2.604  1.5353386275808123  \\
            0.084  1.6978009509184206  \\
            1.189  1.5868614572195217  \\
            2.395  1.5397532096148803  \\
            3.295  1.5212702092494572  \\
            2.471  1.5381161280283588  \\
            3.41  1.5211181533618072  \\
            2.405  1.5394671945871286  \\
            2.433  1.5389387471457545  \\
            3.149  1.5219368384859995  \\
            1.823  1.5536155726806402  \\
            2.244  1.5429481018894395  \\
            2.953  1.5266283429426093  \\
            3.098  1.5224962043193921  \\
            1.967  1.549605770049246  \\
            0.657  1.6969551717877642  \\
            1.149  1.5908221509473308  \\
            0.438  1.6976622532221188  \\
            2.691  1.5333675669710551  \\
            0.309  1.6977777968919112  \\
            0.74  1.6926582820309706  \\
            0.763  1.6831170701150788  \\
            2.198  1.5440139212060473  \\
            3.198  1.5215549281702547  \\
            3.43  1.521107117524239  \\
            1.548  1.5635489701657062  \\
            1.91  1.5511552043201247  \\
            1.184  1.586989436977145  \\
            2.582  1.535822367243656  \\
            0.157  1.6978499311864028  \\
            1.175  1.5882231581231785  \\
            0.685  1.6959282454142233  \\
            2.506  1.5373770843832364  \\
            2.008  1.548525847155542  \\
            3.456  1.5210790166570645  \\
            2.79  1.5310057585168138  \\
            2.004  1.5486820638099401  \\
            1.669  1.5587748675316113  \\
            0.376  1.697704005623505  \\
            3.33  1.5212216553345648  \\
            0.636  1.6970552274923578  \\
            2.503  1.5374322667574691  \\
            2.368  1.540327050807947  \\
            0.97  1.6170186436522356  \\
            3.084  1.5227140077423111  \\
            2.916  1.5277505454677365  \\
            1.147  1.5908740420105751  \\
            1.168  1.5888490851543664  \\
            0.048  1.6978832365499168  \\
            2.37  1.5402737992426734  \\
            0.942  1.6225672344837514  \\
            3.029  1.5241936269265008  \\
            0.265  1.6978230554047657  \\
            1.787  1.5547851622179683  \\
            1.043  1.604905569901034  \\
            1.515  1.5651822120099952  \\
            2.663  1.5339912454116669  \\
            3.226  1.5214429831950154  \\
            1.677  1.558560184188783  \\
            1.323  1.5759243048496812  \\
            1.594  1.5615786103264506  \\
            2.24  1.5431261873117064  \\
            1.595  1.561689652012537  \\
            1.979  1.5493117063651196  \\
            2.318  1.5414632065572647  \\
            2.967  1.526282567619447  \\
            2.025  1.5480843753617577  \\
            0.345  1.6977137628898415  \\
            0.642  1.6967839125967463  \\
            0.574  1.6973894250959325  \\
            1.278  1.57906257305291  \\
            0.307  1.6977822693060232  \\
            2.04  1.5477839739091892  \\
            3.186  1.521596925960435  \\
            1.876  1.5519991949846017  \\
            1.65  1.559538578038134  \\
            3.341  1.521200935919527  \\
            2.159  1.544905412125194  \\
            2.068  1.5470648392399118  \\
            2.977  1.5259101844967735  \\
            0.978  1.6152842776977328  \\
            0.185  1.697900169675102  \\
            2.149  1.5451462194009995  \\
            2.128  1.545605349386973  \\
            1.736  1.5564347249351103  \\
            0.075  1.6978828665142716  \\
            1.715  1.557099154373994  \\
            0.934  1.6250233650243602  \\
            2.703  1.5331794178005071  \\
            1.804  1.5542353508865376  \\
            0.878  1.6391728652653639  \\
            1.378  1.5722266640631224  \\
            2.944  1.526936857425722  \\
            2.683  1.5335797678896461  \\
            0.224  1.6978340982561744  \\
            0.161  1.6978757542430418  \\
            2.896  1.5283412447170905  \\
            0.863  1.6450904804000404  \\
            0.492  1.697536828763303  \\
            2.205  1.5438371053819469  \\
            1.74  1.556156125353892  \\
            1.884  1.5519022885901008  \\
            2.719  1.532714526144098  \\
            0.553  1.69746764359636  \\
            0.576  1.6973617593994936  \\
            1.628  1.560345528869104  \\
            0.634  1.6968072763506405  \\
            1.386  1.5717865815764147  \\
            2.908  1.5279581291076163  \\
            2.991  1.5254533506238521  \\
            1.793  1.554657385885228  \\
            2.045  1.5476294033432034  \\
            0.451  1.6976812112639155  \\
            3.365  1.521170368388974  \\
            3.214  1.5215081760208524  \\
            1.509  1.5653619313535345  \\
            2.064  1.5470915566161845  \\
            2.241  1.5430860010969192  \\
            1.029  1.6062934237657163  \\
            3.451  1.5210817672192238  \\
            1.986  1.5490650019673957  \\
            3.078  1.5229076738286556  \\
            2.06  1.5471674620782627  \\
            1.689  1.5580897256135082  \\
            1.303  1.5772979779718341  \\
            1.131  1.5926640349997727  \\
            1.329  1.5756190742215672  \\
            2.717  1.5328052421494833  \\
            1.604  1.5611562260993326  \\
            1.88  1.5520484123108604  \\
            0.584  1.6971466925419865  \\
            2.914  1.527911979359435  \\
            2.494  1.537697784059169  \\
            0.811  1.6650624166084345  \\
            2.589  1.5355989584941443  \\
            1.827  1.5534796547736258  \\
            1.187  1.586993372724944  \\
            2.365  1.5404721016169285  \\
            2.541  1.5366675880303895  \\
            1.838  1.5531486200657802  \\
            3.031  1.5241810652089227  \\
            2.802  1.530692491145462  \\
            2.774  1.531434403308766  \\
            3.256  1.521366628949725  \\
            0.72  1.6943051626262025  \\
            1.794  1.5545684661258634  \\
            3.083  1.52271292857472  \\
            2.834  1.5299500004784543  \\
            0.194  1.6978050276379528  \\
            1.349  1.5740762904084382  \\
            0.877  1.640004577734287  \\
            1.552  1.5634212168062185  \\
            2.119  1.545922588493572  \\
            1.494  1.5658249082961415  \\
            3.081  1.5228869635121547  \\
            1.86  1.55259698747228  \\
            1.848  1.5529221258646846  \\
            0.217  1.6978781636255855  \\
            0.225  1.6978495568303842  \\
            2.127  1.5456812956911041  \\
            0.517  1.6973332628715354  \\
            1.602  1.5613475789862516  \\
            0.151  1.6978801834456263  \\
            1.919  1.550938231195645  \\
            3.165  1.521776421478281  \\
            2.411  1.5394112029871903  \\
            1.342  1.5745162446964618  \\
            0.71  1.695348855938402  \\
            0.331  1.697751445331734  \\
            2.892  1.5284357880018908  \\
            1.365  1.5731521157552495  \\
            1.952  1.5499299300492115  \\
            1.324  1.5759936109276806  \\
            1.996  1.548805042768415  \\
            3.107  1.5223966168974308  \\
            3.393  1.521130603256748  \\
            2.786  1.5312547089282238  \\
            0.193  1.6978117309463205  \\
            1.565  1.5629033433958441  \\
            2.906  1.527992815675149  \\
            0.548  1.6974029763043958  \\
            1.025  1.607109774957968  \\
            3.16  1.5218096874243707  \\
            0.104  1.6978476114290475  \\
            2.029  1.5480607090448606  \\
            1.665  1.558848707413183  \\
            2.432  1.5389761669444588  \\
            2.974  1.525985348885176  \\
            3.21  1.5215012015175065  \\
            0.065  1.697885963717063  \\
            0.384  1.6976597127183153  \\
            3.09  1.5226138320310905  \\
            1.936  1.5504899601253803  \\
            0.319  1.697753566269565  \\
            0.805  1.6668755194407292  \\
            3.401  1.5211238554319495  \\
            0.454  1.6976193119582825  \\
            2.457  1.5384756953348047  \\
            3.121  1.522097108855042  \\
            1.905  1.5512766208625433  \\
            3.261  1.5213371912172122  \\
            0.315  1.697825971719201  \\
            1.235  1.5822891489471682  \\
            0.582  1.6970346020121294  \\
            0.466  1.6977091034401288  \\
            2.46  1.5384182474893042  \\
            2.056  1.5473737272504524  \\
            1.214  1.584247070500888  \\
            2.664  1.5340343734445256  \\
            0.56  1.6974387505750774  \\
            2.409  1.539371789432974  \\
            2.53  1.5369194347880197  \\
            0.57  1.6973953778387871  \\
            2.864  1.5292012563153958  \\
            2.171  1.5445931517601443  \\
            2.325  1.541205563101863  \\
            1.089  1.5978224194262187  \\
            2.268  1.5425149371266451  \\
            1.272  1.5795085812049963  \\
            1.284  1.578814837309793  \\
            0.947  1.6215090629012598  \\
            1.502  1.5657064519768835  \\
            2.138  1.5454144068185196  \\
            1.442  1.5686496042649283  \\
            2.522  1.5370344218481609  \\
            3.221  1.5214581235282265  \\
            2.206  1.5439049405139431  \\
            0.691  1.6960293745848278  \\
            1.598  1.5613059002667566  \\
            2.643  1.5344483114736822  \\
            0.124  1.6979054625848218  \\
            1.428  1.5694151664423845  \\
            0.214  1.69784500919166  \\
            1.549  1.5635967621508315  \\
            1.375  1.5726245780155435  \\
            1.523  1.5649233867711831  \\
            1.544  1.5637197280347739  \\
            2.941  1.5269852668407826  \\
            1.725  1.556737976890174  \\
            0.097  1.6979055484941727  \\
            1.085  1.5983977273254408  \\
            1.668  1.5587308831460553  \\
            0.26  1.6977958446832841  \\
            1.402  1.5709737397771268  \\
            1.661  1.5591301777502737  \\
            2.251  1.5429356664962885  \\
            0.903  1.6317751152306115  \\
            0.276  1.697768312425711  \\
            0.201  1.6978268067760152  \\
            1.091  1.597890919233349  \\
            2.926  1.527412364956751  \\
            1.077  1.5993886342161658  \\
            2.811  1.5305827397673235  \\
            0.604  1.6970204280839232  \\
            2.835  1.5299293154886464  \\
            1.064  1.6007800497574978  \\
            0.988  1.6133760114324953  \\
            1.438  1.5689181750154515  \\
            2.486  1.537910115938043  \\
            1.818  1.5537692543446198  \\
            2.266  1.5425643332533105  \\
            3.097  1.5225478741269494  \\
            1.537  1.564029764596791  \\
            0.396  1.6977493773537242  \\
            0.03  1.697838208092438  \\
            1.497  1.5658780450573044  \\
            1.97  1.5494636885627546  \\
            0.17  1.6979024399012255  \\
            2.316  1.541442417422526  \\
            2.059  1.5472604055719492  \\
            1.772  1.5552322086633936  \\
            1.587  1.561903402447206  \\
            1.68  1.558241568907984  \\
            3.188  1.5216099009052084  \\
            0.392  1.6976769452401639  \\
            2.921  1.5275643914912627  \\
            1.654  1.559245408313064  \\
            1.468  1.5674529822725973  \\
            1.946  1.5501161102761685  \\
            2.948  1.526827485864853  \\
            2.221  1.5434339837396691  \\
            2.042  1.54779428193122  \\
            1.045  1.6038025809887644  \\
            2.329  1.541240916382212  \\
            1.406  1.5707107579351636  \\
            2.312  1.5414925174336178  \\
            2.709  1.532991962143459  \\
            0.08  1.697813070309623  \\
            1.082  1.599261891185302  \\
            1.78  1.5548421656950058  \\
            1.536  1.5640636677025845  \\
            0.932  1.625801606198179  \\
            1.662  1.5589420280785358  \\
            2.21  1.543781034806132  \\
            0.505  1.69752705132427  \\
            1.778  1.5550483279898275  \\
            3.094  1.5225895440816455  \\
            0.212  1.6978261673704123  \\
            1.708  1.5573461888293842  \\
            1.815  1.553895805638032  \\
            1.506  1.565491394548699  \\
            3.066  1.5232581057318901  \\
            1.116  1.5946749605120027  \\
            2.966  1.5262415866872807  \\
            0.847  1.6486091984332396  \\
            2.849  1.5295600089856765  \\
            1.361  1.573310783133947  \\
            3.426  1.5210901933504297  \\
            1.449  1.5682831584386172  \\
            2.943  1.5270406819383764  \\
            1.388  1.5718275470320806  \\
            2.327  1.5412251529335776  \\
            3.434  1.5210948494647432  \\
            0.973  1.6168245102073926  \\
            0.731  1.6927981216803827  \\
            2.334  1.541063704266092  \\
            0.931  1.6256125635558558  \\
            2.878  1.5288432354043857  \\
            3.075  1.5228738797820671  \\
            2.126  1.545673889884457  \\
            1.058  1.6022116923484757  \\
            0.364  1.697779471215549  \\
            1.212  1.5844406578014922  \\
            1.377  1.5724386127713683  \\
            3.166  1.5217510287094902  \\
            0.749  1.6909268439182312  \\
            2.306  1.541638165629167  \\
            3.255  1.521366675115267  \\
            2.355  1.5406197384644986  \\
            1.241  1.582212008473275  \\
            1.917  1.550916313480175  \\
            2.675  1.5337324874777034  \\
            0.501  1.6974657069737509  \\
            1.434  1.5689748250655207  \\
            0.163  1.6978821972330198  \\
            3.076  1.5228563069834087  \\
            1.041  1.605355188945471  \\
            0.404  1.6976842302198292  \\
            2.081  1.5467848659196788  \\
            3.334  1.521209036942483  \\
            1.809  1.5541096122577664  \\
            1.012  1.6092820501267113  \\
            2.143  1.5452504312347863  \\
            0.7  1.6954105768057448  \\
            1.013  1.6090813416361458  \\
            2.569  1.5361273579714783  \\
            1.741  1.556291868518678  \\
            2.3  1.5418301984313323  \\
            1.445  1.5687099777650657  \\
            2.101  1.5462338699570868  \\
            1.418  1.5700863418762498  \\
            2.323  1.5412850485892593  \\
            0.618  1.6972522321978136  \\
            1.0  1.6108701821824445  \\
            0.901  1.6329876970649224  \\
            1.576  1.562356936388552  \\
            2.005  1.5486793815883746  \\
            0.823  1.657914521346665  \\
            3.471  1.5210696885965151  \\
            1.75  1.555879896846175  \\
            2.839  1.5298473840075488  \\
            3.162  1.5218008257660012  \\
            0.659  1.6963439153264315  \\
            2.226  1.5434301540242072  \\
            3.369  1.5211562766677784  \\
            1.248  1.581541479340371  \\
            0.07  1.6979348708345219  \\
            0.924  1.626993981361703  \\
            2.28  1.5422405097227818  \\
            1.022  1.6076105529240008  \\
            0.485  1.6976188516663275  \\
            2.918  1.5277125089649137  \\
            0.368  1.6977349764272736  \\
            1.283  1.5786268728523696  \\
            2.54  1.5366412048040468  \\
            2.466  1.538281363679518  \\
            2.775  1.531472150571935  \\
            2.815  1.5303836800870572  \\
            1.006  1.6106272139707323  \\
            3.274  1.521325002094776  \\
            1.879  1.5519266181695757  \\
            0.515  1.6975016376450953  \\
            0.543  1.6974124696185158  \\
            3.03  1.5242508761179998  \\
            1.605  1.5612040790504234  \\
            0.263  1.6978118432304043  \\
            0.824  1.6584251971191577  \\
            3.286  1.5212681531047987  \\
            1.244  1.58170976991302  \\
            0.801  1.6676328910910685  \\
            2.454  1.5385120075934466  \\
            0.154  1.697892796701979  \\
            1.938  1.5503244311925395  \\
            2.296  1.5418816297480398  \\
            1.666  1.5588477412712407  \\
            2.47  1.5381631405586706  \\
            0.23  1.6977971740598579  \\
            2.43  1.5390643895954037  \\
            0.136  1.6978654596538867  \\
            2.978  1.525863884678239  \\
            0.45  1.6975844779610763  \\
            2.979  1.5258245697648842  \\
            3.002  1.5250537780460482  \\
            3.022  1.5244698785502995  \\
            3.046  1.5236095176118265  \\
            3.425  1.521100616278863  \\
            2.242  1.5431269099606173  \\
            0.465  1.6975839598633853  \\
            0.342  1.697809146262311  \\
            0.028  1.6978406628930354  \\
            1.463  1.5675491600243951  \\
            1.02  1.607835403350329  \\
            3.009  1.5249158590642127  \\
            1.832  1.5532730573364473  \\
            0.298  1.697810202176767  \\
            3.459  1.521085851400805  \\
            1.698  1.5577416016029317  \\
            2.935  1.5272355155518966  \\
            1.306  1.577256335160004  \\
            0.094  1.69785653966815  \\
            2.707  1.5330363588655096  \\
            0.179  1.697850642846427  \\
            2.359  1.5405179697589402  \\
            0.82  1.66117816635925  \\
            0.696  1.6955778182638108  \\
            0.005  1.6978700561089335  \\
            0.737  1.6897320021724829  \\
            1.27  1.5798409120869723  \\
            0.053  1.6979142373022553  \\
            2.946  1.5268913963129973  \\
            0.554  1.6975358791702664  \\
            2.804  1.5306967187149325  \\
            0.798  1.6704391561082241  \\
            0.945  1.6219897912140597  \\
            0.05  1.6978423245701495  \\
            1.771  1.5552487072937669  \\
            0.069  1.6979313702313168  \\
            3.105  1.5222521401121527  \\
            0.729  1.6931356383377967  \\
            1.176  1.5879918035083633  \\
            1.567  1.562770495616972  \\
            3.042  1.5238514071769618  \\
            1.588  1.5619291176374535  \\
            1.474  1.5671205995039181  \\
            0.098  1.6978708995926661  \\
            0.024  1.6978797634577958  \\
            2.725  1.5325962955300676  \\
            2.938  1.527092984512264  \\
            0.493  1.6976191991943215  \\
            0.254  1.697872967479066  \\
            2.437  1.5389272827968885  \\
            0.712  1.6941295339783486  \\
            0.948  1.6212578377170033  \\
            0.271  1.697804462259931  \\
            0.904  1.6324527699020726  \\
            2.592  1.5355658073053833  \\
            0.792  1.6729652765210028  \\
            1.839  1.5531747402645326  \\
            2.548  1.536540690794725  \\
            0.421  1.697667157768313  \\
            0.651  1.6966725049664613  \\
            1.965  1.549652490436468  \\
            3.462  1.5210728907277378  \\
            1.716  1.557195137407628  \\
            0.304  1.6977262864943379  \\
            2.581  1.5357752878214952  \\
            1.165  1.5892251833612026  \\
            2.084  1.5466783710428496  \\
            1.462  1.567683911832258  \\
            2.885  1.5286666022493622  \\
            0.91  1.6309421873697658  \\
            0.034  1.6979375226398559  \\
            1.488  1.5663885261957071  \\
            1.562  1.5630297319299569  \\
            1.033  1.6061509051742726  \\
            3.297  1.521249879199566  \\
            1.639  1.5598938086910232  \\
            1.2  1.585804644799393  \\
            1.978  1.549311604490359  \\
            3.026  1.5242297157890274  \\
            0.207  1.6978955913633285  \\
            3.101  1.5224840740026861  \\
            2.434  1.5389313419818471  \\
            2.925  1.5275562876389301  \\
            2.865  1.5291088305083607  \\
            1.655  1.5593742499420211  \\
            1.769  1.555403092943344  \\
            3.011  1.5247924723331379  \\
            2.78  1.5312956677586882  \\
            1.791  1.5546096530306104  \\
            2.207  1.5438509857310443  \\
            1.256  1.5809410033618954  \\
            2.87  1.5290180157508888  \\
            3.01  1.5249175468202443  \\
            0.145  1.6978649156101346  \\
            2.81  1.5305322721822874  \\
            2.381  1.5400877530993051  \\
            1.392  1.5714618964545208  \\
            2.972  1.5260428809553026  \\
            1.616  1.5606453459257292  \\
            3.155  1.5217925697948804  \\
            1.687  1.5579666747846628  \\
            1.206  1.585223539280746  \\
            0.867  1.6424904999302539  \\
            0.702  1.6953691318224946  \\
            3.244  1.5213991632439976  \\
            3.159  1.5218454276772413  \\
            1.318  1.576390232806901  \\
            3.218  1.5214805248953414  \\
            3.074  1.5229191233658042  \\
            3.427  1.5211021411448449  \\
            3.388  1.5211284520966042  \\
            0.156  1.6978208277945024  \\
            0.809  1.6662125787426096  \\
            3.291  1.5212716887234456  \\
            3.49  1.5210400124685184  \\
            3.088  1.5227388814740266  \\
            1.856  1.5526959799426452  \\
            2.821  1.5303226506146947  \\
            3.082  1.5228465525466173  \\
            0.385  1.697675517997885  \\
            1.456  1.5679652420944004  \\
            1.128  1.593017225039166  \\
            0.448  1.6976802581966424  \\
            0.716  1.6952945155446753  \\
            0.759  1.6853673193662686  \\
            1.634  1.5600802843051549  \\
            2.011  1.5483933433657453  \\
            1.52  1.5647663442189526  \\
            3.038  1.5240339204837892  \\
            3.167  1.5217202624857415  \\
            1.135  1.5922742458908465  \\
            0.386  1.6977825401740296  \\
            1.358  1.5736498101649556  \\
            2.869  1.529018760790974  \\
            1.119  1.594681463680882  \\
            2.469  1.5381836968999847  \\
            0.865  1.6429309517595874  \\
            0.859  1.6458347679143412  \\
            2.778  1.531327746972352  \\
            1.285  1.5788369044006154  \\
            0.769  1.6823742627751987  \\
            1.36  1.5734665568693047  \\
            1.075  1.600121888459958  \\
            2.858  1.5293787639073233  \\
            1.792  1.5545668964331087  \\
            3.245  1.5213897615444063  \\
            2.826  1.5301460184788909  \\
            2.924  1.5274312683658087  \\
            3.372  1.5211574870611966  \\
            0.679  1.6962006695513525  \\
            1.64  1.5598811418213088  \\
            2.976  1.525901832066778  \\
            0.148  1.697886545017793  \\
            1.901  1.551382810369618  \\
            2.886  1.5286478248933912  \\
            0.267  1.6977687964566308  \\
            0.991  1.6131242558161356  \\
            1.014  1.6091274381889293  \\
            2.408  1.5395029601596806  \\
            2.657  1.53417155650767  \\
            2.402  1.5396004849663785  \\
            1.161  1.5896037118110982  \\
            2.836  1.529949592589764  \\
            2.694  1.5332607091911028  \\
            2.483  1.5379165508731603  \\
            3.13  1.5220049874072996  \\
            2.931  1.527370781776193  \\
            2.112  1.5459697282391518  \\
            0.688  1.6961883915917517  \\
            0.695  1.6947879001280215  \\
            0.803  1.6694673110057145  \\
            0.232  1.6978310222989914  \\
            2.153  1.5450783293621744  \\
            0.595  1.6971269125963868  \\
            1.446  1.5684242380584268  \\
            2.456  1.5384470332060478  \\
            0.732  1.6932922525322764  \\
            2.531  1.5368792005373437  \\
            1.036  1.6049204315004724  \\
            1.115  1.5949376211711979  \\
            1.343  1.5745800533982928  \\
            1.807  1.5540373849731244  \\
            3.44  1.5210898569015388  \\
            0.419  1.6977368059866504  \\
            0.751  1.6902387988684722  \\
            2.419  1.5392807702043565  \\
            0.129  1.6979120233646068  \\
            1.5  1.5658195561672064  \\
            0.203  1.6978931654285252  \\
            0.09  1.697929789183246  \\
            3.239  1.5214232685601112  \\
            3.294  1.521263311465309  \\
            2.958  1.5265576389084685  \\
            2.727  1.532546019216206  \\
            2.302  1.5417304357336394  \\
            1.758  1.555708278231537  \\
            3.136  1.5220054166293628  \\
            2.854  1.5294048020156714  \\
            1.937  1.550371789843033  \\
            1.997  1.548907898562177  \\
            0.744  1.6892250988008644  \\
            2.31  1.5416030536913383  \\
            0.286  1.6978217367411366  \\
            1.98  1.549264190939001  \\
            1.579  1.5621389654260593  \\
            3.208  1.5215289989956764  \\
            3.48  1.521057427861469  \\
            2.49  1.5377596591867089  \\
            3.349  1.5211918549075332  \\
            2.422  1.5391997695657267  \\
            1.047  1.6036042616377095  \\
            0.872  1.642043948617824  \\
            3.126  1.5220765625001413  \\
            1.744  1.556191777988792  \\
            0.997  1.6121932466202524  \\
            3.339  1.5212123240547357  \\
            1.144  1.5914270196707103  \\
            1.399  1.5712000190100721  \\
            0.481  1.697557955741282  \\
            2.304  1.5416643639155834  \\
            2.715  1.5327922567007488  \\
            2.797  1.5308706690342626  \\
            2.911  1.5279385144940119  \\
            1.254  1.5812304230516163  \\
            0.088  1.6978735555295068  \\
            3.429  1.5211017241634774  \\
            2.109  1.5460391477835227  \\
            3.005  1.5250166907032352  \\
            2.317  1.5413454772014017  \\
            1.824  1.5535679957376305  \\
            2.533  1.536888333989996  \\
            1.614  1.5608424619814034  \\
            1.571  1.5625778068412528  \\
            0.064  1.6978869732003554  \\
            1.762  1.5554615730188324  \\
            2.646  1.534390703259705  \\
            0.233  1.697834314791265  \\
            1.645  1.559516677580756  \\
            3.143  1.5219710318485657  \\
            1.43  1.5693053466787095  \\
            2.842  1.5297431711698017  \\
            0.416  1.6977261500834928  \\
            2.688  1.533447795786929  \\
            0.905  1.6307551400668687  \\
            2.932  1.527198946240145  \\
            0.377  1.697765058056628  \\
            0.933  1.6248224254452384  \\
            2.825  1.5302004247358052  \\
            1.078  1.5988272482881165  \\
            0.799  1.671210707984645  \\
            2.563  1.5361834965538956  \\
            0.353  1.6977459807138588  \\
            2.939  1.5271152241785428  \\
            0.768  1.6871588832789417  \\
            0.36  1.6977819406924044  \\
            1.491  1.5663778253108953  \\
            2.55  1.536478432772036  \\
            3.284  1.521291585881677  \\
            1.295  1.5778254773991875  \\
            0.126  1.697907460692266  \\
            2.67  1.5338378629029312  \\
            1.314  1.5763784615288459  \\
            1.775  1.555122585137418  \\
            3.26  1.5213633498397863  \\
            1.441  1.5688286916345389  \\
            2.155  1.5450142924518575  \\
            1.348  1.5740737106758136  \\
            1.527  1.5646027876876423  \\
            2.187  1.5442562787228142  \\
            1.817  1.5536912132231238  \\
            2.214  1.5437200274060483  \\
            2.5  1.5374826593443225  \\
            0.327  1.697748688459717  \\
            1.389  1.5716372912442136  \\
            2.86  1.5292774744832744  \\
            2.828  1.530156006832045  \\
            2.12  1.5458548010811382  \\
            2.386  1.539908508604376  \\
            2.429  1.5390259728313025  \\
            1.755  1.5556504440962242  \\
            0.523  1.697475391286635  \\
            0.369  1.69768211663299  \\
            0.899  1.6339718159081127  \\
            0.53  1.697596508878673  \\
            2.324  1.5413216378361774  \\
        }
        ;
    \addlegendentry {L=20}
    \addplot[color={rgb,1:red,0.8889;green,0.4356;blue,0.2781}, name path={eecbfce2-5ad1-4b3d-bf48-99b2e4d9c532}, draw opacity={0.7}, line width={1}, solid]
        table[row sep={\\}]
        {
            \\
            0.73  1.3259736260592037  \\
            0.73  1.8950663864093147  \\
        }
        ;
    \addlegendentry {$E_F^*$}
    \addplot[color={rgb,1:red,0.8275;green,0.8275;blue,0.8275}, name path={60d24c26-fd47-406f-abbd-084b98530b08}, draw opacity={1.0}, line width={1}, solid, forget plot]
        table[row sep={\\}]
        {
            \\
            0.8  1.6879878651459206  \\
            2.1210000000000004  1.6525474632494008  \\
        }
        ;
    \addplot[color={rgb,1:red,0.8275;green,0.8275;blue,0.8275}, name path={542b7470-90f5-4b62-ab58-31f308541077}, draw opacity={1.0}, line width={1}, solid, forget plot]
        table[row sep={\\}]
        {
            \\
            0.65  1.68  \\
            0.65  1.7  \\
            0.8  1.7  \\
            0.8  1.68  \\
            0.65  1.68  \\
        }
        ;
\end{axis}
\begin{axis}[point meta max={nan}, point meta min={nan}, legend cell align={left}, title={}, title style={at={{(0.5,1)}}, anchor={south}, font={{\fontsize{14 pt}{18.2 pt}\selectfont}}, color={rgb,1:red,0.0;green,0.0;blue,0.0}, draw opacity={1.0}, rotate={0.0}}, legend style={color={rgb,1:red,0.0;green,0.0;blue,0.0}, draw opacity={1.0}, line width={1}, solid, fill={rgb,1:red,1.0;green,1.0;blue,1.0}, fill opacity={1.0}, text opacity={1.0}, font={{\fontsize{8 pt}{10.4 pt}\selectfont}}, text={rgb,1:red,0.0;green,0.0;blue,0.0}, at={(1.02, 1)}, anchor={north west}}, axis background/.style={fill={rgb,1:red,1.0;green,1.0;blue,1.0}, opacity={1.0}}, anchor={north west}, xshift={89.44mm}, yshift={-11.16mm}, width={42.22mm}, height={48.8mm}, scaled x ticks={false}, xlabel={}, x tick style={color={rgb,1:red,0.0;green,0.0;blue,0.0}, opacity={1.0}}, x tick label style={color={rgb,1:red,0.0;green,0.0;blue,0.0}, opacity={1.0}, rotate={0}}, xlabel style={at={(ticklabel cs:0.5)}, anchor=near ticklabel, font={{\fontsize{11 pt}{14.3 pt}\selectfont}}, color={rgb,1:red,0.0;green,0.0;blue,0.0}, draw opacity={1.0}, rotate={0.0}}, xmajorgrids={true}, xmin={0.65}, xmax={0.8}, xtick={{0.65,0.7000000000000001,0.75,0.8}}, xticklabels={{$0.65$,$0.70$,$0.75$,$0.80$}}, xtick align={inside}, xticklabel style={font={{\fontsize{8 pt}{10.4 pt}\selectfont}}, color={rgb,1:red,0.0;green,0.0;blue,0.0}, draw opacity={1.0}, rotate={0.0}}, x grid style={color={rgb,1:red,0.0;green,0.0;blue,0.0}, draw opacity={0.1}, line width={0.5}, solid}, extra x ticks={{0.66,0.67,0.68,0.6900000000000001,0.7100000000000001,0.7200000000000001,0.7300000000000001,0.74,0.76,0.77,0.78,0.79}}, extra x tick labels={}, extra x tick style={grid={major}, x grid style={color={rgb,1:red,0.0;green,0.0;blue,0.0}, draw opacity={0.05}, line width={0.5}, solid}, major tick length={0.1cm}}, axis x line*={left}, x axis line style={color={rgb,1:red,0.0;green,0.0;blue,0.0}, draw opacity={1.0}, line width={1}, solid}, scaled y ticks={false}, ylabel={}, y tick style={color={rgb,1:red,0.0;green,0.0;blue,0.0}, opacity={1.0}}, y tick label style={color={rgb,1:red,0.0;green,0.0;blue,0.0}, opacity={1.0}, rotate={0}}, ylabel style={at={(ticklabel cs:0.5)}, anchor=near ticklabel, font={{\fontsize{11 pt}{14.3 pt}\selectfont}}, color={rgb,1:red,0.0;green,0.0;blue,0.0}, draw opacity={1.0}, rotate={0.0}}, ymajorgrids={true}, ymin={1.68}, ymax={1.7}, ytick={{1.68,1.685,1.69,1.695,1.7}}, yticklabels={{$1.680$,$1.685$,$1.690$,$1.695$,$1.700$}}, ytick align={inside}, yticklabel style={font={{\fontsize{8 pt}{10.4 pt}\selectfont}}, color={rgb,1:red,0.0;green,0.0;blue,0.0}, draw opacity={1.0}, rotate={0.0}}, y grid style={color={rgb,1:red,0.0;green,0.0;blue,0.0}, draw opacity={0.1}, line width={0.5}, solid}, extra y ticks={{1.681,1.6820000000000002,1.683,1.6840000000000002,1.686,1.6869999999999998,1.688,1.6889999999999998,1.691,1.6920000000000002,1.693,1.6940000000000002,1.696,1.6969999999999998,1.698,1.6989999999999998}}, extra y tick labels={}, extra y tick style={grid={major}, y grid style={color={rgb,1:red,0.0;green,0.0;blue,0.0}, draw opacity={0.05}, line width={0.5}, solid}, major tick length={0.1cm}}, axis y line*={left}, y axis line style={color={rgb,1:red,0.0;green,0.0;blue,0.0}, draw opacity={1.0}, line width={1}, solid}]
    \addplot[color={rgb,1:red,0.0;green,0.6056;blue,0.9787}, name path={f798893f-8ff0-44ca-b138-8d6b41fea775}, only marks, draw opacity={1.0}, line width={0}, solid, mark={*}, mark size={1.5 pt}, mark repeat={1}, mark options={color={rgb,1:red,0.0;green,0.0;blue,0.0}, draw opacity={0.0}, fill={rgb,1:red,0.0;green,0.6056;blue,0.9787}, fill opacity={1.0}, line width={0.75}, rotate={0}, solid}]
        table[row sep={\\}]
        {
            \\
            0.259  1.6978202464730645  \\
            2.234  1.5432443898802009  \\
            2.462  1.5383769825963358  \\
            2.166  1.5447022862925264  \\
            2.851  1.5295383620852943  \\
            2.154  1.5450267722888336  \\
            0.785  1.6814900109469648  \\
            0.756  1.69100710568753  \\
            3.387  1.5211316828061647  \\
            2.509  1.5373422409031319  \\
            0.418  1.6977476048779925  \\
            2.378  1.5401454657206994  \\
            1.719  1.5570325572202053  \\
            3.463  1.521067374724601  \\
            3.169  1.5217185735994163  \\
            3.109  1.522275000484881  \\
            2.388  1.5399601418100552  \\
            2.993  1.5254522795662575  \\
            0.761  1.6853694719845065  \\
            0.721  1.6928651220999635  \\
            0.171  1.6978408641974516  \\
            2.816  1.5304141459190366  \\
            3.225  1.5214681840023925  \\
            1.672  1.558638072916439  \\
            0.58  1.697431841287041  \\
            0.437  1.697535089291652  \\
            2.819  1.5303336230704467  \\
            3.299  1.5212633768442638  \\
            1.249  1.5816449394200904  \\
            2.767  1.5315778755662572  \\
            0.127  1.6979118250127356  \\
            0.626  1.6968046337366396  \\
            1.088  1.598142538001969  \\
            0.99  1.6134737485222188  \\
            0.935  1.6248852311262605  \\
            2.116  1.545879132199836  \\
            1.512  1.5652591684268056  \\
            3.069  1.5231191334393779  \\
            3.063  1.5231141618417647  \\
            0.55  1.6973954829333109  \\
            2.191  1.5442333208245147  \\
            0.892  1.634193949293939  \\
            0.889  1.636000468811691  \\
            1.255  1.580769010072519  \\
            2.963  1.5264259190499856  \\
            0.197  1.6977796215486143  \\
            2.276  1.5423473669583827  \\
            1.276  1.579520609290797  \\
            2.511  1.5373575699834567  \\
            1.322  1.5759139778997642  \\
            1.001  1.6110549513241263  \\
            3.263  1.5213535979373254  \\
            2.565  1.5361680391623342  \\
            3.034  1.5241884693467849  \\
            3.044  1.52385672976882  \\
            3.466  1.5210608334942346  \\
            2.255  1.5428047402834506  \\
            0.918  1.6279753377920279  \\
            0.172  1.6978673159881068  \\
            0.762  1.684697154321532  \\
            2.315  1.5414592073289883  \\
            2.513  1.5372940551765564  \\
            1.288  1.5782783726490517  \\
            3.254  1.5213599885982192  \\
            2.891  1.528513556995719  \\
            1.423  1.5696567333092122  \\
            0.147  1.6978859459601532  \\
            0.499  1.697537490174692  \\
            2.26  1.5426637879145515  \\
            2.246  1.542885604225657  \\
            0.215  1.697801163218106  \\
            0.834  1.655478565721448  \\
            1.805  1.5541287194525162  \\
            3.313  1.5212223120388002  \\
            2.147  1.5451238081287368  \\
            2.463  1.5382808556610081  \\
            2.667  1.5339889651620753  \\
            1.678  1.5583940496264161  \\
            3.321  1.521220924186742  \\
            0.181  1.697853030639084  \\
            1.574  1.562399301306102  \\
            3.195  1.5215280988725124  \\
            3.174  1.5217008279718611  \\
            2.677  1.5337353974407006  \\
            0.125  1.6979083301156115  \\
            0.462  1.6976805017782677  \\
            2.989  1.5255283538495183  \\
            3.053  1.5236112345670785  \\
            0.982  1.6146456965708114  \\
            0.007  1.6978737761283795  \\
            3.454  1.5210711732302469  \\
            2.444  1.5387046447831252  \\
            1.289  1.5783650956410091  \\
            3.036  1.5239982994114518  \\
            1.369  1.5728868776152258  \\
            1.584  1.5619857797409313  \\
            3.345  1.5211978254217957  \\
            2.982  1.5257827572689309  \\
            2.286  1.542059403602439  \\
            1.999  1.548820417156781  \\
            0.348  1.6977777549454496  \\
            0.038  1.6978827453289689  \\
            2.139  1.545441939350156  \\
            0.743  1.6922672233888825  \\
            1.617  1.5607135841816013  \\
            3.206  1.52152865383097  \\
            0.227  1.6979265743871688  \\
            2.57  1.536002708907576  \\
            1.534  1.5643890335512758  \\
            3.428  1.521096100594318  \\
            2.108  1.5461608223095098  \\
            1.899  1.5514864263666022  \\
            3.315  1.5212453844800267  \\
            0.902  1.631956319749883  \\
            1.471  1.5672146911790807  \\
            2.066  1.5470703816426543  \\
            1.417  1.5700918206727044  \\
            1.142  1.5914990561348072  \\
            3.478  1.5210541752928644  \\
            1.722  1.5568537344260558  \\
            1.4  1.5710392807783309  \\
            0.856  1.6465024890870328  \\
            2.952  1.5266053395740853  \\
            2.568  1.5361267994667624  \\
            0.089  1.6978798673726792  \\
            1.274  1.579643945659493  \\
            1.982  1.5491818419464911  \\
            2.942  1.5269710024704994  \\
            0.886  1.637090311338126  \\
            0.091  1.6979179742397437  \\
            1.478  1.5668995336109859  \\
            1.773  1.555186247986811  \\
            2.173  1.544654603773181  \\
            2.383  1.5399974797171392  \\
            0.062  1.6979081585528928  \\
            3.248  1.521384676603107  \\
            1.518  1.5649518390662651  \\
            2.591  1.5355414693524456  \\
            2.134  1.5454682103428536  \\
            0.27  1.697814324229027  \\
            1.191  1.586350115647175  \\
            2.863  1.5291246598925194  \\
            1.521  1.5646615307181795  \\
            3.353  1.5211779905256044  \\
            3.196  1.5215571546953433  \\
            2.544  1.5366407062764398  \\
            3.395  1.521129760688739  \\
            1.166  1.5891396721892832  \\
            0.266  1.6978426653642884  \\
            1.753  1.555825284784931  \\
            0.955  1.6201632981648695  \\
            0.473  1.6975405210828378  \\
            0.106  1.6978478075774113  \\
            0.907  1.6314816159972396  \\
            0.775  1.6789902298096773  \\
            2.236  1.543211885691929  \\
            0.087  1.6978798507807669  \\
            1.18  1.5877143205940318  \\
            0.689  1.695784839436051  \\
            3.337  1.5212113124029774  \\
            0.917  1.6281671720545803  \\
            0.949  1.6213732723210854  \\
            2.7  1.5332120570650452  \\
            1.825  1.553627088820441  \\
            1.394  1.5713440948208686  \\
            1.317  1.5761028984544398  \\
            3.298  1.5212593483307542  \\
            1.062  1.600990981816864  \\
            2.856  1.529437693744415  \\
            3.402  1.5211249674332745  \\
            2.909  1.5278672754313056  \\
            2.075  1.5469120385521777  \\
            1.657  1.5592803678961993  \\
            3.25  1.5213490549468591  \\
            3.42  1.5211114181744967  \\
            2.759  1.5318286933759828  \\
            2.252  1.5428036596962236  \\
            1.932  1.550538977685741  \\
            2.351  1.5407131539410153  \\
            2.398  1.539752430165966  \\
            2.739  1.5323175370910187  \\
            2.248  1.542891617207868  \\
            1.367  1.5731653331421278  \\
            1.252  1.5812564728719893  \\
            2.115  1.5458632164204407  \\
            3.495  1.5210454976115355  \\
            2.441  1.5387407711330545  \\
            0.504  1.6975198011359027  \\
            3.398  1.521126111967818  \\
            2.146  1.5452161069199746  \\
            3.47  1.5210702907140883  \\
            0.164  1.6978510342695674  \\
            2.983  1.5259109401127031  \\
            2.152  1.545108219526764  \\
            0.043  1.6979051806883285  \\
            0.603  1.6973788570272927  \\
            2.142  1.5452629951347414  \\
            0.508  1.6974484697415366  \\
            0.965  1.6183932725585433  \\
            1.404  1.5707813364430732  \\
            2.308  1.5415858764764012  \\
            2.782  1.5312696585230081  \\
            2.847  1.5296332568258146  \\
            1.871  1.552206866578577  \\
            1.54  1.5639857039136003  \\
            0.522  1.6975474014409344  \\
            1.981  1.549227125459964  \\
            1.227  1.5831615269959602  \\
            1.099  1.5967153449480673  \\
            3.014  1.5246127510127105  \\
            2.708  1.5329312427918396  \\
            0.139  1.6978644032673689  \\
            0.62  1.6970688677304255  \\
            0.43  1.6976095863569158  \\
            1.023  1.6077584019428015  \\
            1.868  1.5523046637808067  \\
            2.556  1.5363410416296448  \\
            0.503  1.6976091677384229  \\
            2.436  1.538923484044123  \\
            0.755  1.6857695847585963  \\
            1.181  1.5874681234152634  \\
            2.772  1.5315067076813487  \\
            0.357  1.6976737647030495  \\
            3.039  1.5238771428725875  \\
            0.921  1.6272078818430133  \\
            2.294  1.5419397720186865  \\
            2.228  1.5433403362453673  \\
            2.161  1.5448540656225267  \\
            0.673  1.695828374806338  \\
            3.238  1.521410349293256  \\
            0.857  1.6459834766292745  \\
            1.612  1.5608377631513404  \\
            1.968  1.5496071216150566  \\
            1.529  1.5645892023943286  \\
            1.974  1.5493827525694486  \\
            1.754  1.5558529704716346  \\
            2.164  1.5447875970935039  \\
            2.218  1.5435618246803415  \\
            0.506  1.6975180122462972  \\
            0.4  1.697683661078072  \\
            2.742  1.5322019407602474  \\
            2.313  1.5414777841550407  \\
            2.216  1.5436264747576196  \\
            0.277  1.697822665171039  \\
            1.73  1.5566088574033023  \\
            2.384  1.5399745466054577  \\
            0.044  1.6978761778002753  \\
            1.21  1.5846972664051613  \\
            1.95  1.5500294556876473  \\
            3.118  1.522178571861527  \\
            3.061  1.523223719814995  \\
            0.861  1.644931520041318  \\
            2.89  1.5284517105215483  \\
            1.903  1.55134341468166  \\
            3.171  1.5217019622620855  \\
            2.461  1.5384066326183992  \\
            0.869  1.6422739357688325  \\
            0.577  1.6972876047966319  \\
            0.684  1.6956164784555474  \\
            0.646  1.696574256647858  \\
            3.314  1.521238838752401  \\
            1.217  1.583766478517585  \\
            2.215  1.5436743441300862  \\
            1.713  1.557147075604868  \\
            1.401  1.5711116190440821  \\
            1.531  1.5643139236455437  \\
            1.607  1.5611243767431715  \\
            2.97  1.5262490685901775  \\
            0.406  1.6976946783369558  \\
            0.572  1.6972050206699403  \\
            0.116  1.697836654153997  \\
            0.149  1.6978939918924298  \\
            0.435  1.6977211639379408  \\
            1.325  1.5756295195685135  \\
            1.694  1.5578231749435238  \\
            0.734  1.6936014178892767  \\
            0.431  1.6975570723545763  \\
            1.853  1.5527393215367833  \\
            2.88  1.5287113128267693  \\
            0.777  1.6821762551064312  \\
            1.887  1.5516824574149386  \\
            0.687  1.6957516613291497  \\
            2.487  1.537751477319869  \\
            0.195  1.6979161892520256  \\
            0.242  1.6978427776952982  \\
            0.897  1.632937454926271  \\
            2.907  1.5279928010063384  \\
            2.74  1.5323029460686646  \\
            3.252  1.5213730764474733  \\
            1.008  1.6096269734850763  \\
            0.354  1.6977249030704056  \\
            0.848  1.6501970551145038  \\
            3.062  1.5233013176065595  \\
            0.081  1.6978542255299598  \\
            2.014  1.5484304274284775  \\
            2.016  1.5483363540137285  \\
            2.102  1.546254791494004  \\
            0.852  1.6464574823725338  \\
            0.638  1.6967537887521227  \\
            2.633  1.5346634780772692  \\
            1.476  1.5669823363163962  \\
            0.512  1.6975417988845576  \\
            3.041  1.523912278489622  \\
            3.392  1.521138366808603  \\
            2.721  1.5326853937544664  \\
            1.379  1.5722830631447917  \\
            1.842  1.5530679439643404  \\
            2.478  1.5379624618694394  \\
            0.552  1.6973211404985136  \\
            2.13  1.5455658889930692  \\
            0.876  1.6392236213544786  \\
            0.158  1.6978568803890663  \\
            2.414  1.5393436999929413  \\
            3.048  1.5235731248963054  \\
            2.527  1.5369792185576914  \\
            3.308  1.5212352345962639  \\
            0.229  1.6978938206151697  \\
            0.898  1.6346603727696054  \\
            3.017  1.524594847642942  \\
            0.165  1.6978272521701276  \\
            1.554  1.563297226601684  \\
            0.44  1.6976717500545375  \\
            2.917  1.5277219099407748  \\
            2.632  1.5346937367749516  \\
            1.874  1.5521402646571962  \\
            2.671  1.5337758042258296  \\
            0.252  1.6978582180997044  \\
            1.291  1.5782792768762004  \\
            1.774  1.555127388048749  \\
            2.231  1.5432724682053198  \\
            1.514  1.5650566950467668  \\
            0.836  1.6540821733812208  \\
            1.625  1.5604261431051047  \\
            0.278  1.6978195617622327  \\
            0.244  1.6978741706330227  \\
            2.508  1.537406609106021  \\
            1.262  1.5803083711118728  \\
            2.732  1.5324502560168283  \\
            1.221  1.5839378653292886  \\
            3.27  1.521310766080244  \\
            1.419  1.5699967056959094  \\
            0.628  1.69724566567861  \\
            1.066  1.6005731009642408  \\
            2.961  1.5264036206319078  \\
            3.176  1.521712275539421  \\
            1.618  1.5607507202168158  \\
            3.227  1.5214347595537314  \\
            1.63  1.5601584738007532  \\
            1.847  1.5529993931330486  \\
            1.541  1.5639333860390485  \\
            3.057  1.523502126339382  \\
            2.377  1.5402180911172336  \\
            2.755  1.5318684305137442  \\
            1.159  1.590053717815565  \\
            0.896  1.635273646839519  \\
            2.41  1.5394911210745377  \\
            0.76  1.6856255343549622  \\
            3.32  1.5212449657520848  \\
            2.346  1.540822006221142  \\
            3.447  1.5210835675651562  \\
            3.494  1.5210502806926058  \\
            1.642  1.5597404616114596  \\
            0.63  1.697051657285522  \\
            1.756  1.5558156827662737  \\
            1.035  1.6061858308689931  \\
            2.165  1.5447974862188367  \\
            2.227  1.5433994415529886  \\
            0.066  1.6979104893247752  \\
            3.431  1.5211063943522847  \\
            2.704  1.5330806146228424  \\
            2.389  1.5398121746171511  \\
            3.301  1.5212687767659914  \\
            0.112  1.6978817871426184  \\
            2.762  1.5316801889872895  \\
            1.988  1.5490559400154624  \\
            2.881  1.5287398611458105  \\
            0.521  1.697425809074808  \\
            3.296  1.5212708359726526  \\
            3.23  1.5214287297234887  \\
            2.156  1.5450452768131815  \\
            0.764  1.687101068489801  \\
            2.493  1.537657196801368  \\
            0.04  1.6979082402678813  \\
            2.629  1.534740323084156  \\
            2.882  1.5286876736976653  \\
            1.646  1.5596919205579367  \\
            0.191  1.6978598525255815  \\
            0.363  1.6977871294795492  \\
            2.055  1.5473734220541144  \\
            2.007  1.5485174042400174  \\
            0.487  1.6976203502589187  \\
            1.589  1.5619241124589363  \\
            2.609  1.5352292306135014  \\
            1.56  1.5631363386711865  \\
            0.264  1.6978654498614534  \\
            3.115  1.5222008547779242  \\
            0.983  1.6152397017305762  \\
            2.492  1.5377031337199167  \\
            1.829  1.5534752130723062  \\
            2.189  1.544175557530864  \\
            0.095  1.6978979630565727  \\
            1.553  1.5633736925729829  \\
            0.198  1.697892215274497  \\
            2.36  1.5404202308458623  \\
            1.695  1.5577748547203134  \\
            1.24  1.5820653223812402  \\
            3.085  1.522802116547321  \\
            3.5  1.5210464775688437  \\
            0.297  1.6978263368217013  \\
            0.282  1.6977424620025223  \\
            0.289  1.6977964550584914  \\
            2.96  1.5264361124290318  \\
            0.784  1.67709259517469  \\
            2.893  1.5285187110969947  \\
            0.295  1.6977884004581518  \\
            1.875  1.5521298424769756  \\
            2.724  1.5326384134067452  \\
            1.195  1.5864670794682703  \\
            1.551  1.563393236355276  \\
            3.122  1.522185381210071  \\
            0.446  1.6976909929567134  \\
            2.237  1.5431370408898795  \\
            2.418  1.5392689169227394  \\
            1.846  1.5528714889214308  \\
            3.409  1.5211196523102792  \\
            0.937  1.6231928215930784  \\
            1.304  1.57740675995255  \\
            1.302  1.5775907355238639  \\
            3.161  1.5217985919723578  \\
            2.079  1.546824432764586  \\
            1.019  1.6080581834286247  \\
            1.545  1.5639093877386212  \\
            2.217  1.5436032442100487  \\
            1.15  1.5906542976527769  \\
            0.815  1.6614594451321312  \\
            2.681  1.5335967644853423  \\
            0.891  1.6363262178144755  \\
            1.09  1.5981948535623163  \\
            1.073  1.6000762280503278  \\
            0.632  1.6967635518779496  \\
            3.303  1.5212599988332451  \\
            3.168  1.5217313685869496  \\
            1.109  1.5949880065651294  \\
            2.495  1.537623069689047  \\
            3.212  1.521494433699487  \\
            3.246  1.5213959879631591  \\
            2.545  1.536619003613027  \\
            2.723  1.532647074327653  \\
            2.05  1.5475244885900536  \\
            3.266  1.521331981013179  \\
            2.078  1.5468118131687545  \\
            3.133  1.5220620286819113  \\
            3.142  1.5220019211843598  \\
            0.41  1.6976003891694245  \\
            0.558  1.6973546706718605  \\
            2.75  1.5320272311066128  \\
            1.421  1.56968146835493  \\
            3.242  1.521402812801543  \\
            0.719  1.693083619452179  \\
            3.211  1.5215028367741643  \\
            1.328  1.5756204006211516  \\
            2.403  1.5395797123260666  \\
            0.24  1.6978562257348995  \\
            1.087  1.5984283908336376  \\
            1.599  1.5614325042342156  \\
            2.611  1.535171441329123  \\
            2.551  1.5364463485382716  \\
            1.949  1.550059844544465  \\
            3.405  1.5211344942124954  \\
            1.156  1.5902605236489213  \\
            1.061  1.6017048996811094  \\
            2.19  1.54418135441632  \\
            0.039  1.6978477725200483  \\
            0.199  1.6978729339067018  \\
            3.415  1.5211064342491019  \\
            2.22  1.5435581640932463  \\
            0.611  1.6968456586519993  \\
            2.502  1.5374943348822063  \\
            1.311  1.5767300482606437  \\
            1.764  1.55542787458787  \\
            3.317  1.521224231753924  \\
            2.09  1.546482827859478  \\
            0.597  1.697093050748906  \\
            0.478  1.6976517053228624  \\
            1.114  1.5946932868808903  \\
            1.072  1.6002608839167605  \\
            0.985  1.6142746401102412  \\
            1.331  1.5754287633373045  \\
            3.421  1.521118097794386  \\
            1.327  1.575717749034109  \\
            2.605  1.5352380920536242  \\
            3.027  1.5242527324372477  \\
            3.449  1.5210814608065  \\
            0.709  1.6950568136879487  \\
            1.315  1.5764493206748575  \\
            2.833  1.5300056754450913  \\
            2.507  1.5374227661371245  \\
            2.464  1.5383140564921125  \\
            0.752  1.6899218885412852  \\
            1.102  1.5963770010699148  \\
            0.459  1.697623929518313  \\
            1.108  1.5955422592905488  \\
            1.734  1.5564081185519099  \\
            2.765  1.5316517676618888  \\
            1.393  1.5713183790757002  \\
            1.635  1.5600289551105957  \\
            0.703  1.6954710884833626  \\
            1.458  1.5678137361992204  \\
            3.28  1.521301917945163  \\
            1.106  1.5959802086906325  \\
            1.41  1.5702408638645209  \\
            2.394  1.5397756097614095  \\
            2.555  1.536355130633593  \\
            2.473  1.5380973293266726  \\
            0.371  1.697785478346114  \\
            3.173  1.5216967519715743  \\
            2.424  1.5391157656341208  \\
            3.338  1.521192327321352  \\
            0.283  1.6977916523461543  \\
            1.015  1.6092718327365934  \\
            2.225  1.5434146468514651  \\
            0.141  1.6978181821323322  \\
            0.458  1.6976330790456948  \\
            1.945  1.550183251312689  \\
            1.51  1.5652147962267025  \\
            0.031  1.6978778616608592  \\
            1.877  1.5521037035135914  \\
            0.344  1.697884517569016  \\
            1.136  1.5923164902849545  \\
            3.175  1.5217019327201782  \\
            2.638  1.5345960837203392  \\
            3.461  1.5210821701555493  \\
            0.906  1.6319295963860418  \\
            0.153  1.6979133125898027  \\
            3.157  1.5217846335075587  \\
            1.231  1.5828132228544107  \\
            1.586  1.5620178818149648  \\
            2.032  1.547923020142095  \\
            0.884  1.6372435907317688  \\
            2.987  1.525367735702009  \\
            1.782  1.554848527336825  \\
            0.754  1.6897902127900888  \\
            3.455  1.5210787576940934  \\
            1.452  1.5684026397551065  \\
            2.557  1.536311637306789  \\
            0.59  1.6972475314506212  \\
            1.785  1.5548406527346734  \\
            1.893  1.551455408328703  \\
            3.103  1.5224303608535046  \\
            3.281  1.5212986052591972  \\
            2.584  1.5357406197330667  \\
            3.391  1.5211383051069032  \\
            3.348  1.5211859409379278  \\
            2.157  1.5449862940700456  \\
            1.862  1.5525105232694765  \\
            0.6  1.696930976630043  \\
            0.822  1.6589703668460427  \\
            0.654  1.6964733915471781  \\
            1.532  1.564261477635435  \\
            2.204  1.5439151952324817  \\
            1.42  1.569970150883445  \\
            2.295  1.54187027143998  \\
            2.528  1.5369968209780378  \\
            1.947  1.5501509216789338  \\
            1.801  1.554261262231112  \\
            1.245  1.5818124641203717  \\
            1.345  1.5745184223219788  \\
            0.722  1.6943215538837775  \\
            2.776  1.5313560981256338  \\
            2.669  1.5338577538442142  \\
            3.407  1.5211131030151364  \\
            0.607  1.6968581004605312  \\
            0.238  1.6978602755367402  \\
            1.39  1.5716238464700896  \\
            1.526  1.5644562709084582  \\
            0.279  1.6977877070049556  \\
            1.69  1.5580115407561659  \\
            0.008  1.697936281167871  \\
            2.573  1.535925020112858  \\
            2.03  1.5479936365144562  \\
            0.944  1.622164303545178  \\
            0.318  1.6978496755825292  \\
            0.736  1.6935780621414116  \\
            3.223  1.5214884602827454  \\
            0.374  1.697797765415448  \\
            1.1  1.5964746655910844  \\
            2.912  1.5279650877980466  \\
            0.524  1.6975129973806342  \\
            1.219  1.5840696292840595  \\
            2.697  1.5331880747225013  \\
            0.0  1.6978789832609218  \\
            3.178  1.5216503652236075  \\
            0.479  1.6975260147042222  \\
            0.167  1.6978660391190403  \\
            2.841  1.5297798179198048  \\
            1.352  1.5741124583624797  \\
            0.476  1.6976009343521388  \\
            0.69  1.6952542262937886  \\
            1.267  1.579947130720147  \\
            1.356  1.5735359730777823  \\
            1.198  1.585657153342031  \\
            1.205  1.5851600229677119  \\
            0.096  1.6978825676945317  \\
            0.226  1.6978072159276412  \\
            2.716  1.532806380321249  \\
            2.832  1.5300580083292088  \\
            1.958  1.5498192393380459  \\
            1.403  1.5709071033346944  \\
            2.879  1.5287529531153328  \\
            1.391  1.5715320427627486  \\
            2.674  1.5337113373379476  \\
            1.196  1.5862109037849503  \\
            0.65  1.6965614235280744  \\
            2.773  1.5314817904026068  \\
            2.305  1.5416634163595286  \\
            1.525  1.5644887697749792  \\
            0.975  1.616551606317045  \\
            0.686  1.6964500615802824  \\
            0.133  1.697927813945437  \\
            2.813  1.5304571972874705  \\
            0.54  1.697377962700065  \\
            0.563  1.6973617609730827  \\
            0.045  1.6978970049970634  \\
            1.82  1.5537390051764173  \\
            2.072  1.5469836029455155  \\
            3.007  1.524879611631271  \\
            1.203  1.5853238890793744  \\
            1.475  1.567067340281924  \\
            1.422  1.5699403951059254  \\
            1.14  1.5917473976197283  \\
            1.158  1.589639784226421  \\
            0.912  1.629859172079278  \\
            1.207  1.5849959903616802  \\
            1.371  1.5728447691734748  \\
            0.963  1.6175297206202925  \\
            0.987  1.613558201209168  \\
            1.572  1.5625504132263073  \\
            3.099  1.52248899321909  \\
            0.511  1.6974417367603405  \\
            0.541  1.697331274219359  \\
            1.739  1.5562774364346155  \\
            3.129  1.522051062856777  \\
            0.623  1.6968823037570018  \\
            0.018  1.6978627055446867  \\
            0.627  1.6969395749533867  \\
            3.004  1.5249824764922204  \\
            2.73  1.5324680222990439  \\
            2.902  1.528197815270825  \\
            1.851  1.5527914002633991  \\
            3.29  1.521287452738366  \\
            1.718  1.5569985687935104  \\
            1.578  1.562378854499872  \\
            2.653  1.534251691916977  \\
            0.647  1.6964037061562653  \\
            1.046  1.603597455854875  \\
            0.835  1.6547614456132718  \\
            0.234  1.697844021657933  \\
            1.652  1.559313479517644  \\
            0.578  1.6972767415314225  \\
            1.692  1.5579249212721258  \\
            2.505  1.5374243043868592  \\
            2.83  1.5300106339768114  \\
            2.901  1.528114016518059  \\
            0.789  1.6740018578482705  \\
            2.077  1.5468132103056247  \\
            2.136  1.5454981253908668  \\
            1.178  1.587838571116594  \\
            0.455  1.697661761581423  \\
            1.908  1.5510784915692395  \\
            1.637  1.5600160021287377  \\
            3.358  1.52118026061506  \\
            2.898  1.5282730679530538  \\
            1.933  1.55048094933619  \\
            3.016  1.5246350872285832  \\
            3.416  1.5211149459378168  \\
            2.065  1.5471477505496467  \\
            0.962  1.6186313922457614  \\
            3.019  1.5245323934115098  \\
            0.192  1.69786628711697  \\
            0.951  1.6210476709370483  \\
            2.624  1.5348844798862826  \\
            1.743  1.55607036833555  \\
            0.083  1.6978736531557153  \\
            2.232  1.5432409999496084  \\
            0.323  1.6978392671824796  \\
            0.093  1.6978828711468161  \\
            1.281  1.5787396338721487  \\
            1.852  1.5527488734768706  \\
            3.342  1.521190173213279  \\
            1.577  1.5623713328459348  \\
            2.874  1.528964909256404  \\
            1.226  1.583349733385273  \\
            3.373  1.5211537140850098  \\
            2.4  1.539672187902601  \\
            1.293  1.5780043243558113  \\
            1.712  1.5573287731889678  \\
            3.377  1.5211521967789559  \\
            2.202  1.543881368402351  \\
            2.39  1.5398935080014022  \\
            1.951  1.549992493345083  \\
            2.239  1.5430945278111197  \\
            1.845  1.552985851522121  \\
            2.178  1.544539095617074  \\
            0.072  1.6979290224155879  \\
            2.012  1.5484589161699682  \\
            1.709  1.5572837969173356  \\
            0.027  1.6979194109294424  \\
            1.437  1.5688444411170266  \\
            2.758  1.5317837617033527  \\
            1.495  1.565856439655985  \\
            0.51  1.6975891437749817  \\
            0.253  1.697879164498016  \\
            2.481  1.5378669601401194  \\
            1.357  1.5737057638350869  \\
            2.855  1.5293974713023828  \\
            3.473  1.5210640292587143  \\
            2.676  1.533699150249686  \\
            1.038  1.6050453576070627  \\
            1.704  1.557468689976773  \\
            0.086  1.6978860679483616  \\
            1.59  1.5618376584006533  \\
            2.273  1.5423761937734533  \\
            2.805  1.5306760235725319  \\
            0.026  1.6978918931271945  \\
            2.058  1.547256621672187  \\
            0.245  1.6978313018169038  \\
            1.351  1.5739377430578623  \\
            0.797  1.6731235057693803  \\
            1.265  1.5803644160810946  \\
            0.333  1.6977899217985235  \\
            3.125  1.5221486197418665  \\
            2.52  1.5371138428001152  \\
            0.879  1.6393156967721647  \\
            0.132  1.697900197355562  \\
            1.517  1.5649501242113115  \\
            2.793  1.5309149742031019  \\
            2.637  1.5345603796649427  \\
            3.124  1.5220569290791008  \\
            2.092  1.5465446315178721  \\
            2.824  1.5301666042265676  \\
            2.526  1.5370145105400144  \\
            2.673  1.5337569535383884  \\
            1.063  1.6015275974224723  \\
            1.31  1.5768300150370085  \\
            2.973  1.526039768250738  \\
            3.117  1.52213225713206  \\
            0.491  1.6976016460861647  \\
            0.9  1.6336101034735457  \\
            1.034  1.6066138399698429  \\
            2.307  1.5416086375907956  \\
            0.913  1.6312346528827122  \\
            1.608  1.5610197188932067  \\
            1.606  1.5611061424505015  \\
            0.016  1.6978421604705463  \\
            1.23  1.582868460114648  \\
            0.527  1.6974584991970807  \\
            0.832  1.6554948858354959  \\
            1.835  1.553200508149677  \\
            0.609  1.6971231780060085  \\
            3.399  1.5211248586156336  \\
            1.163  1.5892868801648856  \\
            1.264  1.5803301345379661  \\
            1.92  1.5508363861332188  \\
            1.542  1.5638887239041666  \\
            1.84  1.553187198112529  \\
            2.648  1.5343401268598849  \\
            1.611  1.5609136953410299  \\
            0.557  1.697354795301739  \\
            0.587  1.6972479631679656  \\
            3.024  1.5244289873971548  \\
            2.391  1.5398635851415183  \\
            0.113  1.697850692427567  \\
            0.18  1.6979295552798643  \\
            2.073  1.546870641041576  \\
            1.914  1.5509742729500327  \\
            1.432  1.5691499042272317  \\
            3.352  1.5211839560743796  \\
            1.601  1.561295919774175  \\
            1.04  1.6049720034669441  \\
            1.068  1.6006872162842074  \\
            2.364  1.540421536062788  \\
            3.382  1.5211555591771653  \\
            1.431  1.5693404198799346  \\
            1.767  1.5553735093603822  \\
            2.271  1.5424191148317907  \\
            0.135  1.697892856088248  \\
            2.245  1.5429805477587608  \\
            0.442  1.697678946997701  \\
            2.76  1.5317657147912984  \\
            0.957  1.619478430775275  \\
            2.517  1.5371443626157548  \\
            2.66  1.5340483077039344  \\
            1.134  1.5924179499380011  \\
            2.617  1.5350145031942408  \\
            2.521  1.53713881088605  \\
            1.638  1.5598816754070006  \\
            2.049  1.547495945331567  \\
            3.424  1.521089540353245  \\
            1.808  1.5541667735217366  \\
            0.586  1.6970709719102188  \\
            0.629  1.6970330143243302  \\
            0.209  1.6978493367345038  \\
            2.113  1.5459934866514682  \\
            0.235  1.6978776656540011  \\
            1.57  1.5627128132633819  \\
            2.38  1.5401145025411598  \\
            3.262  1.521335980002039  \\
            2.781  1.531237703882935  \\
            1.924  1.5507843167516506  \\
            0.137  1.6978919267778498  \\
            1.055  1.6026877984148082  \\
            2.866  1.5291527059042802  \\
            2.028  1.547955594233606  \\
            2.091  1.546515871489164  \\
            2.844  1.5296781319878092  \\
            0.382  1.6977349280879526  \\
            0.544  1.6973163258144617  \\
            2.336  1.5410025943755508  \\
            0.961  1.6189301103519498  \\
            2.265  1.5425326279486318  \\
            3.467  1.5210731486529316  \\
            1.96  1.5497837306945488  \\
            1.636  1.5600290255332347  \\
            1.726  1.5567003493778113  \\
            3.468  1.5210690614984683  \\
            0.223  1.6977787252092071  \\
            1.788  1.554644651173939  \\
            2.311  1.5415186020984128  \\
            1.103  1.595848336683  \\
            1.407  1.5704952894015474  \\
            1.079  1.5994175748885822  \\
            0.68  1.6959808893829036  \\
            2.542  1.5366245452046499  \\
            0.471  1.6975515169511022  \\
            3.013  1.5245628423909978  \\
            0.115  1.6978850241886603  \\
            2.292  1.5419972995568547  \\
            1.745  1.556083477799065  \\
            2.598  1.5354173078929285  \\
            2.203  1.5439508894046494  \\
            1.896  1.551555166971646  \\
            0.996  1.6120171454209873  \\
            2.795  1.5309164540232276  \\
            0.356  1.6977306744518312  \\
            2.754  1.5319111262951033  \\
            3.222  1.5214614753963684  \\
            3.438  1.5210958666485657  \\
            1.307  1.5770976860060282  \\
            1.251  1.5811700050352764  \\
            3.077  1.5228174823363871  \\
            1.443  1.568756698885767  \\
            0.06  1.6978563625483274  \\
            2.92  1.527608559696886  \\
            0.528  1.6974913397133284  \\
            3.055  1.5234448833349523  \\
            0.336  1.6977863137066067  \\
            0.774  1.684486099210111  \\
            0.538  1.6974904856027453  \\
            3.437  1.5210956880895181  \\
            1.018  1.6086303531957757  \\
            2.254  1.5427207311774636  \\
            0.033  1.6978914773233824  \\
            1.522  1.5647760781396116  \\
            2.345  1.5408492076262694  \\
            2.301  1.5418213334745006  \\
            1.263  1.5801277474430813  \\
            3.219  1.5214993717513206  \\
            2.588  1.5356579874535567  \\
            0.766  1.685270121975039  \\
            0.343  1.6976998186576944  \\
            0.42  1.6977316779842184  \\
            1.301  1.5772524683849793  \\
            2.699  1.5331563155560382  \\
            2.726  1.5325699430852624  \\
            2.77  1.5315440073132067  \\
            0.32  1.697788009341635  \\
            1.644  1.5596394948254921  \\
            3.498  1.5210475155598973  \\
            2.442  1.5387235998989295  \\
            1.883  1.5518774164531997  \\
            1.056  1.6022208214183367  \\
            0.243  1.6978445386945429  \\
            3.156  1.521820096723904  \\
            2.488  1.5377936396949972  \\
            0.655  1.6966436279070796  \\
            3.154  1.5218379221371476  \\
            0.262  1.6977773137856573  \\
            1.16  1.5894744188636953  \\
            3.146  1.5218958904724689  \\
            1.021  1.6081081953065424  \\
            1.096  1.5971834624005836  \\
            1.38  1.5723012150062254  \\
            2.84  1.529749728222015  \\
            1.296  1.5779992961566172  \\
            0.303  1.697770503150785  \\
            2.32  1.541392170129121  \\
            0.795  1.6713859325218194  \\
            0.316  1.6977996103115727  \\
            1.287  1.5786653474256314  \\
            2.583  1.5357832374620808  \\
            1.752  1.5558069259139868  \\
            3.332  1.521208203303054  \\
            1.268  1.5800330191500527  \\
            1.059  1.6019404592610769  \\
            0.849  1.6484169664546071  \\
            1.19  1.586810975232027  \\
            0.875  1.6404876573193086  \\
            3.277  1.5213000517224131  \\
            3.093  1.5226085175852473  \\
            2.867  1.5290374534598985  \\
            1.35  1.5740955043031377  \\
            3.087  1.522733541265843  \\
            0.658  1.6965124651416736  \\
            0.449  1.69765723077344  \\
            3.209  1.5215478910805527  \\
            2.445  1.5385333248109876  \\
            2.448  1.5386377821913126  \\
            2.654  1.5342177573001174  \\
            3.253  1.5213907265395086  \\
            1.425  1.5694039642605844  \\
            1.581  1.5620086457574833  \\
            1.208  1.5850212464558993  \\
            3.408  1.5211172935461181  \\
            0.758  1.6898062658604565  \\
            2.923  1.5275429990108584  \\
            1.436  1.56905730492338  \\
            2.713  1.5328708064900272  \\
            0.545  1.6974862892554194  \\
            3.376  1.521147880712946  \\
            0.5  1.697529925771125  \\
            0.704  1.6948417694662823  \\
            0.741  1.6936624693070668  \\
            1.346  1.5743486158549462  \\
            1.621  1.5604994560444985  \\
            0.355  1.6977399965983622  \\
            3.033  1.5240985543582242  \\
            0.929  1.6258844494066558  \\
            1.863  1.552439271272626  \\
            1.66  1.559025756464378  \\
            3.24  1.5214237512752131  \\
            2.224  1.5434780308376896  \\
            2.038  1.5478170769029476  \\
            1.341  1.5745767742735999  \\
            0.494  1.6976301417554909  \\
            0.426  1.6977152903542148  \\
            1.964  1.5496690813347795  \\
            3.354  1.5211721581191613  \\
            1.681  1.5582585285128046  \\
            0.328  1.6977838537431593  \\
            0.707  1.6954460711363948  \\
            0.306  1.697766740734485  \\
            2.554  1.5363833322558693  \\
            1.305  1.5769624760189902  \\
            0.993  1.6127768207894893  \\
            2.285  1.5421466455174941  \\
            0.273  1.6977844893722795  \\
            1.885  1.5518834371064085  \\
            2.404  1.5395848711780524  \\
            0.49  1.6975905762575898  \\
            1.498  1.5658882394111235  \\
            2.951  1.5267968460438148  \\
            2.929  1.527405789379028  \\
            1.148  1.5907727174793576  \\
            0.302  1.697802400196907  \\
            1.098  1.596645697285541  \\
            0.999  1.6117782560460916  \\
            0.821  1.6585863634839744  \\
            1.439  1.5688513191570246  \\
            3.152  1.5218777700577912  \\
            2.2  1.5439936164516033  \\
            3.089  1.522578407891061  \\
            1.209  1.5848460510809628  \\
            0.573  1.6972329233702923  \\
            0.042  1.6978960016281304  \\
            0.294  1.6977857509561547  \\
            1.049  1.603608171737133  \\
            2.627  1.5347691906718879  \\
            2.211  1.5437249712923966  \\
            2.083  1.5466245078571843  \\
            0.895  1.6348657448706683  \\
            2.693  1.5333698414170953  \\
            1.575  1.5624505232452692  \\
            1.836  1.5532869901474913  \\
            3.224  1.5214750587934105  \\
            1.218  1.5842638415284986  \\
            1.048  1.6034064157081858  \\
            1.747  1.5561418579256086  \\
            0.928  1.625756098597262  \\
            0.021  1.6979034310152599  \\
            2.655  1.5342086320826223  \\
            2.71  1.5329617229917372  \\
            1.277  1.5793740589983059  \\
            0.665  1.6962229842922114  \\
            1.992  1.5488919740564877  \\
            1.259  1.58075820461502  \\
            2.812  1.5305153788373134  \\
            3.433  1.5210922389931338  \\
            3.145  1.5218874153769029  \\
            2.934  1.5271832966922374  \\
            3.202  1.5215292113997736  \\
            0.619  1.6972778244150175  \\
            1.118  1.5943761051256524  \\
            1.765  1.5555254059354249  \\
            0.633  1.6969641812012477  \\
            3.499  1.5210418318609933  \\
            3.111  1.5222287424226275  \\
            0.361  1.6976559966155693  \\
            0.663  1.6964493046481304  \\
            2.954  1.526659813012985  \\
            2.538  1.5367292983204828  \\
            0.38  1.6977067966437591  \\
            2.666  1.533997964007591  \\
            0.011  1.69787352512384  \\
            3.475  1.521060846191934  \\
            2.082  1.5466831929231657  \\
            1.528  1.5645127879225986  \\
            3.234  1.5214315325268428  \\
            2.82  1.5302297074074533  \\
            3.394  1.5211399850082985  \\
            1.564  1.5628719518395753  \\
            2.33  1.541119947685273  \\
            0.783  1.6757293271525027  \\
            1.493  1.5660018160042828  \\
            1.641  1.559813603197768  \\
            3.183  1.5216289122066615  \\
            0.338  1.6977836415384373  \\
            2.558  1.53627650132194  \\
            0.425  1.6975695317730286  \\
            1.496  1.5659578231071085  \\
            3.481  1.5210558786366837  \\
            0.325  1.6977653351860953  \\
            3.383  1.521159306581177  \\
            2.447  1.5386720130894296  \\
            0.474  1.6975735722062053  \\
            0.105  1.697895942877897  \\
            3.197  1.5215593144484947  \\
            1.451  1.5683352738552343  \\
            0.383  1.6977523038411044  \\
            1.735  1.5564296056887243  \\
            0.668  1.6959783551403131  \\
            1.372  1.5728292369926145  \\
            3.191  1.5215984100011173  \\
            0.475  1.6976090412374911  \\
            1.154  1.5904061835854153  \\
            2.338  1.5409513251952653  \\
            2.585  1.5357436959929738  \\
            1.784  1.5549066749049063  \\
            2.026  1.5480933955289877  \\
            1.76  1.5555386135613634  \\
            2.817  1.5303708792136674  \\
            1.781  1.554953464860642  \\
            2.995  1.5253869819973476  \\
            2.413  1.5394176855768575  \\
            2.003  1.5486618657312081  \\
            2.771  1.5314663033441145  \\
            1.892  1.5516002819140937  \\
            1.007  1.6099561085146055  \\
            1.453  1.5680774423169301  \\
            0.186  1.6978143179899206  \\
            0.866  1.6435394225718216  \\
            3.259  1.5213687028231577  \\
            1.79  1.5547282315755835  \\
            3.364  1.5211640621158584  \\
            0.525  1.6975144231456325  \\
            2.361  1.5405319568859952  \\
            1.913  1.5509932637561397  \\
            1.448  1.5684237504004386  \\
            2.182  1.5444723149415784  \\
            2.843  1.5296159495208348  \\
            2.01  1.548540397147247  \\
            0.631  1.6970034513582457  \\
            3.302  1.5212607265786642  \\
            0.915  1.6306270539696248  \\
            1.338  1.574910473283095  \\
            1.172  1.5883121147474675  \\
            0.13  1.6978723239321667  \\
            1.811  1.5539507475443715  \\
            2.29  1.5419786593833424  \\
            2.319  1.5413879025376935  \\
            2.63  1.534721313492538  \\
            2.019  1.5483147521507326  \\
            1.261  1.580395706456224  \\
            2.451  1.538577797024507  \\
            2.661  1.5340941708365372  \\
            0.456  1.6976413926554836  \\
            2.44  1.5388135852919638  \\
            2.288  1.5420325985940573  \\
            1.37  1.5728837880931155  \\
            2.23  1.5433473327449099  \\
            2.587  1.5356342485155903  \\
            1.182  1.5871088283259613  \\
            3.497  1.521042449285344  \\
            2.278  1.5422744546354719  \\
            3.322  1.5212194512811934  \\
            1.354  1.573866188688955  \\
            1.472  1.567049348802036  \\
            2.373  1.5402438674601713  \\
            3.485  1.5210560520391712  \\
            0.793  1.673390744941141  \\
            0.324  1.697767787159968  \\
            3.127  1.5220374334090165  \\
            2.679  1.5336881895480088  \\
            3.278  1.521322929642833  \\
            1.749  1.5559574760653676  \\
            2.706  1.5330877479453184  \\
            1.143  1.5914370124707666  \\
            2.406  1.5395384992842795  \\
            2.504  1.5374608713898006  \\
            1.225  1.5835708064667704  \\
            1.489  1.5663259295272438  \\
            0.989  1.6134208237685488  \\
            2.097  1.5463619336779246  \\
            2.219  1.5435590870293385  \\
            1.47  1.5671724599009462  \\
            1.9  1.551440271878563  \\
            1.516  1.5650509061339584  \\
            1.101  1.5965103697105358  \\
            2.539  1.5366620292193256  \\
            2.964  1.5262684691074035  \\
            2.636  1.5345840218248314  \\
            1.849  1.5528630639705083  \\
            1.691  1.5580375080174254  \\
            2.399  1.5396649921824455  \\
            2.452  1.5385519649694446  \\
            1.027  1.6061322890126022  \\
            0.447  1.697633558340877  \\
            3.483  1.521042881918627  \\
            0.841  1.6516103320361346  \\
            0.549  1.6970479051461775  \\
            2.949  1.5267380399421404  \\
            2.641  1.5345271119664972  \\
            0.463  1.6975871456869618  \\
            1.696  1.557697084099489  \\
            2.537  1.5367536960212063  \\
            2.185  1.544335762461624  \\
            0.87  1.642555794933655  \\
            2.746  1.5320966736043629  \\
            1.812  1.553981025277162  \\
            0.529  1.697462559277871  \\
            2.048  1.5475477350513105  \\
            2.61  1.5351730183759527  \\
            0.1  1.6978495500279458  \\
            1.942  1.5502585919179053  \\
            1.519  1.5647714814781817  \\
            3.194  1.5215726784193324  \\
            1.484  1.566600457508265  \\
            3.106  1.5223909610630242  \\
            1.355  1.5738062690657335  \\
            3.446  1.5210812908248292  \\
            3.346  1.521202173066327  \\
            0.022  1.6979182444240917  \\
            2.496  1.5375925380610214  \\
            3.264  1.5213327480144772  \\
            0.537  1.6975039665547642  \\
            2.347  1.540727924363488  \\
            1.539  1.5639912333363668  \\
            1.55  1.5634422452903503  \\
            1.816  1.5537979042749535  \\
            0.591  1.6972048383440896  \\
            2.547  1.5365755930550404  \\
            2.698  1.5332504609421787  \\
            3.379  1.5211576628372712  \\
            1.479  1.566820601223006  \\
            0.804  1.6687222302957845  \\
            0.37  1.6977947234291115  \\
            1.333  1.5750992263849304  \\
            1.215  1.5842085298689492  \\
            0.403  1.6976994308846023  \\
            2.614  1.5350822157072082  \\
            0.071  1.6979180890184467  \\
            2.401  1.5396323590299879  \\
            0.346  1.6977832547341598  \\
            0.378  1.697661196774582  \\
            2.352  1.5406726452347546  \\
            0.939  1.6239904801974585  \\
            0.251  1.6978146855318728  \\
            0.981  1.6149386890779411  \\
            3.021  1.524430764999341  \\
            0.109  1.6978805402966533  \\
            3.482  1.5210527545171  \\
            1.503  1.5657268991329487  \\
            0.213  1.6977857552982494  \\
            0.882  1.6383202779781438  \\
            1.533  1.564285672854251  \\
            1.353  1.5740355686742529  \\
            2.8  1.5308454611696052  \\
            2.141  1.545294224443301  \\
            3.217  1.5214933382737548  \\
            2.986  1.5255552832543184  \\
            0.281  1.6978372929082726  \\
            2.981  1.5257341522979655  \\
            2.877  1.5288603915361016  \\
            0.208  1.6978555321411994  \\
            1.819  1.5537231206392588  \\
            0.387  1.6977082707874172  \\
            0.218  1.69786177606034  \\
            1.569  1.5627528690033035  \\
            3.306  1.521271998084008  \\
            0.669  1.6966019142353819  \\
            1.45  1.5683304615182172  \\
            1.909  1.5510337169440618  \\
            2.013  1.5483914494717492  \\
            2.121  1.5458245535712964  \\
            3.329  1.5211997167870512  \\
            0.864  1.6440167117696534  \\
            0.531  1.6975706114783262  \\
            2.85  1.5295504637346071  \\
            0.551  1.6973527443001297  \\
            2.393  1.5398354700879722  \\
            0.317  1.6977322949181866  \\
            1.03  1.605772174211554  \\
            0.808  1.6647210534382373  \\
            0.984  1.6143668855739415  \\
            3.228  1.5214644794775434  \\
            1.337  1.5749581394386394  \\
            0.497  1.6975452272112166  \\
            2.341  1.5408792933937099  \\
            0.964  1.6181052540944767  \\
            1.742  1.5562517350089908  \\
            3.445  1.521087228544152  \\
            0.054  1.6978902572830432  \\
            3.355  1.5211745977909323  \\
            2.65  1.5342861981546094  \\
            1.125  1.5936226970526362  \\
            3.006  1.5250983602362453  \\
            0.051  1.6978177472428249  \\
            2.477  1.538043306857836  \\
            3.357  1.5211710978109965  \\
            2.144  1.5452770897589654  \\
            0.909  1.6313673574603866  \\
            2.107  1.5460986666833978  \\
            1.615  1.5608536759097302  \\
            2.895  1.528329011682251  \\
            2.665  1.5339120271055937  \\
            0.187  1.6978696983826598  \\
            0.375  1.6976897357869671  \\
            3.051  1.5235169203852845  \\
            0.339  1.697786785166124  \\
            3.037  1.524027904850169  \\
            2.169  1.5447075394350738  \\
            2.652  1.5342663269128738  \\
            1.313  1.576595702291328  \\
            1.374  1.5725564294361662  \\
            3.324  1.5212313659322516  \\
            0.601  1.697228320522093  \\
            0.29  1.697805385253557  \\
            2.17  1.5446635475640726  \\
            0.122  1.6979118759336869  \\
            0.919  1.6274840983796681  \\
            1.32  1.5759722699902603  \\
            0.322  1.6977437825387096  \\
            0.724  1.6927971272317048  \\
            1.408  1.570594351883597  \\
            1.508  1.5653621696919666  \\
            0.966  1.6179071054151184  \\
            2.988  1.5256728593180224  \\
            2.453  1.53851042927639  \\
            3.028  1.5242873986801624  \\
            1.731  1.5566769361028292  \\
            1.779  1.555016576361713  \\
            2.647  1.5343349021100314  \\
            1.963  1.5496268897631964  \\
            2.223  1.543583394355464  \\
            2.822  1.5302799723614087  \\
            2.971  1.5260693812921164  \\
            2.106  1.5461370700940202  \\
            0.842  1.6512485478051602  \\
            0.34  1.6977954985804007  \\
            3.072  1.5230303969428234  \\
            2.181  1.5444118601975294  \\
            1.796  1.5544103971446277  \\
            2.656  1.5341433939702278  \\
            0.366  1.6977488010578656  \\
            1.721  1.5569008958272157  \\
            1.3  1.5773617763335979  \\
            2.534  1.5368633844957174  \\
            0.428  1.6977784121897739  \\
            0.711  1.6951431023370238  \\
            1.624  1.5603653083600564  \\
            3.071  1.5229710776482093  \\
            2.888  1.5286164813550267  \\
            0.429  1.6977907464646793  \\
            1.878  1.5520521627399493  \\
            3.457  1.5210714987558605  \\
            0.812  1.6624284135940388  \\
            0.138  1.6978074831051768  \\
            1.112  1.5948293037884813  \\
            0.423  1.6977126971023686  \\
            2.768  1.531566082305687  \\
            1.627  1.560374788295739  \\
            2.597  1.5354745427604113  \\
            1.024  1.6075514065793415  \\
            1.086  1.598665621329614  \\
            0.28  1.6978826732941743  \\
            0.992  1.612821749406453  \\
            2.314  1.5414965994431422  \\
            1.499  1.565853042744105  \\
            0.829  1.6549827355706337  \\
            2.596  1.5354893374124763  \\
            1.08  1.5995342398979826  \\
            3.268  1.5213449873429041  \\
            0.843  1.647886769569797  \\
            0.771  1.6833243561510063  \\
            0.922  1.62794668207702  \\
            3.086  1.5226864540438327  \\
            0.35  1.697775340803512  \\
            2.465  1.5382417301527287  \\
            2.096  1.5464146178284326  \\
            2.519  1.537198630400186  \\
            0.827  1.6568683595906213  \\
            1.729  1.5565947277592618  \\
            2.577  1.5358871399779719  \\
            0.292  1.6978232378929161  \\
            1.153  1.5904064990263664  \\
            3.404  1.521129416302084  \\
            3.187  1.5215762536208688  \\
            0.509  1.6975004003090115  \\
            0.723  1.6945049051323433  \\
            2.536  1.5367437224497307  \\
            0.023  1.697910506287102  \\
            0.735  1.6935315658068526  \\
            1.961  1.5497346549686029  \\
            1.675  1.5585931626698182  \\
            2.501  1.5375385831090533  \\
            1.693  1.557860530426081  \\
            2.177  1.544487324226997  \\
            3.172  1.5217050332448656  \\
            0.439  1.6977548872019839  \\
            1.405  1.5705828992988342  \\
            0.643  1.6966187934923496  \\
            1.659  1.5592174086309851  \\
            1.962  1.5497335738778601  \\
            2.14  1.5454394797994895  \\
            0.624  1.6969886505125413  \\
            3.385  1.5211391891436978  \\
            3.001  1.5251178591069934  \\
            2.686  1.533509265631204  \\
            0.498  1.6975500895849696  \\
            1.912  1.5511507559587172  \\
            0.77  1.6846229026633688  \\
            1.93  1.5506039029166792  \\
            0.569  1.697332495722341  \\
            1.309  1.5766071329774045  \\
            0.284  1.6978202625353904  \\
            1.169  1.5890178179636076  \\
            0.874  1.6414825840888267  \\
            2.927  1.5275106291949228  \\
            1.347  1.5742956382988074  \\
            1.854  1.552674082306386  \\
            3.493  1.5210502613226222  \\
            0.585  1.6972906324060877  \\
            3.108  1.522310836467922  \\
            2.264  1.5426412738516198  \\
            3.1  1.5224114734742802  \\
            1.763  1.555520675768397  \\
            2.046  1.5475849088518108  \\
            0.819  1.659468389756212  \\
            2.975  1.526009320540069  \\
            2.827  1.5301373390164899  \\
            1.902  1.5513366545005656  \\
            0.484  1.697573961652655  \\
            0.977  1.6167144757616418  \\
            2.321  1.5413798394401415  \\
            1.768  1.55526355288844  \\
            0.853  1.6492077655493127  \\
            0.483  1.6975542794656462  \\
            2.044  1.547553772858525  \\
            3.435  1.521089259609616  \\
            1.084  1.5987336247431991  \\
            0.969  1.6173483107223516  \\
            2.95  1.5267022420030327  \\
            0.275  1.6978252444267457  \\
            1.473  1.5672091762217422  \\
            3.272  1.521334562172102  \\
            0.788  1.6787725648803604  \\
            2.535  1.5367942230528084  \\
            2.687  1.5334780239282315  \\
            0.009  1.6978942990873007  \\
            0.677  1.6964112313723796  \\
            1.121  1.5939177645587634  \\
            2.299  1.541833354164141  \\
            2.962  1.5264648873648432  \\
            0.358  1.6977596362010945  \\
            0.839  1.6529567191169843  \\
            2.431  1.5389822312473136  \\
            2.969  1.5262176659229032  \\
            1.596  1.5614163961650844  \\
            2.945  1.5268972159415595  \\
            2.179  1.5444904041283787  \\
            1.411  1.570535990969999  \\
            2.02  1.548255499968489  \\
            3.073  1.5230637884984708  \\
            2.163  1.5448299436716957  \\
            3.135  1.5219724679993323  \\
            1.117  1.5944724194948305  \\
            3.378  1.521144286353083  \\
            2.162  1.5448572619549907  \\
            0.166  1.6978528459680446  \\
            3.422  1.5211124593709155  \\
            1.535  1.5642733657429124  \\
            0.261  1.6978087288574846  \\
            3.113  1.5222833421031419  \\
            0.953  1.6205554961478423  \\
            0.329  1.6977700859735045  \\
            3.184  1.521633510588401  \\
            3.448  1.521084098137599  \\
            2.823  1.5302761999062553  \\
            2.284  1.5421610730238764  \\
            3.361  1.5211745929115588  \\
            3.374  1.5211549676637717  \\
            2.848  1.5295944934594161  \\
            0.923  1.6280869025807692  \\
            2.586  1.5357358465072606  \\
            0.102  1.6979111053182394  \\
            0.334  1.6977341262298107  \\
            0.301  1.6978363390025206  \\
            0.337  1.6977516133837047  \\
            0.894  1.6354547781108315  \\
            3.375  1.5211580621517324  \\
            0.468  1.6976943261882704  \\
            3.043  1.523796351845156  \\
            1.167  1.5887652908975558  \\
            2.599  1.535431593566581  \\
            0.236  1.6978531891055102  \\
            2.607  1.5352074383687773  \\
            3.049  1.5236061888593282  \\
            0.579  1.6971422629599306  \\
            0.131  1.6978834795170745  \\
            3.204  1.521562849299331  \\
            2.814  1.5304746869855785  \\
            0.645  1.6970257268986508  \\
            0.188  1.6978891983623585  \\
            2.876  1.5288438366955737  \\
            2.845  1.5296274410562234  \\
            0.502  1.6974578464617271  \\
            0.397  1.6976939652381189  \\
            1.483  1.5666523135114008  \\
            1.336  1.5750176554354085  \\
            3.177  1.5216707273377705  \\
            2.272  1.5424093303732613  \\
            3.153  1.521826936664633  \\
            1.673  1.5587640230268691  \\
            0.119  1.697899882517636  \\
            2.475  1.537992109585684  \\
            2.335  1.541016850190762  \\
            2.618  1.5350418425168468  \\
            0.184  1.6978001576128108  \\
            3.271  1.5213064770929159  \\
            2.476  1.5380613098188225  \\
            2.047  1.5475848477139695  \\
            2.701  1.5332098317198635  \\
            0.408  1.6976927983897385  \\
            0.79  1.6766542313618837  \\
            3.2  1.5215423384411815  \\
            0.143  1.697892478224206  \\
            1.834  1.5533217933013344  \\
            1.833  1.5533220122061435  \\
            0.593  1.6973390733523037  \\
            0.01  1.697856709372078  \\
            2.98  1.5257937049846413  \\
            0.535  1.6974427135466597  \\
            3.257  1.5213542042305965  \\
            1.197  1.5861774866613905  \\
            3.464  1.5210755129654066  \\
            2.052  1.5474726180008402  \\
            1.626  1.5602620038169253  \\
            0.398  1.697608301704476  \\
            2.59  1.5355937469333174  \\
            0.908  1.6314463406912396  \\
            0.351  1.6977722849134698  \\
            3.46  1.521076444215459  \\
            0.972  1.6162752679871355  \\
            0.237  1.6978887851509026  \\
            1.891  1.5516660535181344  \\
            3.384  1.5211445704591646  \\
            0.452  1.69767856817184  \\
            2.685  1.5334944648265618  \\
            2.549  1.5365065048550286  \\
            1.684  1.5581252402653736  \\
            2.69  1.533379267804831  \\
            3.147  1.521873049294843  \\
            2.897  1.5283090918714695  \\
            2.425  1.539154141782081  \\
            1.49  1.566233604228744  \\
            1.563  1.5629799328494858  \\
            2.645  1.5344156174178065  \\
            0.547  1.6974084434936185  \\
            0.372  1.6977378988097684  \\
            0.725  1.6935918052521315  \\
            0.25  1.6978569959088279  \\
            1.814  1.5538886357100106  \\
            2.443  1.538806197378806  \\
            1.619  1.5605179496387132  \\
            1.632  1.56018690414597  \\
            3.148  1.5218876313683287  \\
            1.485  1.5665975408684516  \\
            1.717  1.557000675869102  \\
            0.389  1.6977717876361336  \\
            0.956  1.619469000947373  \\
            0.373  1.697785540490725  \\
            0.02  1.6979183891099943  \\
            1.697  1.5578547403105916  \\
            0.73  1.6936227615603667  \\
            0.017  1.6978725464373665  \\
            1.546  1.5637453660701264  \\
            1.057  1.60156881180402  \\
            1.643  1.5597202401908667  \\
            2.562  1.5361936653781854  \\
            0.002  1.697893215077544  \\
            0.15  1.6979094714005778  \\
            0.615  1.6970319844096335  \\
            1.312  1.5766863843234862  \\
            1.141  1.5916952627070653  \\
            2.634  1.5346847316693688  \\
            1.828  1.553416708843167  \\
            0.014  1.6979326717537988  \\
            2.635  1.53466613664179  \\
            3.309  1.5212446492738996  \\
            0.025  1.6978974068936659  \\
            0.162  1.6978637695910128  \\
            3.36  1.5211729645528749  \\
            3.412  1.5211132656937436  \\
            1.991  1.549008401317309  \\
            0.534  1.6973267703566277  \\
            0.708  1.6955831782349766  \\
            3.047  1.5236730345782057  \\
            2.735  1.5324050539631113  \\
            2.375  1.5402818837144117  \\
            1.939  1.5504017980063927  \\
            1.507  1.5654210748461685  \\
            0.312  1.6978197057719753  \\
            2.034  1.5479301290542555  \\
            3.269  1.521326634580007  \\
            1.573  1.5625347279792379  \\
            0.925  1.6272933962888643  \\
            1.275  1.5794428902639706  \\
            2.606  1.535232534770973  \\
            2.649  1.5343120903410175  \\
            0.144  1.6978405222249737  \\
            2.349  1.5408131623456187  \\
            2.428  1.538997998112501  \\
            2.9  1.5282457642677927  \\
            3.025  1.5244249580114777  \\
            0.828  1.6543265863221124  \\
            0.032  1.6978838782718715  \\
            1.748  1.5559636188693684  \\
            1.266  1.5800911873742678  \\
            2.616  1.5350155548570095  \\
            0.995  1.6126380756304568  \\
            3.4  1.5211260310380468  \\
            0.39  1.6976848016493729  \\
            3.397  1.521138580383532  \\
            1.921  1.5507953286807898  \\
            1.092  1.5972695454736054  \\
            1.682  1.5581880488368618  \\
            2.212  1.5436794827995515  \\
            2.829  1.5300645309283347  \\
            1.344  1.5744025401143424  \\
            3.292  1.5212812964010727  \\
            0.694  1.6957830624188532  \\
            1.864  1.552491176943181  \\
            0.414  1.6976627557406236  \\
            2.808  1.5306362012588228  \\
            2.103  1.5462060919165266  \\
            1.597  1.561679087866205  \\
            0.015  1.697899870154417  \\
            1.139  1.5918323322041499  \\
            2.745  1.5321424296049657  \\
            2.062  1.5472229071042065  \\
            0.049  1.6978761108157052  \\
            0.954  1.6207852167514842  \\
            1.286  1.5783346799284725  \\
            3.243  1.5214210896455218  \\
            2.037  1.5478461400605572  \\
            3.096  1.5224225373129332  \\
            0.464  1.6975998801215324  \\
            0.976  1.6156420169732688  \\
            2.783  1.5312746414931406  \\
            0.614  1.6971993415330557  \\
            0.807  1.6671886104003653  \\
            0.701  1.6954266348339007  \\
            0.773  1.6827515731041247  \\
            2.696  1.533304102908444  \\
            3.158  1.5218083477445736  \\
            2.309  1.5415705843036105  \\
            2.769  1.5316160214099968  \\
            2.257  1.5427679657006934  \\
            2.524  1.5370536453536536  \\
            1.247  1.5817258879019758  \\
            0.189  1.6979311539353856  \\
            3.185  1.5216443027301327  \\
            0.036  1.697893089309006  \\
            3.359  1.5211707802802694  \\
            0.667  1.6960999324400043  \\
            0.077  1.697887259970264  \\
            1.888  1.5517709307126133  \\
            3.249  1.5213443884809574  \\
            2.512  1.5373414417082725  \\
            1.017  1.6085719766675113  \\
            2.791  1.5310427797445452  \\
            0.256  1.697882489346195  \\
            1.6  1.561274160614247  \\
            3.181  1.521625755946464  \\
            2.579  1.5358626546590801  \\
            0.461  1.6976322672710769  \\
            1.538  1.5641529319141148  \\
            3.283  1.52131992514631  \\
            2.564  1.5361985488877163  \\
            0.986  1.6134995809287278  \\
            2.326  1.5412459513014742  \\
            1.25  1.5815104215354872  \\
            3.474  1.5210647119727039  \\
            0.8  1.6706811100930983  \\
            3.241  1.5213841266043655  \\
            0.121  1.6978813449677157  \\
            0.441  1.697670070876691  \\
            1.841  1.553146953399894  \\
            1.11  1.5956037062618402  \\
            1.797  1.554518769406884  \\
            0.753  1.6869539352093819  \\
            2.031  1.547949540046932  \\
            1.395  1.5713791767296117  \\
            2.407  1.5394789955914325  \\
            2.372  1.5403208696819208  \\
            1.133  1.592939774086857  \\
            2.023  1.5482033401929298  \\
            0.672  1.696211837063222  \\
            3.203  1.52154277956309  \\
            2.499  1.5375306910903046  \\
            2.379  1.540078288115783  \\
            2.438  1.5388000667317268  \\
            3.056  1.5234596185514386  \\
            2.053  1.5474794921540864  \\
            2.857  1.5294373453041041  \\
            0.833  1.6528184489457287  \\
            0.86  1.6444070693125588  \\
            2.621  1.5349263369902697  \\
            0.705  1.695556425123695  \\
            2.595  1.5355041734735377  \\
            2.342  1.5409164787947642  \\
            2.039  1.5477720232609973  \\
            1.107  1.5955861207582152  \\
            0.649  1.6969570737305202  \\
            2.741  1.532203776701438  \\
            0.699  1.6957300080417264  \\
            2.333  1.5411484939883162  \\
            0.782  1.6746065743466882  \\
            1.83  1.553451136912131  \\
            3.237  1.5214106601231534  \\
            0.142  1.69786481581876  \\
            0.727  1.6945425976438369  \\
            2.801  1.5307786353501027  \\
            2.275  1.5423547497769243  \\
            1.969  1.5495846909614077  \\
            2.131  1.5455594221282576  \\
            1.28  1.5790295986863043  \\
            0.814  1.6647539478796052  \\
            0.556  1.6973897839342411  \\
            1.282  1.5787370235806415  \\
            0.845  1.6497559568882896  \\
            0.868  1.6433283528202782  \\
            2.868  1.5290313719995763  \\
            2.998  1.5251342670268466  \\
            1.592  1.561835435714543  \\
            0.085  1.6978924077743547  \\
            1.622  1.5607003171899143  \\
            2.022  1.5481865190832913  \\
            1.973  1.5493804008267757  \\
            2.905  1.528024668042821  \\
            1.699  1.5576156369540937  \\
            2.175  1.544495929207293  \\
            1.925  1.5507199168007932  \\
            0.012  1.6978878133652164  \\
            2.233  1.543296126109512  \\
            1.005  1.6107434036545023  \\
            1.685  1.558173553879521  \\
            0.728  1.6938194468736392  \\
            0.605  1.697155893048824  \\
            3.192  1.5215476684340512  \\
            0.196  1.6978685933723032  \\
            1.501  1.5657077076088082  \\
            3.282  1.521307645262806  \\
            1.233  1.582788615127619  \\
            1.016  1.6081131134612487  \\
            0.349  1.6977641005310253  \\
            2.263  1.5426391593914153  \\
            1.07  1.6005222914896151  \\
            0.765  1.6827603500561374  \\
            0.305  1.6978043253751376  \\
            1.48  1.5667550165897173  \\
            1.505  1.5654067499282227  \\
            1.186  1.587222094243998  \\
            0.111  1.6978760885110666  \\
            1.155  1.5901917054794295  \\
            3.045  1.5237340662915737  \\
            3.128  1.5220712779322674  \\
            3.336  1.5212168395935814  \\
            2.915  1.5277229424840915  \\
            0.013  1.6979066126936653  \\
            3.251  1.5213574979595894  \\
            2.992  1.5254341203393247  \\
            1.926  1.5507193758186864  \\
            3.054  1.5234834215796562  \\
            1.104  1.595866556066935  \\
            2.376  1.5401430658503674  \\
            1.983  1.549195882822349  \\
            2.158  1.54492280303264  \\
            2.467  1.5382283989008052  \\
            1.179  1.5873543834439727  \\
            1.701  1.5576981179747518  \\
            3.436  1.5210943155038694  \\
            0.794  1.6719338657002412  \\
            0.311  1.6978070134049665  \\
            0.98  1.615271553676925  \\
            1.382  1.5721289153654991  \\
            1.316  1.5761518111223116  \\
            2.753  1.5319529166263495  \\
            2.612  1.5351337216325542  \\
            1.003  1.6113632232791313  \\
            1.396  1.5715549401894506  \\
            0.635  1.6972799847839783  \\
            0.967  1.617775189293828  \\
            0.514  1.6975668899671637  \\
            0.846  1.6515976976340776  \\
            3.247  1.5213875769006884  \\
            2.151  1.545137270985294  \\
            1.822  1.5536348811240956  \\
            1.906  1.5512124191622052  \\
            3.14  1.5218918282965204  \\
            0.718  1.695147118199997  \\
            1.258  1.5807365328846763  \\
            1.243  1.581995728432155  \\
            2.93  1.5273199320995452  \\
            2.744  1.5321459371448563  \\
            1.81  1.554033383357984  \\
            0.239  1.6977835415160616  \\
            2.913  1.527848632714423  \\
            1.454  1.5680042047336553  \\
            2.873  1.5289036765294732  \\
            2.889  1.5285365673086417  \\
            0.675  1.696111570532077  \\
            2.015  1.5483281643940423  \\
            2.489  1.537767582481718  \\
            3.18  1.5216552125293523  \\
            0.169  1.697918058060443  \\
            3.215  1.5214941774782214  \\
            1.511  1.565351418229851  \\
            3.06  1.5233679802846647  \\
            2.807  1.5305393475763853  \\
            0.402  1.6976928017546558  \\
            2.684  1.5335221514630908  \\
            1.882  1.551868022452349  \\
            2.571  1.536001822044812  \\
            2.956  1.526552690194477  \\
            1.242  1.5819131256779264  \\
            0.838  1.6521420407798038  \\
            2.48  1.5379182661718842  \\
            0.046  1.697880577246145  \\
            0.564  1.6972221510125476  \\
            2.291  1.5419932108354268  \\
            3.403  1.5211246008154236  \\
            0.12  1.6978462876467304  \\
            0.959  1.6188699540277358  \\
            2.421  1.539203770768049  \\
            0.706  1.6946771426456217  \\
            0.159  1.6978903430036916  \\
            0.561  1.697339491674524  \\
            0.697  1.6960653649649349  \\
            0.037  1.6978920677770466  \\
            1.728  1.5566623538539925  \\
            0.432  1.6976476319153575  \\
            2.382  1.5400521436672008  \\
            3.179  1.5216589421055409  \\
            2.662  1.534046996617984  \\
            3.0  1.52518107270609  \\
            0.858  1.6460830109861477  \\
            0.606  1.6972605987837157  \\
            0.47  1.6975789090535776  \\
            2.682  1.5335663292183914  \\
            3.34  1.521186184640675  \\
            1.934  1.5505085780917038  \\
            2.348  1.54070743142826  \\
            2.615  1.5350733365134797  \\
            0.916  1.6293029435280484  \\
            0.173  1.6978046643908253  \\
            1.097  1.5967138348160657  \\
            0.21  1.697886874245256  \\
            0.247  1.6978766730782093  \\
            2.287  1.542064559440657  \\
            2.186  1.5442827353821151  \\
            1.174  1.5883292122128752  \\
            2.498  1.537523119370846  \\
            1.593  1.5616066656085064  \\
            0.89  1.6369411087419152  \\
            3.311  1.5212421166075576  \\
            2.628  1.5347779186071888  \\
            1.568  1.5627623569417193  \\
            2.559  1.536297761083515  \\
            0.489  1.697575562165038  \\
            2.928  1.5274627117453594  \\
            3.453  1.5210771399422034  \\
            0.061  1.6978662222905507  \\
            2.017  1.5482752024649622  \\
            1.34  1.5746209005723095  \\
            2.959  1.5264016036144081  \\
            2.25  1.5429266004319302  \\
            1.273  1.5797716745137098  \\
            2.356  1.540569523264159  \\
            3.258  1.5213364085377417  \\
            3.232  1.5214141669538894  \\
            0.776  1.6828624703287227  \\
            0.714  1.6943245021893936  \\
            0.599  1.6971891819019478  \\
            0.006  1.6979032567157293  \\
            1.412  1.5701162300812332  \\
            3.201  1.521560946067759  \\
            2.561  1.5362219965237922  \\
            3.406  1.5211116996883447  \\
            2.129  1.5456323179864901  \\
            2.061  1.5471892594698218  \\
            0.174  1.6978222463208024  \\
            2.479  1.5379937453490282  \\
            1.455  1.5680480050240815  \\
            0.844  1.6513399000312103  \\
            0.648  1.6968968663043398  \\
            1.257  1.5807317636777791  \\
            3.11  1.5222557672762187  \\
            1.922  1.550789766037227  \\
            3.144  1.5218739627505806  \\
            1.414  1.5702494938955445  \\
            3.213  1.5215008453713197  \\
            1.656  1.559136887233568  \\
            0.424  1.6975894635671687  \\
            1.633  1.5601023898336341  \\
            0.625  1.6969474920824055  \\
            0.31  1.6978332664925497  \\
            3.058  1.5234356045104303  \\
            1.292  1.5780247900637716  \\
            0.681  1.6961924966136828  \\
            0.717  1.6941073985206825  \\
            1.895  1.5516255387841746  \\
            3.032  1.5240135398142975  \\
            3.419  1.5211040157618771  \\
            0.063  1.697866770497563  \\
            1.62  1.5606985045540802  \\
            2.722  1.5326537646344376  \\
            0.83  1.6549330034020049  \\
            3.45  1.5210832969321162  \\
            2.27  1.5424649891961366  \\
            2.35  1.540705884509813  \\
            1.859  1.5525463798591932  \\
            3.492  1.5210479494552076  \\
            2.262  1.5426199410688646  \\
            3.134  1.5220891214866095  \\
            1.067  1.6006851489330247  \\
            1.054  1.6027600104673208  \\
            0.816  1.6599195920362286  \\
            1.872  1.5522268659550293  \\
            2.695  1.5333085702175324  \\
            3.279  1.521296477368367  \\
            0.88  1.6384829264425806  \\
            2.281  1.542251460907497  \\
            1.676  1.5584492487842565  \\
            1.844  1.5529372174982707  \\
            2.567  1.5360965449205797  \\
            2.626  1.5347705501286197  \\
            1.031  1.605698431383278  \\
            3.31  1.5212476819654417  \\
            2.729  1.5325218902088544  \\
            2.796  1.5308836393883134  \\
            1.185  1.586874469233012  \\
            0.202  1.697862453884715  \\
            3.472  1.521061821421626  \\
            1.009  1.6101534183222463  \\
            1.239  1.5823151647939597  \\
            2.16  1.5449281901750818  \\
            0.103  1.69787714098969  \\
            2.997  1.525235492084044  \\
            2.613  1.5350867567517326  \\
            3.137  1.5218762159344592  \\
            0.602  1.6970536038855555  \\
            0.641  1.696784046595391  \\
            2.188  1.544294303084153  \\
            1.8  1.554348527701818  \\
            1.397  1.571111646005574  \\
            0.205  1.697925346951852  \\
            2.374  1.5402012412921078  \\
            2.516  1.5372380967943384  \\
            1.13  1.5929646092053182  \\
            1.004  1.6108145363174653  \\
            3.163  1.521759395909697  \\
            2.42  1.5392783053387582  \\
            2.894  1.5283620617595515  \\
            2.041  1.5476730208962497  \\
            1.559  1.5631731209104487  \\
            0.178  1.6978617083151595  \\
            1.67  1.5586825989837694  \\
            1.723  1.5568148228926666  \\
            3.486  1.5210526857703102  \\
            0.817  1.660765198273016  \\
            1.948  1.5501132305962884  \\
            1.211  1.5845247963964013  \\
            1.152  1.5904094138024212  \\
            2.122  1.5457812204226016  \\
            0.526  1.6974393802761711  \\
            3.363  1.5211725255195345  \\
            3.131  1.5220240454463954  \\
            0.742  1.6924541273110312  \\
            3.012  1.5247745163685975  \\
            3.064  1.5233001189283997  \\
            1.416  1.5701752385688885  \\
            0.664  1.696411559337864  \\
            1.566  1.5627435665266656  \\
            2.747  1.5321032680735012  \\
            1.026  1.607341013362858  \\
            2.602  1.535353909599316  \\
            3.289  1.5212810519536357  \\
            3.17  1.5217322065428627  \\
            1.011  1.6098897741011113  \\
            2.734  1.5323392255634363  \\
            1.051  1.6029916711072414  \\
            1.332  1.5750444418725507  \\
            2.133  1.5455028673163356  \\
            2.397  1.5396891893810642  \\
            2.837  1.5298492115863382  \\
            2.702  1.5330566858010821  \\
            2.458  1.538458064544022  \\
            0.75  1.689281801741531  \\
            2.176  1.544550467720504  \\
            2.435  1.5389388732879026  \\
            0.16  1.6979404831265044  \\
            0.85  1.6497421405427999  \\
            2.798  1.5308253146883346  \\
            1.706  1.5575779436946031  \\
            3.489  1.5210596626799504  \\
            0.182  1.6979385005433127  \\
            2.497  1.5376242216453047  \\
            1.993  1.548993645267044  \\
            2.021  1.5481960753399548  \\
            1.164  1.589000761012062  \\
            1.72  1.557016356260466  \\
            2.362  1.5404895436932589  \\
            0.362  1.6978381785506504  \\
            2.087  1.5466372501340178  \\
            3.356  1.521176704709002  \\
            0.831  1.6559964796166984  \\
            2.859  1.5292662926206124  \\
            2.474  1.5380758204273137  \\
            1.339  1.5748424364317053  \\
            2.054  1.5473676809416121  \\
            1.703  1.5575356864631003  \\
            2.6  1.5354267057432749  \\
            2.792  1.530994922623355  \\
            2.668  1.533946665341297  \\
            0.622  1.6969577268592368  \\
            1.081  1.5988407255308297  \\
            2.076  1.5469005045724444  \\
            0.862  1.6450015007980483  \\
            0.893  1.6358449713871568  \\
            2.91  1.5278515384974451  \\
            3.12  1.522196776855649  \\
            2.035  1.5478552880184517  \\
            1.798  1.5543971854829721  \\
            0.211  1.6978559665451787  \\
            3.04  1.5238270933661018  \\
            2.94  1.5271471375058696  \\
            0.417  1.6977513995675926  \\
            1.44  1.568907951303848  \\
            2.846  1.5296473828962573  \\
            1.87  1.5523682104344876  \\
            3.417  1.5211165920073215  \\
            1.558  1.5630763408079076  \\
            0.661  1.6962870980629685  \\
            1.629  1.5601419531427196  \\
            0.936  1.6240249018861432  \\
            3.327  1.5212092495542788  \\
            2.001  1.5487967622378969  \\
            2.145  1.545250024628633  \\
            3.02  1.5244211898356061  \\
            1.802  1.5542160899193738  \\
            0.381  1.6977040392633773  \\
            1.867  1.5523684333493954  \\
            0.347  1.6977762087354342  \\
            2.018  1.548282193906807  \\
            2.809  1.530690621260157  \\
            1.555  1.5632085138350946  \\
            0.134  1.6979223518722288  \\
            0.436  1.697717857479236  \\
            0.434  1.6976870303742992  \\
            0.099  1.6978821604653747  \\
            2.111  1.5460312756820285  \\
            2.095  1.5464284703275206  \\
            1.759  1.5555839060222714  \\
            3.325  1.5212282849278396  \\
            2.884  1.528684999348747  \\
            0.779  1.6784773510857531  \\
            0.488  1.6974991198914389  \\
            1.216  1.5844834979028997  \\
            0.507  1.6975485433049018  \\
            1.094  1.5974601048276167  \\
            1.145  1.5912633676644  \\
            0.052  1.697907777914622  \\
            3.477  1.5210610602108485  \\
            0.66  1.6967834480947652  \\
            1.623  1.560583320922405  \\
            0.445  1.697654725134493  \\
            3.102  1.5223382119933706  \\
            2.088  1.5465945628184212  \\
            3.05  1.5236166461949763  \\
            0.533  1.6975122542373509  \\
            2.213  1.5436820261437665  \\
            0.241  1.6978462121347964  \\
            1.183  1.5873417554110434  \\
            1.943  1.5502344838914943  \\
            0.52  1.697371936777817  \\
            3.414  1.5211125078675307  \\
            2.331  1.5411998234563091  \\
            1.039  1.6048702450207564  \\
            2.269  1.542442126407098  \\
            0.22  1.6978223290910452  \\
            1.459  1.5677522678963567  \\
            2.933  1.5272830220063174  \\
            0.293  1.6977918570698538  \\
            0.272  1.6977965230919299  \\
            3.068  1.5229577418039812  \\
            1.111  1.595584113058627  \\
            1.679  1.5584001366076472  \\
            0.216  1.6978736616485441  \\
            0.682  1.6963528784506838  \\
            0.979  1.6154704826921025  \\
            1.714  1.5570655169357348  \\
            0.004  1.6978739628832527  \\
            2.006  1.5485667009187134  \\
            1.363  1.5730952140572951  \\
            1.076  1.5998412219109612  \\
            1.795  1.5544937412593505  \\
            0.183  1.6978402130741737  \\
            2.659  1.5340866715814414  \\
            1.177  1.587766449663036  \\
            0.826  1.6594915627249427  \\
            0.413  1.6977294115960193  \\
            0.639  1.697056672998973  \\
            0.411  1.6976751066115232  \\
            2.871  1.528904928783614  \\
            0.035  1.6979400137219016  \\
            1.376  1.5725006221253657  \\
            2.689  1.533431653324524  \\
            2.412  1.5394103899034592  \\
            0.291  1.6977557691358078  \\
            1.984  1.5492252043322945  \\
            2.303  1.5417652504641006  \\
            2.369  1.540279943104458  \\
            1.05  1.6032542954713604  \\
            3.116  1.522196836446013  \\
            0.128  1.6978468732210448  \\
            2.922  1.5276217151577427  \\
            3.389  1.5211334346002863  \\
            3.079  1.52286299414667  \\
            2.514  1.5372250437504071  \\
            2.118  1.5458845459720711  \\
            3.015  1.5246531213408498  \\
            3.35  1.5211723924792127  \\
            0.482  1.6975250741159358  \\
            2.518  1.53709788808553  \\
            3.003  1.525108950848487  \\
            0.352  1.6976713758031403  \\
            2.069  1.5469819799744182  \\
            2.651  1.5342017124973188  \\
            0.118  1.6978893361955623  \\
            3.236  1.521405442372112  \\
            0.495  1.697598841584836  \\
            2.757  1.5318519089148726  \\
            0.443  1.6976515266534191  \\
            2.117  1.5458833535880008  \\
            1.954  1.5499527633912462  \\
            0.399  1.6975836274493765  \\
            2.947  1.5268213784010254  \\
            0.621  1.6967654836182835  \\
            1.543  1.5638253195061915  \\
            2.872  1.5289514422898647  \\
            2.427  1.5390969664781022  \\
            1.122  1.5942469861792354  \\
            1.585  1.5619028552791876  \\
            1.898  1.5514579807488877  \\
            3.265  1.5213477473895143  \\
            1.457  1.567801414292085  \\
            1.89  1.5516856476313325  \\
            1.238  1.5823115678732154  \\
            2.594  1.535534474457193  \\
            1.873  1.5521281365209039  \\
            0.068  1.6978302958017535  \\
            2.64  1.534568102263971  \\
            2.282  1.542169089250131  \\
            1.61  1.5609890221389164  \\
            1.653  1.5592545052255689  \\
            2.293  1.54190608806531  \\
            2.337  1.5409956563909344  \\
            3.362  1.5211811236024617  \\
            1.127  1.593969588342378  \\
            2.194  1.5440996151761122  \\
            2.766  1.5315945342485038  \\
            2.957  1.5266072829752098  \\
            2.546  1.5365508886699724  \\
            2.45  1.5385755461814123  \\
            0.837  1.6534323540528213  \\
            2.125  1.5456568252277925  \\
            2.172  1.544527899756564  \\
            2.196  1.5441257159787893  \\
            0.757  1.686620328161402  \\
            1.074  1.6000907179026997  \\
            1.409  1.5704336489837643  \\
            2.426  1.5391322608691445  \\
            2.764  1.5316896847574746  \\
            2.788  1.5311040686041826  \\
            3.233  1.5214097659171844  \\
            2.243  1.5429968144135549  \\
            1.935  1.5504103742825823  \\
            1.907  1.551192681469902  \\
            2.68  1.5336133541528172  \\
            1.151  1.5904789171670033  \\
            0.041  1.6979257092060445  \\
            1.504  1.5655843293755602  \\
            3.441  1.521098995383268  \\
            0.67  1.6966397905797264  \\
            2.575  1.535903575423966  \\
            0.472  1.697571925561134  \\
            2.353  1.5406277574498706  \\
            3.381  1.5211429078835423  \\
            2.806  1.5306552894634635  \\
            1.976  1.5493530484344813  \\
            3.312  1.5212382880573978  \\
            0.974  1.616518749386603  \\
            0.058  1.6979150460933818  \\
            1.766  1.555413404103096  \\
            0.802  1.6663344022133058  \\
            1.647  1.5595629188084548  \\
            1.319  1.576128604672463  \\
            2.763  1.531724113458102  \\
            1.998  1.5487571962252606  \\
            1.664  1.558842781230397  \\
            0.871  1.6406173531389674  \\
            0.308  1.6977703216825635  \\
            2.192  1.5442138050323262  \\
            0.887  1.6383560389202505  \\
            2.446  1.53871631236322  \\
            0.107  1.6979288684428337  \\
            0.073  1.6979104851158653  \\
            3.442  1.5210893871886517  \\
            2.371  1.5402488935170677  \\
            2.799  1.5307846271861965  \\
            3.496  1.5210444748321008  \\
            2.354  1.5406129073940917  \\
            2.247  1.5430120761844963  \\
            0.562  1.6974958544435006  \\
            1.757  1.5556911776566256  \\
            0.998  1.6122968417581585  \\
            0.796  1.6726249078777249  \\
            1.173  1.5885948818895619  \\
            0.825  1.6565667773513828  \\
            1.044  1.6042769772267855  \\
            2.803  1.5307399387589444  \\
            2.718  1.5327413519464521  \\
            2.002  1.5486402618131037  \\
            0.401  1.6976027059510588  \\
            3.476  1.5210598236223511  \\
            1.427  1.5696305956171108  \\
            1.065  1.601581904281664  \\
            0.409  1.6976845284626978  \\
            0.057  1.6978707997223863  \\
            1.487  1.5662905226242227  \\
            0.855  1.646977464711163  \\
            2.738  1.532259176460562  \\
            2.818  1.5303561988878358  \\
            0.444  1.6977283383961135  \\
            3.052  1.5235072269037555  \\
            0.61  1.6969802710318427  \\
            1.631  1.5602574193963876  \\
            1.093  1.5974201984528138  \\
            1.7  1.5576779192907275  \\
            2.56  1.5362003280928782  \\
            2.574  1.5360102033634302  \\
            2.297  1.541871315613565  \\
            1.229  1.5833131247323593  \\
            1.433  1.5691840761931815  \\
            0.791  1.677498847797368  \\
            0.001  1.697891331594167  \\
            0.056  1.697874623361833  \\
            0.516  1.697620329545388  \\
            1.157  1.589868684735151  \\
            2.728  1.5325796344780587  \\
            1.429  1.5693400940522142  \\
            0.269  1.6978236411118  \\
            3.452  1.521068560706919  \\
            3.067  1.5231075514816104  \\
            2.51  1.5373050995114101  \\
            2.063  1.5471131515631145  \\
            2.036  1.5478507356940956  \\
            0.914  1.6301673407803599  \\
            2.15  1.5451820562355139  \\
            1.481  1.5666645535439685  \\
            2.328  1.5411756165362436  \\
            3.114  1.5222744627720253  \\
            1.486  1.5664200144528992  \\
            2.253  1.5428458797806859  \\
            2.861  1.5293107907497903  \\
            2.756  1.5318180819586078  \\
            1.237  1.5824639558683464  \\
            2.071  1.5469522745083382  \\
            2.968  1.5261750463113908  \\
            0.155  1.6978780303930336  \\
            1.837  1.5532356494857273  \\
            3.07  1.5230340672877707  \\
            1.928  1.5506990230853457  \\
            0.693  1.69610860246166  \\
            1.927  1.5506685728844765  \\
            0.433  1.6976332158255039  \\
            2.714  1.5328698257238484  \\
            0.818  1.6597453270480147  \\
            0.299  1.697873050946051  \\
            1.761  1.555710912492112  \\
            2.737  1.5323382033429518  \\
            1.271  1.5798092590568034  \\
            0.314  1.6977715686604973  \\
            1.707  1.557307773781564  \\
            2.417  1.5393390352335488  \\
            1.918  1.5508684563395858  \\
            2.168  1.5447129387965093  \\
            1.123  1.5936227897081938  \\
            2.267  1.5425148292604378  \\
            1.032  1.6058751876981159  \\
            0.772  1.6834543229483407  \\
            0.453  1.6976606204092841  \\
            2.785  1.5311716983902222  \\
            2.249  1.5428497174247815  \\
            1.916  1.550979320656214  \\
            0.2  1.6978558423762533  \\
            0.48  1.6976461756635728  \\
            1.776  1.5550739780703713  \\
            2.209  1.5438201325968093  \\
            0.927  1.6259124955961177  \\
            0.288  1.6977699579903012  \\
            2.093  1.5465370186648537  \\
            1.737  1.556446736674789  \\
            2.094  1.5464198889093823  \\
            2.074  1.5469089482699188  \\
            3.276  1.5213027141565758  \\
            0.11  1.6978610021137854  \\
            3.293  1.5212752277286128  \\
            1.957  1.5498615733964713  \\
            1.738  1.5563208378338904  \\
            2.256  1.5427009515019083  \\
            1.751  1.5559200922224456  \\
            3.123  1.5221085207499816  \\
            1.702  1.557548015813118  \\
            2.099  1.546298084790893  \\
            0.644  1.6970012592973747  \\
            1.424  1.5696378295967044  \\
            1.583  1.5620026062918444  \\
            0.477  1.6976507455882766  \\
            0.81  1.6633285288929347  \\
            3.15  1.5218442458286758  \\
            0.067  1.6978715327433394  \\
            3.235  1.5214419263423469  \\
            2.167  1.5447303131745853  \\
            2.34  1.5409505312907994  \\
            1.649  1.5594331473407212  \\
            1.042  1.6050788387081583  \\
            2.259  1.542740069808941  \\
            0.246  1.6978389260783802  \\
            0.881  1.6381099242751307  \\
            3.151  1.5218464881943494  \\
            0.726  1.6950450790711835  \\
            2.593  1.5355257350550813  \\
            1.683  1.5581705902242042  \\
            1.298  1.5777222402381825  \\
            1.911  1.551078496282584  \\
            1.959  1.5498218475504537  \\
            1.01  1.6097287717006366  \\
            1.783  1.5548784323972635  \\
            1.334  1.5753201353828883  \\
            2.678  1.5337027734284527  \\
            2.784  1.531139506254283  \\
            2.261  1.5426134660507937  \\
            3.418  1.5211208214494156  \\
            1.966  1.549613848092093  \\
            1.413  1.5704193890590472  \\
            1.813  1.5539580118632943  \\
            3.288  1.5212675413036423  \\
            2.619  1.535016181732108  \\
            1.492  1.5661773676704676  \\
            2.137  1.5454642958678664  \\
            0.93  1.6260915583365714  \\
            1.482  1.566553873750713  \\
            1.194  1.5862876375864214  \\
            1.989  1.5490293341334203  \\
            2.984  1.5256634228089245  \\
            1.297  1.5778594227522915  \\
            3.386  1.5211455611790805  \\
            0.733  1.6941085697889182  \\
            1.236  1.582667500410768  \\
            2.415  1.53931855641437  \\
            3.443  1.5210863990834094  \\
            2.067  1.547094831201005  \\
            1.46  1.5678326607604007  \\
            1.126  1.5936274145703035  \\
            1.603  1.5612602985786055  \\
            2.491  1.5377334928302722  \\
            2.58  1.5358843489418543  \\
            0.422  1.6977535075795094  \\
            1.362  1.573303584530234  \\
            1.651  1.5593225031931364  \\
            3.267  1.5213395308394981  \\
            1.364  1.5732248022710473  \\
            0.941  1.623599179170189  \\
            3.035  1.5240461738211555  \\
            0.029  1.697917456191878  \\
            3.432  1.5210990210081248  \\
            2.639  1.5344762667656426  \\
            2.875  1.528876957429368  \\
            2.114  1.5460153090822595  \\
            1.674  1.5585979036659352  \\
            0.738  1.694051214584007  \\
            2.343  1.5408660935927698  \\
            1.162  1.5894143698756853  \\
            1.326  1.5758102666313452  \\
            3.328  1.5212056359513106  \\
            2.543  1.536614450809864  \\
            0.598  1.6971997629918845  \\
            1.188  1.5871028140300907  \\
            3.164  1.5217764235101718  \\
            2.387  1.5399874794538055  \\
            1.556  1.5632847112261283  \\
            1.435  1.5692178188980443  \\
            1.58  1.5622394363687553  \\
            0.994  1.6132307587942636  \\
            1.53  1.5644028184393963  \\
            1.99  1.548981764978205  \\
            2.794  1.530955948618033  \\
            1.224  1.5831760031475022  \\
            1.561  1.5629325678981951  \\
            0.101  1.6978831341337048  \\
            2.482  1.5379250631065662  \\
            0.206  1.697842350916986  \\
            1.557  1.5632684864605921  \\
            2.195  1.544102842580023  \\
            2.123  1.5457228182800755  \\
            1.858  1.5524737919611749  \\
            1.94  1.550293815174399  \\
            2.999  1.5251265151073021  \\
            2.566  1.536127824182272  \\
            1.246  1.5815157867763558  \\
            3.323  1.5212305835710578  \\
            0.581  1.6971991587886288  \\
            3.273  1.5213117179150126  \\
            3.305  1.521250562222323  \\
            0.092  1.6978126018452009  \\
            0.575  1.697250002742412  \\
            1.228  1.5833981772235193  \\
            2.672  1.533758552189727  \\
            2.201  1.543903131605922  \\
            1.193  1.5861713323051325  \\
            2.994  1.5253350974916207  \\
            3.19  1.5215973328014456  \\
            1.904  1.5513022496339257  \\
            3.333  1.521218726981165  \\
            3.423  1.5211062671625348  \\
            1.658  1.5592057560277288  \\
            3.275  1.521296183876814  \\
            1.789  1.5546949181765581  \\
            2.996  1.525261573546489  \\
            1.387  1.5717589291816247  \\
            3.199  1.5215438913889148  \\
            1.308  1.576900719317626  \\
            2.184  1.544368807293784  \\
            1.513  1.56507711192729  \\
            2.183  1.544298144283454  \\
            1.299  1.577707864875253  \\
            3.38  1.5211476318089787  \\
            2.174  1.5446815709442723  \\
            0.748  1.6923374982994615  \\
            1.524  1.5646637793633238  \\
            1.613  1.5608385195606247  \\
            1.138  1.5922212108676321  \\
            3.371  1.5211607014461763  \\
            2.339  1.5410003802931052  \\
            0.379  1.6977562019667978  \\
            0.467  1.6976288515461826  \\
            2.105  1.546217623640057  \\
            0.249  1.697823586839758  \\
            1.29  1.578248657148865  \\
            1.083  1.5989826141561578  \\
            1.711  1.557272361435909  \\
            0.258  1.6978364419957819  \\
            3.207  1.5215213429447916  \\
            2.222  1.5434889174347672  \\
            2.283  1.5421426329314503  \\
            3.39  1.5211330319913265  \\
            2.258  1.5427345502973082  \\
            0.152  1.697927747123699  \\
            2.553  1.5364165505698395  \\
            3.3  1.5212524558883231  \\
            0.388  1.6977227161154191  \\
            1.733  1.556443070214264  \\
            3.304  1.5212644962176618  \\
            0.571  1.6973500035779878  \\
            2.208  1.5438617290813406  \\
            1.857  1.5527240284764294  \\
            0.231  1.6978515617880099  \\
            1.894  1.5516006386028136  \\
            0.296  1.697838122127586  \\
            1.222  1.5839148193861292  \\
            0.745  1.69117955090341  \\
            2.086  1.5466104426906997  \\
            1.22  1.5838734511065151  \\
            3.411  1.5211214136225106  \\
            1.77  1.5553303571101416  \\
            2.529  1.5368974627169925  \\
            0.248  1.697820553217221  \\
            3.059  1.5233370816483258  \\
            1.648  1.5595020300094833  \\
            2.887  1.5285712638288003  \\
            0.3  1.69782372081917  \\
            0.219  1.6978115176026798  \\
            3.229  1.521440568696349  \\
            2.392  1.5398133295217398  \\
            3.469  1.521072106793036  \\
            2.705  1.5331206403389543  \\
            1.71  1.5573049747814274  \\
            0.698  1.695358868183331  \\
            1.821  1.5536416342228503  \\
            0.536  1.6974953145076095  \\
            2.322  1.5413194120219185  \\
            1.002  1.611069282323471  \\
            3.335  1.521216356349921  \\
            0.926  1.6267549650318778  \\
            2.608  1.535221178967315  \\
            1.881  1.5518783261549784  \\
            0.268  1.6978293440615717  \\
            3.141  1.5219395853883535  \\
            3.319  1.5212379864950951  \\
            0.567  1.6971243222148962  \\
            1.671  1.5586311302349256  \\
            0.14  1.6979051785834847  \\
            0.616  1.6971465947213675  \\
            0.228  1.6978747173460909  \\
            0.257  1.6978392426920794  \\
            1.037  1.6049815049748126  \\
            2.862  1.529173352191327  \\
            2.132  1.545499676246625  \\
            1.028  1.6069948410405173  \\
            3.487  1.521051386102541  \\
            0.542  1.6974489765330214  \\
            2.644  1.5344426772438973  \\
            2.631  1.534693803614859  \\
            0.335  1.6978069106916784  \\
            3.491  1.5210475047556702  \\
            1.366  1.5728903625472765  \\
            1.929  1.5505591004371344  \\
            0.888  1.636498368164544  \\
            2.057  1.5472835702031644  \\
            2.625  1.5348706890592636  \\
            0.806  1.6651956225916995  \\
            2.777  1.5313776681838858  \\
            0.653  1.6963365162732051  \\
            1.232  1.5827008114730685  \\
            0.287  1.6978190870740009  \\
            0.671  1.6965791925488334  \\
            0.617  1.696846546327884  \\
            0.394  1.697696170200622  \\
            1.398  1.571108546330185  \\
            1.199  1.5859916810119792  \\
            0.274  1.6978773435669898  \\
            2.197  1.5439913746415748  \\
            1.368  1.5729660034142954  \\
            0.95  1.6212585477853523  \\
            3.316  1.5212396342671635  \\
            1.955  1.5499461171610596  \\
            3.285  1.5212936794472185  \\
            0.678  1.695961574968594  \\
            1.201  1.5854033630297732  \\
            0.958  1.6191131174374451  \\
            2.08  1.546788959739486  \\
            0.778  1.6817911067716216  \\
            2.135  1.5454585136034478  \\
            1.253  1.5812436438386483  \\
            1.786  1.5547335047446447  \\
            1.777  1.5550535483022656  \\
            2.622  1.534946112215923  \\
            0.082  1.6978897401162765  \\
            0.968  1.6174786300141895  \\
            3.132  1.521948172272626  \\
            1.931  1.550634785515441  \\
            0.108  1.6978821819158854  \\
            2.603  1.5352865384886383  \\
            2.748  1.5321018753197835  \\
            2.098  1.5463990772161071  \\
            0.971  1.6169226597179225  \\
            1.33  1.5753597344019026  \\
            0.683  1.6956964841561295  \\
            3.138  1.5219622871021552  \\
            0.059  1.6978795057050766  \\
            0.367  1.697790310691661  \\
            2.298  1.5418506866484722  \\
            0.332  1.6977650296586109  \\
            0.555  1.697399174425183  \\
            0.559  1.6974638783598488  \\
            0.486  1.69754076207448  \\
            1.464  1.5675912760475668  \\
            3.367  1.5211695462439343  \\
            0.365  1.6977695353039342  \\
            2.07  1.5470050064947953  \\
            2.736  1.5322875625164587  \\
            1.831  1.5534042573010733  \\
            0.513  1.6975032247486248  \\
            1.994  1.548794878705128  \\
            2.332  1.541073302354639  \\
            1.071  1.6007849833786327  \\
            1.843  1.5531427895200463  \\
            2.027  1.5480618095764083  \\
            3.458  1.5210716665097348  \\
            0.854  1.647030750740665  \\
            1.869  1.5522601672163163  \\
            1.335  1.5752365993259243  \\
            0.146  1.6978686323556367  \\
            1.688  1.5580423551042688  \\
            3.287  1.521277659553611  \\
            1.383  1.5721370580809633  \\
            0.519  1.6976139370528531  \\
            0.078  1.6978680504312038  \\
            2.104  1.5461654547419579  \\
            2.385  1.5399797597526304  \\
            2.439  1.5388822850990551  \\
            3.344  1.5212065834776536  \\
            0.873  1.6414689759204697  \\
            2.344  1.540883983335708  \\
            3.396  1.5211339115549238  \\
            0.583  1.6973580606260354  \\
            1.727  1.556704967261497  \\
            0.883  1.6383880779399027  \\
            0.313  1.697690526086371  \\
            1.582  1.56213939917684  \\
            0.532  1.6974539661484112  \\
            1.053  1.6031421204995044  \\
            2.787  1.5310890125535308  \\
            0.469  1.6976812639554766  \\
            2.009  1.548487838632375  \\
            2.523  1.5370678650424245  \\
            3.205  1.5215043016349967  \\
            0.938  1.6240710597763481  \\
            0.076  1.697860349268579  \\
            1.987  1.5490927114449011  \\
            2.455  1.5384824771019427  \\
            0.666  1.6963718531070515  \\
            0.84  1.6517666346993995  \\
            3.065  1.523076578491008  \\
            0.565  1.6974927600363405  \\
            2.852  1.5295114285799496  \\
            2.033  1.5479066854843768  \\
            0.637  1.6969719298902683  \\
            2.279  1.5422521735045072  \\
            2.731  1.5324488178988493  \\
            3.139  1.521985013965111  \\
            3.347  1.5212031810003057  \\
            0.592  1.6972349575636503  \\
            2.0  1.5487734493929968  \\
            1.975  1.5493797130180296  \\
            0.047  1.6979022979188585  \\
            3.351  1.5211807257329004  \\
            2.919  1.5276572316137484  \\
            2.357  1.5405784574965407  \\
            1.609  1.561006499971865  \\
            2.124  1.5456887935876815  \\
            3.331  1.5212034966496217  \\
            2.416  1.5393130671283792  \\
            1.956  1.5498551441588095  \\
            0.326  1.6977548769631539  \\
            2.11  1.5460978743078673  \\
            3.193  1.521580109399196  \\
            3.08  1.5227748950244857  \\
            3.343  1.5211986017869954  \\
            3.231  1.5214299882809432  \\
            2.576  1.5359330707089391  \\
            0.943  1.6223493722034492  \\
            0.715  1.6942772249180267  \\
            1.705  1.557541483345239  \\
            2.578  1.5358742321893402  \\
            2.193  1.5440884745183825  \\
            0.613  1.6969303068116568  \\
            2.484  1.537873646420799  \\
            1.426  1.5696581083622432  \\
            3.023  1.5244026267821003  \\
            1.384  1.5716496039648853  \\
            2.937  1.5271962982477643  \\
            3.091  1.5225687267460426  \\
            2.761  1.5317761915155435  \\
            2.238  1.5431285438096884  \\
            0.518  1.6974542262883454  \\
            2.1  1.5462967613354535  \\
            1.663  1.558867765194183  \\
            1.223  1.583911630426806  \\
            2.838  1.529860840155572  \\
            2.515  1.5371893916322008  \\
            0.64  1.69673892784243  \\
            0.341  1.6977438332607413  \\
            1.137  1.5921854049093305  \\
            0.92  1.6275413348792585  \\
            0.539  1.6974483483286658  \\
            1.886  1.5518240073949012  \\
            3.413  1.5211119800249666  \\
            0.746  1.6913168017131623  \\
            1.069  1.6008283422252587  \\
            0.94  1.6232305310956134  \\
            1.953  1.549972861464635  \\
            1.213  1.5844972528979975  \\
            1.146  1.5908205280117966  \\
            3.368  1.5211625806318056  \\
            0.204  1.6978610516750405  \\
            1.803  1.5541445052464582  \\
            2.089  1.5465235921145462  \\
            3.37  1.5211429256549878  \\
            0.747  1.6892192239618302  \\
            2.572  1.5359660831869413  \\
            0.588  1.6972251143063926  \\
            1.192  1.5865300063876797  \\
            1.359  1.5736245742340051  \\
            0.952  1.6204650806078589  \\
            0.813  1.663016474359299  \\
            3.112  1.5222177213609545  \\
            0.074  1.6979000743887056  \\
            2.366  1.540392648813105  \\
            2.229  1.5433574649915005  \\
            2.955  1.5266497138571253  \\
            0.767  1.68775607796192  \\
            0.911  1.6301766589777258  \\
            1.373  1.5727666649616832  \\
            3.444  1.5210808175355595  \\
            3.095  1.5225755387233089  \\
            2.367  1.5403726868673346  \\
            2.274  1.5423871263511983  \\
            0.96  1.618861624282605  \\
            1.202  1.5856994641610156  \\
            3.216  1.521495034392237  \\
            0.393  1.6977620340128583  \\
            0.395  1.6977004297993425  \\
            2.899  1.5281676012214058  \\
            2.199  1.5440448734053582  \\
            3.104  1.522310165795836  \\
            2.423  1.5392141846212712  \\
            0.407  1.6977319377316904  \\
            0.359  1.697765986246711  \\
            0.662  1.6968449637165228  \\
            2.601  1.5353735146099803  \\
            1.415  1.5701913041673954  \\
            0.321  1.6977189293269084  \\
            0.079  1.6978710891604987  \\
            2.831  1.5300314040817804  \\
            1.915  1.5510103355642575  \\
            1.269  1.5797708718570824  \\
            2.358  1.5405246023346026  \\
            1.889  1.551723785499116  \\
            1.171  1.5885673037366363  \\
            1.826  1.5534831586263242  \\
            0.168  1.6978998600044894  \\
            0.255  1.6977583288725138  \\
            2.712  1.5328845430125313  \\
            0.285  1.6977778776762364  \\
            1.861  1.5525467385208882  \\
            0.117  1.6978847918758768  \\
            3.092  1.522563774477938  \\
            1.113  1.5950312163203493  \\
            2.235  1.5432177341719617  \\
            1.865  1.552408042125552  \\
            0.221  1.6978899459191876  \\
            1.234  1.5828868145983959  \\
            1.977  1.5493115278336513  \\
            3.189  1.5215935423021671  \\
            2.62  1.5349709968585772  \\
            0.851  1.6481196638599818  \\
            3.318  1.5212299613078553  \\
            3.22  1.5214792093580305  \\
            1.444  1.5687116848307114  \\
            2.051  1.5474997340204666  \\
            1.746  1.5560519428307806  \\
            0.692  1.6955867606633415  \\
            1.204  1.5854259459487576  \\
            1.385  1.5720335968773593  \\
            1.85  1.5528477891559331  \\
            2.148  1.5451680181927494  \\
            2.789  1.5310678995689788  \\
            1.381  1.572152437233475  \\
            3.119  1.5221986330117454  \\
            0.594  1.6971755849879337  \\
            2.085  1.5466641323883978  \\
            2.749  1.532012548037849  \\
            0.457  1.6976393520439794  \\
            2.72  1.5327470868413058  \\
            0.787  1.6751173718596806  \\
            2.711  1.5329063057496988  \\
            2.449  1.5386212583871919  \\
            1.799  1.5543397348055865  \\
            0.674  1.6962552399433097  \\
            0.019  1.6978651500590252  \\
            0.176  1.6978621731253978  \\
            2.363  1.5404355882785392  \\
            0.003  1.697884020400505  \\
            0.656  1.6966040265641795  \\
            1.941  1.5502227655942142  \\
            0.114  1.697892619551964  \\
            1.26  1.5805441177863029  \\
            0.177  1.6978300845481722  \\
            1.294  1.5778393648624267  \\
            2.751  1.5319599470330396  \\
            0.608  1.6972375689760506  \\
            1.591  1.5617354519363176  \\
            2.525  1.5370523035058428  \\
            1.855  1.552660617635101  \\
            1.944  1.5502399074305713  \\
            0.786  1.6775243978614018  \\
            1.724  1.5567542285845308  \\
            0.055  1.69790590578696  \\
            2.853  1.5294989097913176  \\
            1.477  1.5670715054678455  \\
            1.06  1.6012999709751332  \\
            2.658  1.534069170930098  \\
            0.885  1.638291465238935  \\
            1.897  1.5514094900885296  \\
            2.904  1.5280620945471903  \\
            0.652  1.6966399415557711  \\
            2.459  1.538444558109611  \\
            3.366  1.5211627328669401  \\
            1.547  1.5636015473502982  \\
            0.713  1.694140413715397  \\
            3.018  1.5245139621192352  \\
            0.175  1.6978690286622624  \\
            0.78  1.6744047852243267  \\
            1.447  1.5685387575676175  \\
            0.222  1.697800108723385  \\
            2.277  1.5423076222581136  \\
            0.46  1.6976149682961594  \\
            1.12  1.593917822921383  \\
            2.642  1.534456413033204  \\
            3.008  1.5250175788964386  \\
            0.781  1.6775148158471658  \\
            2.733  1.5323626023461456  \\
            2.043  1.5477028797187138  \\
            1.105  1.595764303010743  \\
            2.468  1.5381626676066467  \\
            2.985  1.5257111511162342  \\
            2.883  1.528732036819422  \\
            2.936  1.5272466805259108  \\
            1.132  1.5925330367077966  \\
            2.903  1.5281368220370344  \\
            3.484  1.5210477354655285  \\
            1.095  1.5970362572033177  \\
            0.415  1.6976171322543174  \\
            2.18  1.5444417289499959  \\
            1.806  1.5541891929450344  \\
            0.566  1.6972931561743312  \\
            0.546  1.69740774720302  \\
            1.467  1.5672995865982586  \\
            2.965  1.5262793003256214  \\
            2.752  1.5319803992802417  \\
            1.667  1.55885388495227  \\
            1.466  1.5675287621867333  \\
            1.971  1.5495548867225057  \\
            1.465  1.5675432077746967  \\
            2.623  1.5348732468462822  \\
            0.676  1.696384708556581  \\
            0.568  1.6973898397925284  \\
            1.995  1.548902367319877  \\
            2.99  1.5255159108681924  \\
            1.321  1.575927325615594  \\
            0.412  1.6977808100152425  \\
            2.779  1.5312735228772232  \\
            0.612  1.6971179490290753  \\
            1.052  1.6032130636854287  \\
            1.972  1.5494430960383194  \\
            2.472  1.5381249451359058  \\
            0.496  1.697569024232359  \\
            1.17  1.5885916576648895  \\
            1.866  1.5523447045074166  \\
            1.469  1.5674467226112467  \\
            2.289  1.542086325918579  \\
            3.488  1.5210512641989655  \\
            0.739  1.693713062090221  \\
            2.692  1.5334105724809308  \\
            0.391  1.6977381353793903  \\
            1.279  1.5790333726314376  \\
            1.732  1.5565597860999518  \\
            3.439  1.5210964028770275  \\
            2.552  1.5363931930182018  \\
            0.123  1.697915814161927  \\
            0.19  1.6978289055605216  \\
            3.182  1.5216713381577227  \\
            2.024  1.5481090106456648  \\
            1.686  1.5581434398795657  \\
            3.479  1.5210518147710506  \\
            0.596  1.697209144394698  \\
            0.946  1.6217378152059887  \\
            1.985  1.5490275203194799  \\
            1.124  1.593428871594873  \\
            1.129  1.592970941560568  \\
            0.405  1.697717308600775  \\
            2.485  1.537833479931866  \\
            1.461  1.5675851679197434  \\
            2.396  1.5397919284901902  \\
            0.427  1.6976181363347522  \\
            2.743  1.5321780470766861  \\
            1.923  1.5507090012445195  \\
            3.465  1.5210707088094777  \\
            0.589  1.6971284018120978  \\
            2.532  1.5369005023898936  \\
            3.326  1.5212195836674691  \\
            3.307  1.5212583792938519  \\
            0.33  1.6977689876095432  \\
            2.604  1.5353386275808123  \\
            0.084  1.6978009509184206  \\
            1.189  1.5868614572195217  \\
            2.395  1.5397532096148803  \\
            3.295  1.5212702092494572  \\
            2.471  1.5381161280283588  \\
            3.41  1.5211181533618072  \\
            2.405  1.5394671945871286  \\
            2.433  1.5389387471457545  \\
            3.149  1.5219368384859995  \\
            1.823  1.5536155726806402  \\
            2.244  1.5429481018894395  \\
            2.953  1.5266283429426093  \\
            3.098  1.5224962043193921  \\
            1.967  1.549605770049246  \\
            0.657  1.6969551717877642  \\
            1.149  1.5908221509473308  \\
            0.438  1.6976622532221188  \\
            2.691  1.5333675669710551  \\
            0.309  1.6977777968919112  \\
            0.74  1.6926582820309706  \\
            0.763  1.6831170701150788  \\
            2.198  1.5440139212060473  \\
            3.198  1.5215549281702547  \\
            3.43  1.521107117524239  \\
            1.548  1.5635489701657062  \\
            1.91  1.5511552043201247  \\
            1.184  1.586989436977145  \\
            2.582  1.535822367243656  \\
            0.157  1.6978499311864028  \\
            1.175  1.5882231581231785  \\
            0.685  1.6959282454142233  \\
            2.506  1.5373770843832364  \\
            2.008  1.548525847155542  \\
            3.456  1.5210790166570645  \\
            2.79  1.5310057585168138  \\
            2.004  1.5486820638099401  \\
            1.669  1.5587748675316113  \\
            0.376  1.697704005623505  \\
            3.33  1.5212216553345648  \\
            0.636  1.6970552274923578  \\
            2.503  1.5374322667574691  \\
            2.368  1.540327050807947  \\
            0.97  1.6170186436522356  \\
            3.084  1.5227140077423111  \\
            2.916  1.5277505454677365  \\
            1.147  1.5908740420105751  \\
            1.168  1.5888490851543664  \\
            0.048  1.6978832365499168  \\
            2.37  1.5402737992426734  \\
            0.942  1.6225672344837514  \\
            3.029  1.5241936269265008  \\
            0.265  1.6978230554047657  \\
            1.787  1.5547851622179683  \\
            1.043  1.604905569901034  \\
            1.515  1.5651822120099952  \\
            2.663  1.5339912454116669  \\
            3.226  1.5214429831950154  \\
            1.677  1.558560184188783  \\
            1.323  1.5759243048496812  \\
            1.594  1.5615786103264506  \\
            2.24  1.5431261873117064  \\
            1.595  1.561689652012537  \\
            1.979  1.5493117063651196  \\
            2.318  1.5414632065572647  \\
            2.967  1.526282567619447  \\
            2.025  1.5480843753617577  \\
            0.345  1.6977137628898415  \\
            0.642  1.6967839125967463  \\
            0.574  1.6973894250959325  \\
            1.278  1.57906257305291  \\
            0.307  1.6977822693060232  \\
            2.04  1.5477839739091892  \\
            3.186  1.521596925960435  \\
            1.876  1.5519991949846017  \\
            1.65  1.559538578038134  \\
            3.341  1.521200935919527  \\
            2.159  1.544905412125194  \\
            2.068  1.5470648392399118  \\
            2.977  1.5259101844967735  \\
            0.978  1.6152842776977328  \\
            0.185  1.697900169675102  \\
            2.149  1.5451462194009995  \\
            2.128  1.545605349386973  \\
            1.736  1.5564347249351103  \\
            0.075  1.6978828665142716  \\
            1.715  1.557099154373994  \\
            0.934  1.6250233650243602  \\
            2.703  1.5331794178005071  \\
            1.804  1.5542353508865376  \\
            0.878  1.6391728652653639  \\
            1.378  1.5722266640631224  \\
            2.944  1.526936857425722  \\
            2.683  1.5335797678896461  \\
            0.224  1.6978340982561744  \\
            0.161  1.6978757542430418  \\
            2.896  1.5283412447170905  \\
            0.863  1.6450904804000404  \\
            0.492  1.697536828763303  \\
            2.205  1.5438371053819469  \\
            1.74  1.556156125353892  \\
            1.884  1.5519022885901008  \\
            2.719  1.532714526144098  \\
            0.553  1.69746764359636  \\
            0.576  1.6973617593994936  \\
            1.628  1.560345528869104  \\
            0.634  1.6968072763506405  \\
            1.386  1.5717865815764147  \\
            2.908  1.5279581291076163  \\
            2.991  1.5254533506238521  \\
            1.793  1.554657385885228  \\
            2.045  1.5476294033432034  \\
            0.451  1.6976812112639155  \\
            3.365  1.521170368388974  \\
            3.214  1.5215081760208524  \\
            1.509  1.5653619313535345  \\
            2.064  1.5470915566161845  \\
            2.241  1.5430860010969192  \\
            1.029  1.6062934237657163  \\
            3.451  1.5210817672192238  \\
            1.986  1.5490650019673957  \\
            3.078  1.5229076738286556  \\
            2.06  1.5471674620782627  \\
            1.689  1.5580897256135082  \\
            1.303  1.5772979779718341  \\
            1.131  1.5926640349997727  \\
            1.329  1.5756190742215672  \\
            2.717  1.5328052421494833  \\
            1.604  1.5611562260993326  \\
            1.88  1.5520484123108604  \\
            0.584  1.6971466925419865  \\
            2.914  1.527911979359435  \\
            2.494  1.537697784059169  \\
            0.811  1.6650624166084345  \\
            2.589  1.5355989584941443  \\
            1.827  1.5534796547736258  \\
            1.187  1.586993372724944  \\
            2.365  1.5404721016169285  \\
            2.541  1.5366675880303895  \\
            1.838  1.5531486200657802  \\
            3.031  1.5241810652089227  \\
            2.802  1.530692491145462  \\
            2.774  1.531434403308766  \\
            3.256  1.521366628949725  \\
            0.72  1.6943051626262025  \\
            1.794  1.5545684661258634  \\
            3.083  1.52271292857472  \\
            2.834  1.5299500004784543  \\
            0.194  1.6978050276379528  \\
            1.349  1.5740762904084382  \\
            0.877  1.640004577734287  \\
            1.552  1.5634212168062185  \\
            2.119  1.545922588493572  \\
            1.494  1.5658249082961415  \\
            3.081  1.5228869635121547  \\
            1.86  1.55259698747228  \\
            1.848  1.5529221258646846  \\
            0.217  1.6978781636255855  \\
            0.225  1.6978495568303842  \\
            2.127  1.5456812956911041  \\
            0.517  1.6973332628715354  \\
            1.602  1.5613475789862516  \\
            0.151  1.6978801834456263  \\
            1.919  1.550938231195645  \\
            3.165  1.521776421478281  \\
            2.411  1.5394112029871903  \\
            1.342  1.5745162446964618  \\
            0.71  1.695348855938402  \\
            0.331  1.697751445331734  \\
            2.892  1.5284357880018908  \\
            1.365  1.5731521157552495  \\
            1.952  1.5499299300492115  \\
            1.324  1.5759936109276806  \\
            1.996  1.548805042768415  \\
            3.107  1.5223966168974308  \\
            3.393  1.521130603256748  \\
            2.786  1.5312547089282238  \\
            0.193  1.6978117309463205  \\
            1.565  1.5629033433958441  \\
            2.906  1.527992815675149  \\
            0.548  1.6974029763043958  \\
            1.025  1.607109774957968  \\
            3.16  1.5218096874243707  \\
            0.104  1.6978476114290475  \\
            2.029  1.5480607090448606  \\
            1.665  1.558848707413183  \\
            2.432  1.5389761669444588  \\
            2.974  1.525985348885176  \\
            3.21  1.5215012015175065  \\
            0.065  1.697885963717063  \\
            0.384  1.6976597127183153  \\
            3.09  1.5226138320310905  \\
            1.936  1.5504899601253803  \\
            0.319  1.697753566269565  \\
            0.805  1.6668755194407292  \\
            3.401  1.5211238554319495  \\
            0.454  1.6976193119582825  \\
            2.457  1.5384756953348047  \\
            3.121  1.522097108855042  \\
            1.905  1.5512766208625433  \\
            3.261  1.5213371912172122  \\
            0.315  1.697825971719201  \\
            1.235  1.5822891489471682  \\
            0.582  1.6970346020121294  \\
            0.466  1.6977091034401288  \\
            2.46  1.5384182474893042  \\
            2.056  1.5473737272504524  \\
            1.214  1.584247070500888  \\
            2.664  1.5340343734445256  \\
            0.56  1.6974387505750774  \\
            2.409  1.539371789432974  \\
            2.53  1.5369194347880197  \\
            0.57  1.6973953778387871  \\
            2.864  1.5292012563153958  \\
            2.171  1.5445931517601443  \\
            2.325  1.541205563101863  \\
            1.089  1.5978224194262187  \\
            2.268  1.5425149371266451  \\
            1.272  1.5795085812049963  \\
            1.284  1.578814837309793  \\
            0.947  1.6215090629012598  \\
            1.502  1.5657064519768835  \\
            2.138  1.5454144068185196  \\
            1.442  1.5686496042649283  \\
            2.522  1.5370344218481609  \\
            3.221  1.5214581235282265  \\
            2.206  1.5439049405139431  \\
            0.691  1.6960293745848278  \\
            1.598  1.5613059002667566  \\
            2.643  1.5344483114736822  \\
            0.124  1.6979054625848218  \\
            1.428  1.5694151664423845  \\
            0.214  1.69784500919166  \\
            1.549  1.5635967621508315  \\
            1.375  1.5726245780155435  \\
            1.523  1.5649233867711831  \\
            1.544  1.5637197280347739  \\
            2.941  1.5269852668407826  \\
            1.725  1.556737976890174  \\
            0.097  1.6979055484941727  \\
            1.085  1.5983977273254408  \\
            1.668  1.5587308831460553  \\
            0.26  1.6977958446832841  \\
            1.402  1.5709737397771268  \\
            1.661  1.5591301777502737  \\
            2.251  1.5429356664962885  \\
            0.903  1.6317751152306115  \\
            0.276  1.697768312425711  \\
            0.201  1.6978268067760152  \\
            1.091  1.597890919233349  \\
            2.926  1.527412364956751  \\
            1.077  1.5993886342161658  \\
            2.811  1.5305827397673235  \\
            0.604  1.6970204280839232  \\
            2.835  1.5299293154886464  \\
            1.064  1.6007800497574978  \\
            0.988  1.6133760114324953  \\
            1.438  1.5689181750154515  \\
            2.486  1.537910115938043  \\
            1.818  1.5537692543446198  \\
            2.266  1.5425643332533105  \\
            3.097  1.5225478741269494  \\
            1.537  1.564029764596791  \\
            0.396  1.6977493773537242  \\
            0.03  1.697838208092438  \\
            1.497  1.5658780450573044  \\
            1.97  1.5494636885627546  \\
            0.17  1.6979024399012255  \\
            2.316  1.541442417422526  \\
            2.059  1.5472604055719492  \\
            1.772  1.5552322086633936  \\
            1.587  1.561903402447206  \\
            1.68  1.558241568907984  \\
            3.188  1.5216099009052084  \\
            0.392  1.6976769452401639  \\
            2.921  1.5275643914912627  \\
            1.654  1.559245408313064  \\
            1.468  1.5674529822725973  \\
            1.946  1.5501161102761685  \\
            2.948  1.526827485864853  \\
            2.221  1.5434339837396691  \\
            2.042  1.54779428193122  \\
            1.045  1.6038025809887644  \\
            2.329  1.541240916382212  \\
            1.406  1.5707107579351636  \\
            2.312  1.5414925174336178  \\
            2.709  1.532991962143459  \\
            0.08  1.697813070309623  \\
            1.082  1.599261891185302  \\
            1.78  1.5548421656950058  \\
            1.536  1.5640636677025845  \\
            0.932  1.625801606198179  \\
            1.662  1.5589420280785358  \\
            2.21  1.543781034806132  \\
            0.505  1.69752705132427  \\
            1.778  1.5550483279898275  \\
            3.094  1.5225895440816455  \\
            0.212  1.6978261673704123  \\
            1.708  1.5573461888293842  \\
            1.815  1.553895805638032  \\
            1.506  1.565491394548699  \\
            3.066  1.5232581057318901  \\
            1.116  1.5946749605120027  \\
            2.966  1.5262415866872807  \\
            0.847  1.6486091984332396  \\
            2.849  1.5295600089856765  \\
            1.361  1.573310783133947  \\
            3.426  1.5210901933504297  \\
            1.449  1.5682831584386172  \\
            2.943  1.5270406819383764  \\
            1.388  1.5718275470320806  \\
            2.327  1.5412251529335776  \\
            3.434  1.5210948494647432  \\
            0.973  1.6168245102073926  \\
            0.731  1.6927981216803827  \\
            2.334  1.541063704266092  \\
            0.931  1.6256125635558558  \\
            2.878  1.5288432354043857  \\
            3.075  1.5228738797820671  \\
            2.126  1.545673889884457  \\
            1.058  1.6022116923484757  \\
            0.364  1.697779471215549  \\
            1.212  1.5844406578014922  \\
            1.377  1.5724386127713683  \\
            3.166  1.5217510287094902  \\
            0.749  1.6909268439182312  \\
            2.306  1.541638165629167  \\
            3.255  1.521366675115267  \\
            2.355  1.5406197384644986  \\
            1.241  1.582212008473275  \\
            1.917  1.550916313480175  \\
            2.675  1.5337324874777034  \\
            0.501  1.6974657069737509  \\
            1.434  1.5689748250655207  \\
            0.163  1.6978821972330198  \\
            3.076  1.5228563069834087  \\
            1.041  1.605355188945471  \\
            0.404  1.6976842302198292  \\
            2.081  1.5467848659196788  \\
            3.334  1.521209036942483  \\
            1.809  1.5541096122577664  \\
            1.012  1.6092820501267113  \\
            2.143  1.5452504312347863  \\
            0.7  1.6954105768057448  \\
            1.013  1.6090813416361458  \\
            2.569  1.5361273579714783  \\
            1.741  1.556291868518678  \\
            2.3  1.5418301984313323  \\
            1.445  1.5687099777650657  \\
            2.101  1.5462338699570868  \\
            1.418  1.5700863418762498  \\
            2.323  1.5412850485892593  \\
            0.618  1.6972522321978136  \\
            1.0  1.6108701821824445  \\
            0.901  1.6329876970649224  \\
            1.576  1.562356936388552  \\
            2.005  1.5486793815883746  \\
            0.823  1.657914521346665  \\
            3.471  1.5210696885965151  \\
            1.75  1.555879896846175  \\
            2.839  1.5298473840075488  \\
            3.162  1.5218008257660012  \\
            0.659  1.6963439153264315  \\
            2.226  1.5434301540242072  \\
            3.369  1.5211562766677784  \\
            1.248  1.581541479340371  \\
            0.07  1.6979348708345219  \\
            0.924  1.626993981361703  \\
            2.28  1.5422405097227818  \\
            1.022  1.6076105529240008  \\
            0.485  1.6976188516663275  \\
            2.918  1.5277125089649137  \\
            0.368  1.6977349764272736  \\
            1.283  1.5786268728523696  \\
            2.54  1.5366412048040468  \\
            2.466  1.538281363679518  \\
            2.775  1.531472150571935  \\
            2.815  1.5303836800870572  \\
            1.006  1.6106272139707323  \\
            3.274  1.521325002094776  \\
            1.879  1.5519266181695757  \\
            0.515  1.6975016376450953  \\
            0.543  1.6974124696185158  \\
            3.03  1.5242508761179998  \\
            1.605  1.5612040790504234  \\
            0.263  1.6978118432304043  \\
            0.824  1.6584251971191577  \\
            3.286  1.5212681531047987  \\
            1.244  1.58170976991302  \\
            0.801  1.6676328910910685  \\
            2.454  1.5385120075934466  \\
            0.154  1.697892796701979  \\
            1.938  1.5503244311925395  \\
            2.296  1.5418816297480398  \\
            1.666  1.5588477412712407  \\
            2.47  1.5381631405586706  \\
            0.23  1.6977971740598579  \\
            2.43  1.5390643895954037  \\
            0.136  1.6978654596538867  \\
            2.978  1.525863884678239  \\
            0.45  1.6975844779610763  \\
            2.979  1.5258245697648842  \\
            3.002  1.5250537780460482  \\
            3.022  1.5244698785502995  \\
            3.046  1.5236095176118265  \\
            3.425  1.521100616278863  \\
            2.242  1.5431269099606173  \\
            0.465  1.6975839598633853  \\
            0.342  1.697809146262311  \\
            0.028  1.6978406628930354  \\
            1.463  1.5675491600243951  \\
            1.02  1.607835403350329  \\
            3.009  1.5249158590642127  \\
            1.832  1.5532730573364473  \\
            0.298  1.697810202176767  \\
            3.459  1.521085851400805  \\
            1.698  1.5577416016029317  \\
            2.935  1.5272355155518966  \\
            1.306  1.577256335160004  \\
            0.094  1.69785653966815  \\
            2.707  1.5330363588655096  \\
            0.179  1.697850642846427  \\
            2.359  1.5405179697589402  \\
            0.82  1.66117816635925  \\
            0.696  1.6955778182638108  \\
            0.005  1.6978700561089335  \\
            0.737  1.6897320021724829  \\
            1.27  1.5798409120869723  \\
            0.053  1.6979142373022553  \\
            2.946  1.5268913963129973  \\
            0.554  1.6975358791702664  \\
            2.804  1.5306967187149325  \\
            0.798  1.6704391561082241  \\
            0.945  1.6219897912140597  \\
            0.05  1.6978423245701495  \\
            1.771  1.5552487072937669  \\
            0.069  1.6979313702313168  \\
            3.105  1.5222521401121527  \\
            0.729  1.6931356383377967  \\
            1.176  1.5879918035083633  \\
            1.567  1.562770495616972  \\
            3.042  1.5238514071769618  \\
            1.588  1.5619291176374535  \\
            1.474  1.5671205995039181  \\
            0.098  1.6978708995926661  \\
            0.024  1.6978797634577958  \\
            2.725  1.5325962955300676  \\
            2.938  1.527092984512264  \\
            0.493  1.6976191991943215  \\
            0.254  1.697872967479066  \\
            2.437  1.5389272827968885  \\
            0.712  1.6941295339783486  \\
            0.948  1.6212578377170033  \\
            0.271  1.697804462259931  \\
            0.904  1.6324527699020726  \\
            2.592  1.5355658073053833  \\
            0.792  1.6729652765210028  \\
            1.839  1.5531747402645326  \\
            2.548  1.536540690794725  \\
            0.421  1.697667157768313  \\
            0.651  1.6966725049664613  \\
            1.965  1.549652490436468  \\
            3.462  1.5210728907277378  \\
            1.716  1.557195137407628  \\
            0.304  1.6977262864943379  \\
            2.581  1.5357752878214952  \\
            1.165  1.5892251833612026  \\
            2.084  1.5466783710428496  \\
            1.462  1.567683911832258  \\
            2.885  1.5286666022493622  \\
            0.91  1.6309421873697658  \\
            0.034  1.6979375226398559  \\
            1.488  1.5663885261957071  \\
            1.562  1.5630297319299569  \\
            1.033  1.6061509051742726  \\
            3.297  1.521249879199566  \\
            1.639  1.5598938086910232  \\
            1.2  1.585804644799393  \\
            1.978  1.549311604490359  \\
            3.026  1.5242297157890274  \\
            0.207  1.6978955913633285  \\
            3.101  1.5224840740026861  \\
            2.434  1.5389313419818471  \\
            2.925  1.5275562876389301  \\
            2.865  1.5291088305083607  \\
            1.655  1.5593742499420211  \\
            1.769  1.555403092943344  \\
            3.011  1.5247924723331379  \\
            2.78  1.5312956677586882  \\
            1.791  1.5546096530306104  \\
            2.207  1.5438509857310443  \\
            1.256  1.5809410033618954  \\
            2.87  1.5290180157508888  \\
            3.01  1.5249175468202443  \\
            0.145  1.6978649156101346  \\
            2.81  1.5305322721822874  \\
            2.381  1.5400877530993051  \\
            1.392  1.5714618964545208  \\
            2.972  1.5260428809553026  \\
            1.616  1.5606453459257292  \\
            3.155  1.5217925697948804  \\
            1.687  1.5579666747846628  \\
            1.206  1.585223539280746  \\
            0.867  1.6424904999302539  \\
            0.702  1.6953691318224946  \\
            3.244  1.5213991632439976  \\
            3.159  1.5218454276772413  \\
            1.318  1.576390232806901  \\
            3.218  1.5214805248953414  \\
            3.074  1.5229191233658042  \\
            3.427  1.5211021411448449  \\
            3.388  1.5211284520966042  \\
            0.156  1.6978208277945024  \\
            0.809  1.6662125787426096  \\
            3.291  1.5212716887234456  \\
            3.49  1.5210400124685184  \\
            3.088  1.5227388814740266  \\
            1.856  1.5526959799426452  \\
            2.821  1.5303226506146947  \\
            3.082  1.5228465525466173  \\
            0.385  1.697675517997885  \\
            1.456  1.5679652420944004  \\
            1.128  1.593017225039166  \\
            0.448  1.6976802581966424  \\
            0.716  1.6952945155446753  \\
            0.759  1.6853673193662686  \\
            1.634  1.5600802843051549  \\
            2.011  1.5483933433657453  \\
            1.52  1.5647663442189526  \\
            3.038  1.5240339204837892  \\
            3.167  1.5217202624857415  \\
            1.135  1.5922742458908465  \\
            0.386  1.6977825401740296  \\
            1.358  1.5736498101649556  \\
            2.869  1.529018760790974  \\
            1.119  1.594681463680882  \\
            2.469  1.5381836968999847  \\
            0.865  1.6429309517595874  \\
            0.859  1.6458347679143412  \\
            2.778  1.531327746972352  \\
            1.285  1.5788369044006154  \\
            0.769  1.6823742627751987  \\
            1.36  1.5734665568693047  \\
            1.075  1.600121888459958  \\
            2.858  1.5293787639073233  \\
            1.792  1.5545668964331087  \\
            3.245  1.5213897615444063  \\
            2.826  1.5301460184788909  \\
            2.924  1.5274312683658087  \\
            3.372  1.5211574870611966  \\
            0.679  1.6962006695513525  \\
            1.64  1.5598811418213088  \\
            2.976  1.525901832066778  \\
            0.148  1.697886545017793  \\
            1.901  1.551382810369618  \\
            2.886  1.5286478248933912  \\
            0.267  1.6977687964566308  \\
            0.991  1.6131242558161356  \\
            1.014  1.6091274381889293  \\
            2.408  1.5395029601596806  \\
            2.657  1.53417155650767  \\
            2.402  1.5396004849663785  \\
            1.161  1.5896037118110982  \\
            2.836  1.529949592589764  \\
            2.694  1.5332607091911028  \\
            2.483  1.5379165508731603  \\
            3.13  1.5220049874072996  \\
            2.931  1.527370781776193  \\
            2.112  1.5459697282391518  \\
            0.688  1.6961883915917517  \\
            0.695  1.6947879001280215  \\
            0.803  1.6694673110057145  \\
            0.232  1.6978310222989914  \\
            2.153  1.5450783293621744  \\
            0.595  1.6971269125963868  \\
            1.446  1.5684242380584268  \\
            2.456  1.5384470332060478  \\
            0.732  1.6932922525322764  \\
            2.531  1.5368792005373437  \\
            1.036  1.6049204315004724  \\
            1.115  1.5949376211711979  \\
            1.343  1.5745800533982928  \\
            1.807  1.5540373849731244  \\
            3.44  1.5210898569015388  \\
            0.419  1.6977368059866504  \\
            0.751  1.6902387988684722  \\
            2.419  1.5392807702043565  \\
            0.129  1.6979120233646068  \\
            1.5  1.5658195561672064  \\
            0.203  1.6978931654285252  \\
            0.09  1.697929789183246  \\
            3.239  1.5214232685601112  \\
            3.294  1.521263311465309  \\
            2.958  1.5265576389084685  \\
            2.727  1.532546019216206  \\
            2.302  1.5417304357336394  \\
            1.758  1.555708278231537  \\
            3.136  1.5220054166293628  \\
            2.854  1.5294048020156714  \\
            1.937  1.550371789843033  \\
            1.997  1.548907898562177  \\
            0.744  1.6892250988008644  \\
            2.31  1.5416030536913383  \\
            0.286  1.6978217367411366  \\
            1.98  1.549264190939001  \\
            1.579  1.5621389654260593  \\
            3.208  1.5215289989956764  \\
            3.48  1.521057427861469  \\
            2.49  1.5377596591867089  \\
            3.349  1.5211918549075332  \\
            2.422  1.5391997695657267  \\
            1.047  1.6036042616377095  \\
            0.872  1.642043948617824  \\
            3.126  1.5220765625001413  \\
            1.744  1.556191777988792  \\
            0.997  1.6121932466202524  \\
            3.339  1.5212123240547357  \\
            1.144  1.5914270196707103  \\
            1.399  1.5712000190100721  \\
            0.481  1.697557955741282  \\
            2.304  1.5416643639155834  \\
            2.715  1.5327922567007488  \\
            2.797  1.5308706690342626  \\
            2.911  1.5279385144940119  \\
            1.254  1.5812304230516163  \\
            0.088  1.6978735555295068  \\
            3.429  1.5211017241634774  \\
            2.109  1.5460391477835227  \\
            3.005  1.5250166907032352  \\
            2.317  1.5413454772014017  \\
            1.824  1.5535679957376305  \\
            2.533  1.536888333989996  \\
            1.614  1.5608424619814034  \\
            1.571  1.5625778068412528  \\
            0.064  1.6978869732003554  \\
            1.762  1.5554615730188324  \\
            2.646  1.534390703259705  \\
            0.233  1.697834314791265  \\
            1.645  1.559516677580756  \\
            3.143  1.5219710318485657  \\
            1.43  1.5693053466787095  \\
            2.842  1.5297431711698017  \\
            0.416  1.6977261500834928  \\
            2.688  1.533447795786929  \\
            0.905  1.6307551400668687  \\
            2.932  1.527198946240145  \\
            0.377  1.697765058056628  \\
            0.933  1.6248224254452384  \\
            2.825  1.5302004247358052  \\
            1.078  1.5988272482881165  \\
            0.799  1.671210707984645  \\
            2.563  1.5361834965538956  \\
            0.353  1.6977459807138588  \\
            2.939  1.5271152241785428  \\
            0.768  1.6871588832789417  \\
            0.36  1.6977819406924044  \\
            1.491  1.5663778253108953  \\
            2.55  1.536478432772036  \\
            3.284  1.521291585881677  \\
            1.295  1.5778254773991875  \\
            0.126  1.697907460692266  \\
            2.67  1.5338378629029312  \\
            1.314  1.5763784615288459  \\
            1.775  1.555122585137418  \\
            3.26  1.5213633498397863  \\
            1.441  1.5688286916345389  \\
            2.155  1.5450142924518575  \\
            1.348  1.5740737106758136  \\
            1.527  1.5646027876876423  \\
            2.187  1.5442562787228142  \\
            1.817  1.5536912132231238  \\
            2.214  1.5437200274060483  \\
            2.5  1.5374826593443225  \\
            0.327  1.697748688459717  \\
            1.389  1.5716372912442136  \\
            2.86  1.5292774744832744  \\
            2.828  1.530156006832045  \\
            2.12  1.5458548010811382  \\
            2.386  1.539908508604376  \\
            2.429  1.5390259728313025  \\
            1.755  1.5556504440962242  \\
            0.523  1.697475391286635  \\
            0.369  1.69768211663299  \\
            0.899  1.6339718159081127  \\
            0.53  1.697596508878673  \\
            2.324  1.5413216378361774  \\
        }
        ;
    \addplot[color={rgb,1:red,0.8889;green,0.4356;blue,0.2781}, name path={2499f335-0794-4197-b9df-7143ae347687}, draw opacity={0.7}, line width={1}, solid]
        table[row sep={\\}]
        {
            \\
            0.73  1.66  \\
            0.73  1.72  \\
        }
        ;
\end{axis}
\end{tikzpicture}

    \caption{Zależność średniej wartości efektywnego współczynnika załamania światła w komórce NLC od natężenia zewnętrznego pola elektrycznego w jednostkach zredukowanych. Próg Freedericksza odczytano jako $E_F^*=0.73$ i zaznaczono na wykresie.}
\end{figure}
\begin{figure}
    \centering
    \begin{tikzpicture}[/tikz/background rectangle/.style={fill={rgb,1:red,1.0;green,1.0;blue,1.0}, draw opacity={1.0}}, show background rectangle]
\begin{axis}[point meta max={nan}, point meta min={nan}, legend cell align={left}, title={Ensemble averaged particles' orientation profile}, title style={at={{(0.5,1)}}, anchor={south}, font={{\fontsize{14 pt}{18.2 pt}\selectfont}}, color={rgb,1:red,0.0;green,0.0;blue,0.0}, draw opacity={1.0}, rotate={0.0}}, legend style={color={rgb,1:red,0.0;green,0.0;blue,0.0}, draw opacity={1.0}, line width={1}, solid, fill={rgb,1:red,1.0;green,1.0;blue,1.0}, fill opacity={1.0}, text opacity={1.0}, font={{\fontsize{8 pt}{10.4 pt}\selectfont}}, text={rgb,1:red,0.0;green,0.0;blue,0.0}, at={(1.02, 1)}, anchor={north west}}, axis background/.style={fill={rgb,1:red,1.0;green,1.0;blue,1.0}, opacity={1.0}}, anchor={north west}, xshift={1.0mm}, yshift={-1.0mm}, width={150.4mm}, height={99.6mm}, scaled x ticks={false}, xlabel={z}, x tick style={color={rgb,1:red,0.0;green,0.0;blue,0.0}, opacity={1.0}}, x tick label style={color={rgb,1:red,0.0;green,0.0;blue,0.0}, opacity={1.0}, rotate={0}}, xlabel style={at={(ticklabel cs:0.5)}, anchor=near ticklabel, font={{\fontsize{11 pt}{14.3 pt}\selectfont}}, color={rgb,1:red,0.0;green,0.0;blue,0.0}, draw opacity={1.0}, rotate={0.0}}, xmajorgrids={true}, xmin={0.43000000000000005}, xmax={20.57}, xtick={{1.0,2.0,3.0,4.0,5.0,6.0,7.0,8.0,9.0,10.0,11.0,12.0,13.0,14.0,15.0,16.0,17.0,18.0,19.0,20.0}}, xticklabels={{$1$,$2$,$3$,$4$,$5$,$6$,$7$,$8$,$9$,$10$,$11$,$12$,$13$,$14$,$15$,$16$,$17$,$18$,$19$,$20$}}, xtick align={inside}, xticklabel style={font={{\fontsize{8 pt}{10.4 pt}\selectfont}}, color={rgb,1:red,0.0;green,0.0;blue,0.0}, draw opacity={1.0}, rotate={0.0}}, x grid style={color={rgb,1:red,0.0;green,0.0;blue,0.0}, draw opacity={0.1}, line width={0.5}, solid}, axis x line*={left}, x axis line style={color={rgb,1:red,0.0;green,0.0;blue,0.0}, draw opacity={1.0}, line width={1}, solid}, scaled y ticks={false}, ylabel={<$\cos^2(\phi)$>}, y tick style={color={rgb,1:red,0.0;green,0.0;blue,0.0}, opacity={1.0}}, y tick label style={color={rgb,1:red,0.0;green,0.0;blue,0.0}, opacity={1.0}, rotate={0}}, ylabel style={at={(ticklabel cs:0.5)}, anchor=near ticklabel, font={{\fontsize{11 pt}{14.3 pt}\selectfont}}, color={rgb,1:red,0.0;green,0.0;blue,0.0}, draw opacity={1.0}, rotate={0.0}}, ymajorgrids={true}, ymin={0.4806585834233849}, ymax={1.0151264490265033}, ytick={{0.5,0.6000000000000001,0.7000000000000001,0.8,0.9,1.0}}, yticklabels={{$0.5$,$0.6$,$0.7$,$0.8$,$0.9$,$1.0$}}, ytick align={inside}, yticklabel style={font={{\fontsize{8 pt}{10.4 pt}\selectfont}}, color={rgb,1:red,0.0;green,0.0;blue,0.0}, draw opacity={1.0}, rotate={0.0}}, y grid style={color={rgb,1:red,0.0;green,0.0;blue,0.0}, draw opacity={0.1}, line width={0.5}, solid}, axis y line*={left}, y axis line style={color={rgb,1:red,0.0;green,0.0;blue,0.0}, draw opacity={1.0}, line width={1}, solid}]
    \addplot[color={rgb,1:red,0.0;green,0.6056;blue,0.9787}, name path={1a6d3d02-746e-4986-9e9c-2d36338d72c8}, only marks, draw opacity={1.0}, line width={0}, solid, mark={*}, mark size={3.0 pt}, mark repeat={1}, mark options={color={rgb,1:red,0.0;green,0.0;blue,0.0}, draw opacity={1.0}, fill={rgb,1:red,0.0;green,0.6056;blue,0.9787}, fill opacity={1.0}, line width={0.75}, rotate={0}, solid}]
        table[row sep={\\}]
        {
            \\
            1.0  1.0  \\
            2.0  0.9744945718004556  \\
            3.0  0.9198033488748552  \\
            4.0  0.843389153926221  \\
            5.0  0.7591692245461917  \\
            6.0  0.6764760448156835  \\
            7.0  0.6052245928975541  \\
            8.0  0.5511531111347328  \\
            9.0  0.5141033880601491  \\
            10.0  0.4960208869500432  \\
            11.0  0.49578503244988825  \\
            12.0  0.5159940065489305  \\
            13.0  0.5533438940363196  \\
            14.0  0.608319636598808  \\
            15.0  0.6793809730787009  \\
            16.0  0.7616748438381139  \\
            17.0  0.8451943920680111  \\
            18.0  0.9205778658035556  \\
            19.0  0.9749926051517969  \\
            20.0  1.0  \\
        }
        ;
\end{axis}
\end{tikzpicture}

    \caption{Profil orientacji cząsteczek NLC (uśredniony po zespole statystycznym profil $<\cos^2(\phi)>(z)$, gdzie $z$ jest numerem rzędu) dla wartości natężenia pola $E^*=1.2 E_F^* = 0.876$}
\end{figure}
% cspell:Disable
\newpage
\section*{Kod programu - Julia}
\begin{minted}[breaklines,escapeinside=||,mathescape=true, linenos, numbersep=3pt, gobble=2, frame=lines, fontsize=\small, framesep=2mm]{julia}
using Statistics
using OffsetArrays
using JLD2
using Plots
pgfplotsx()

const L = 20
MCS = 230_000

const Δϕ = 10 #° 
# const T = 2
const ξ = 20.0
const n₀ = 1.5
const nₑ = 1.7

const P_func(β) = (3 * cosd(β)^2 -1)/2 
const P₂ = OffsetArray(map(P_func, -180:180), -180:180)
Iₙ = collect(2:L+1)
Iₚ = collect(0:L-1)
Iₙ[20] = 1
Iₚ[1] = L

const randoms_Δϕ = Int16.(cat(-Δϕ÷2:-1, 1:Δϕ÷2, dims=1))

n_eff(φ::Number, n₀ = n₀, nₑ = nₑ)::Float64 = n₀ * nₑ / √(n₀^2*cosd(φ)^2 + nₑ^2*sind(φ)^2)

function nlc(E::Number, MCS::Integer=230_000, L::Integer=20, skipfirst::Integer=30_000, probe_every::Integer=100, P₂=P₂, Iₙ=Iₙ, Iₚ=Iₚ, ξ=ξ, randoms_Δϕ=randoms_Δϕ)::Float64
    #ordered initial conditions - all ϕ=0°
    ϕ = zeros(Int16,L,L)
    n_eff_sum = 0.0

    for k ∈ 1:MCS
        for j ∈ 1:L, i ∈ 2:L-1
            #metropolis algorithm
            @inbounds ϕ_new::Int16 = ϕ[i,j] + rand(randoms_Δϕ)
            ϕ_new > 90  && (ϕ_new -= 180) # a && b  => if a then run b
            ϕ_new < -90 && (ϕ_new += 180)

            @inbounds U_old::Float64 = -ξ * (P₂[ϕ[i,j] - ϕ[Iₙ[i],j]] + P₂[ϕ[i,j] - ϕ[Iₚ[i],j]] + P₂[ϕ[i,j] - ϕ[i,Iₙ[j]]] + P₂[ϕ[i,j] - ϕ[i,Iₚ[j]]]) - E^2 * P₂[90-ϕ[i,j]]
            
            @inbounds U_new::Float64 = -ξ * (P₂[ϕ_new - ϕ[Iₙ[i],j]] + P₂[ϕ_new - ϕ[Iₚ[i],j]] + P₂[ϕ_new - ϕ[i,Iₙ[j]]] + P₂[ϕ_new - ϕ[i,Iₚ[j]]]) - E^2 * P₂[90-ϕ_new]

            ΔU::Float64 = U_new - U_old
            if ΔU < 0 || rand() ≤ exp(-ΔU)
                @inbounds ϕ[i,j] = ϕ_new
            end
        end
        if k > skipfirst && k%probe_every == 0
            n_eff_sum += sum(n_eff, ϕ)/L^2
        end
    end
    return n_eff_sum / ((MCS-skipfirst)÷probe_every) #return mean <n_eff>
    #return ϕ
end

function nlc_cos2_profile(E::Number, MCS::Integer=230_000, L::Integer=20, skipfirst::Integer=30_000, probe_every::Integer=100, P₂=P₂, Iₙ=Iₙ, Iₚ=Iₚ, ξ=ξ, randoms_Δϕ=randoms_Δϕ)::Array{Float64}
    #ordered initial conditions - all ϕ=0°
    ϕ = zeros(Int16,L,L)
    cos2_profile = zeros(L)

    for k ∈ 1:MCS
        for j ∈ 1:L, i ∈ 2:L-1
            #metropolis algorithm
            @inbounds ϕ_new::Int16 = ϕ[i,j] + rand(randoms_Δϕ)
            ϕ_new > 90  && (ϕ_new -= 180) # a && b  => if a then run b
            ϕ_new < -90 && (ϕ_new += 180)

            @inbounds U_old::Float64 = -ξ * (P₂[ϕ[i,j] - ϕ[Iₙ[i],j]] + P₂[ϕ[i,j] - ϕ[Iₚ[i],j]] + P₂[ϕ[i,j] - ϕ[i,Iₙ[j]]] + P₂[ϕ[i,j] - ϕ[i,Iₚ[j]]]) - E^2 * P₂[90-ϕ[i,j]]
            
            @inbounds U_new::Float64 = -ξ * (P₂[ϕ_new - ϕ[Iₙ[i],j]] + P₂[ϕ_new - ϕ[Iₚ[i],j]] + P₂[ϕ_new - ϕ[i,Iₙ[j]]] + P₂[ϕ_new - ϕ[i,Iₚ[j]]]) - E^2 * P₂[90-ϕ_new]

            ΔU::Float64 = U_new - U_old
            if ΔU < 0 || rand() ≤ exp(-ΔU)
                @inbounds ϕ[i,j] = ϕ_new
            end
        end
        if k > skipfirst && k%probe_every == 0
            cos2_profile += mean(x -> cosd(x)^2, ϕ, dims=2)
        end
    end
    return cos2_profile ./ ((MCS-skipfirst)÷probe_every)
end

function run_nlc(Es)
    result = Dict{Float64,Float64}()
    Threads.@threads for E ∈ Es
        result[E] = @time nlc(E)
    end
    return result
end

Es = 0:0.001:3.5

result_ret = @time run_nlc(Es)
@save "result_002.jld2" result_ret
@load "result_002.jld2" result_ret

# serialize("result_002.jls", result)

scatter(result_ret, markershape=:auto, markersize=2, xlabel="\$E^*\$", ylabel="\$n_{eff}\$",
        title="Effective refracting index as a function of reduced external electric field - ordered initial conditions",
             titlefontsize=8, markerstrokealpha=0.0, label="L=20",  legend=true)
lens!([0.65,0.8],[1.68,1.7], minorticks=5, minorgrid=true, legend=false, inset = (1, bbox(0.6, 0.1, 0.3, 0.5)))
savefig("neff_E.tex")

#@save "result.jld2" result_ret
#@load "result.jld2"

using Profile
using StatProfilerHTML

# Profile.clear()
# result_ret = @@profilehtml run_nlc(Es)


cos2_profile = nlc_cos2_profile(1.2*0.73)
scatter(cos2_profile, grid=:both, xticks=1:20, xlabel="z", ylabel="<\$\\cos^2(\\phi)\$>", legend=:none,
        title="Ensemble averaged particles' orientation profile")
savefig("cos2_profile.tex")
\end{minted}


\end{document}